%% Generated by Sphinx.
\def\sphinxdocclass{report}
\documentclass[letterpaper,10pt,english]{sphinxmanual}
\ifdefined\pdfpxdimen
   \let\sphinxpxdimen\pdfpxdimen\else\newdimen\sphinxpxdimen
\fi \sphinxpxdimen=.75bp\relax

\PassOptionsToPackage{warn}{textcomp}
\usepackage[utf8]{inputenc}
\ifdefined\DeclareUnicodeCharacter
% support both utf8 and utf8x syntaxes
  \ifdefined\DeclareUnicodeCharacterAsOptional
    \def\sphinxDUC#1{\DeclareUnicodeCharacter{"#1}}
  \else
    \let\sphinxDUC\DeclareUnicodeCharacter
  \fi
  \sphinxDUC{00A0}{\nobreakspace}
  \sphinxDUC{2500}{\sphinxunichar{2500}}
  \sphinxDUC{2502}{\sphinxunichar{2502}}
  \sphinxDUC{2514}{\sphinxunichar{2514}}
  \sphinxDUC{251C}{\sphinxunichar{251C}}
  \sphinxDUC{2572}{\textbackslash}
\fi
\usepackage{cmap}
\usepackage[T1]{fontenc}
\usepackage{amsmath,amssymb,amstext}
\usepackage{babel}



\usepackage{times}
\expandafter\ifx\csname T@LGR\endcsname\relax
\else
% LGR was declared as font encoding
  \substitutefont{LGR}{\rmdefault}{cmr}
  \substitutefont{LGR}{\sfdefault}{cmss}
  \substitutefont{LGR}{\ttdefault}{cmtt}
\fi
\expandafter\ifx\csname T@X2\endcsname\relax
  \expandafter\ifx\csname T@T2A\endcsname\relax
  \else
  % T2A was declared as font encoding
    \substitutefont{T2A}{\rmdefault}{cmr}
    \substitutefont{T2A}{\sfdefault}{cmss}
    \substitutefont{T2A}{\ttdefault}{cmtt}
  \fi
\else
% X2 was declared as font encoding
  \substitutefont{X2}{\rmdefault}{cmr}
  \substitutefont{X2}{\sfdefault}{cmss}
  \substitutefont{X2}{\ttdefault}{cmtt}
\fi


\usepackage[Bjarne]{fncychap}
\usepackage{sphinx}

\fvset{fontsize=\small}
\usepackage{geometry}


% Include hyperref last.
\usepackage{hyperref}
% Fix anchor placement for figures with captions.
\usepackage{hypcap}% it must be loaded after hyperref.
% Set up styles of URL: it should be placed after hyperref.
\urlstyle{same}


\usepackage{sphinxmessages}
\setcounter{tocdepth}{0}



\title{WESTPA Documentation}
\date{Jan 15, 2021}
\release{1.0b1}
\author{Matthew C. Zwier and Lillian T. Chong}
\newcommand{\sphinxlogo}{\vbox{}}
\renewcommand{\releasename}{Release}
\makeindex
\begin{document}

\pagestyle{empty}
\sphinxmaketitle
\pagestyle{plain}
\sphinxtableofcontents
\pagestyle{normal}
\phantomsection\label{\detokenize{sphinx_index::doc}}



\chapter{Overview}
\label{\detokenize{sphinx_index:overview}}
WESTPA is a package for constructing and running stochastic simulations using the “weighted ensemble” approach
of Huber and Kim (1996) (see \sphinxhref{https://westpa.github.io/westpa/overview.html}{overview}).

For use of WESTPA please cite the following:

Zwier, M.C., Adelman, J.L., Kaus, J.W., Pratt, A.J., Wong, K.F., Rego, N.B., Suarez, E., Lettieri, S.,
Wang, D. W., Grabe, M., Zuckerman, D. M., and Chong, L. T. “WESTPA: An Interoperable, Highly
Scalable Software Package For Weighted Ensemble Simulation and Analysis,” J. Chem. Theory Comput., 11: 800−809 (2015).

To help us fund development and improve WESTPA please fill out a one\sphinxhyphen{}minute \sphinxhref{https://docs.google.com/forms/d/e/1FAIpQLSfWaB2aryInU06cXrCyAFmhD\_gPibgOfFk-dspLEsXuS9-RGQ/viewform}{survey} and consider
contributing documentation or code to the WESTPA community.

WESTPA is free software, licensed under the terms of the GNU General Public
License, Version 3. See the file \sphinxcode{\sphinxupquote{COPYING}} for more information.


\chapter{Obtaining and Installing WESTPA}
\label{\detokenize{sphinx_index:obtaining-and-installing-westpa}}
WESTPA is developed and tested on Unix\sphinxhyphen{}like operating systems, including Linux and Mac OS X.

Before installing WESTPA, you will need to first install the Python 2.7 version provided by the latest free \sphinxhref{https://www.continuum.io/downloads}{Anaconda Python distribution}. After installing the Anaconda Python distribution, either add the Python executable to your \$PATH or set the environment variable WEST\_PYTHON:

\begin{sphinxVerbatim}[commandchars=\\\{\}]
\PYG{n}{export} \PYG{n}{WEST\PYGZus{}PYTHON}\PYG{o}{=}\PYG{o}{/}\PYG{n}{opt}\PYG{o}{/}\PYG{n}{anaconda}\PYG{o}{/}\PYG{n+nb}{bin}\PYG{o}{/}\PYG{n}{python3}
\end{sphinxVerbatim}

We recommend obtaining the latest release of WESTPA by downloading the corresponding tar.gz file from the \sphinxhref{https://github.com/westpa/westpa/releases}{releases page}. After downloading the file, unpack the file and install WESTPA by executing the following:

\begin{sphinxVerbatim}[commandchars=\\\{\}]
\PYG{n}{tar} \PYG{n}{xvzf} \PYG{n}{westpa}\PYG{o}{\PYGZhy{}}\PYG{n}{master}\PYG{o}{.}\PYG{n}{tar}\PYG{o}{.}\PYG{n}{gz}
\PYG{n}{cd} \PYG{n}{westpa}
\PYG{o}{.}\PYG{o}{/}\PYG{n}{setup}\PYG{o}{.}\PYG{n}{sh}
\end{sphinxVerbatim}

A westpa.sh script is created during installation, and will set the following environment variables:

\begin{sphinxVerbatim}[commandchars=\\\{\}]
\PYG{n}{WEST\PYGZus{}ROOT}
\PYG{n}{WEST\PYGZus{}BIN}
\PYG{n}{WEST\PYGZus{}PYTHON}
\end{sphinxVerbatim}

These environment variables must be set in order to run WESTPA on your computing cluster.

To define environment variables post\sphinxhyphen{}installation, simply source the
\sphinxcode{\sphinxupquote{westpa.sh}} script in the \sphinxcode{\sphinxupquote{westpa}} directory from the command line
or your setup scripts.


\chapter{Getting started}
\label{\detokenize{sphinx_index:getting-started}}
A Quickstart guide and tutorials are provided \sphinxhref{https://github.com/westpa/westpa/wiki}{here}.


\chapter{Getting help}
\label{\detokenize{sphinx_index:getting-help}}

\section{FAQ}
\label{\detokenize{sphinx_index:faq}}
Responses to frequently asked questions (FAQ) can be found in the following page:
\begin{itemize}
\item {} 
\sphinxhref{https://github.com/westpa/westpa/wiki/Frequently-Asked-Questions-\%28FAQ\%29}{Frequently Asked Questions (FAQ)}

\end{itemize}

A mailing list for WESTPA is available, at which one can ask questions (or see
if a question one has was previously addressed). This is the preferred means
for obtaining help and support. See \sphinxurl{http://groups.google.com/group/westpa-users}
to sign up or search archived messages.

Further, all WESTPA command\sphinxhyphen{}line tools (located in \sphinxcode{\sphinxupquote{westpa/bin}}) provide detailed help when
given the \sphinxhyphen{}h/\textendash{}help option.

Finally, while WESTPA is a powerful tool that enables expert simulators to access much longer
timescales than is practical with standard simulations, there can be a steep learning curve to
figuring out how to effectively run the simulations on your computing resource of choice.
For serious users who have completed the online tutorials and are ready for production simulations
of their system, we invite you to contact Lillian Chong (ltchong AT pitt DOT edu) about spending
a few days with her lab and/or setting up video conferencing sessions to help you get your
simulations off the ground.


\chapter{Copyright, license, and warranty information}
\label{\detokenize{sphinx_index:copyright-license-and-warranty-information}}

\section{For WESTPA}
\label{\detokenize{sphinx_index:for-westpa}}
The WESTPA package is copyright (c) 2013, Matthew C. Zwier and
Lillian T. Chong. (Individual contributions noted in each source file.)

WESTPA is free software: you can redistribute it and/or modify
it under the terms of the GNU General Public License as published by
the Free Software Foundation, either version 3 of the License, or
(at your option) any later version.

WESTPA is distributed in the hope that it will be useful,
but WITHOUT ANY WARRANTY; without even the implied warranty of
MERCHANTABILITY or FITNESS FOR A PARTICULAR PURPOSE.  See the
GNU General Public License for more details.

You should have received a copy of the GNU General Public License
along with this program (see the included file \sphinxcode{\sphinxupquote{COPYING}}).  If not,
see \textless{}\sphinxurl{http://www.gnu.org/licenses/}\textgreater{}.

Unless otherwise noted, source files included in this distribution and
lacking a more specific attribution are subject to the above copyright,
terms, and conditions.


\section{For included software}
\label{\detokenize{sphinx_index:for-included-software}}
Distributions of WESTPA include a number of components without modification,
each of which is subject to its own individual terms and conditions. Please
see each package’s documentation for the most up\sphinxhyphen{}to\sphinxhyphen{}date possible information
on authorship and licensing. Such packages include:
\begin{quote}
\begin{description}
\item[{h5py}] \leavevmode
See lib/h5py/docs/source/licenses.rst

\item[{blessings}] \leavevmode
See lib/blessings/LICENSE

\end{description}
\end{quote}

In addition, the \sphinxcode{\sphinxupquote{wwmgr}} work manager is derived from the
\sphinxcode{\sphinxupquote{concurrent.futures}} module (as included in Python 3.2) by Brian Quinlan and
copyright 2011 the Python Software Foundation. See
\sphinxurl{http://docs.python.org/3/license.html} for more information.


\chapter{Advanced References}
\label{\detokenize{sphinx_index:advanced-references}}

\section{WEST}
\label{\detokenize{users_guide/west:west}}\label{\detokenize{users_guide/west::doc}}

\subsection{Setup}
\label{\detokenize{users_guide/west/setup:setup}}\label{\detokenize{users_guide/west/setup:id1}}\label{\detokenize{users_guide/west/setup::doc}}

\subsubsection{Defining and Calculating Progress Coordinates}
\label{\detokenize{users_guide/west/setup:defining-and-calculating-progress-coordinates}}

\subsubsection{Binning}
\label{\detokenize{users_guide/west/setup:binning}}
The Weighted Ensemble method enhances sampling by partitioning the space
defined by the progress coordinates into non\sphinxhyphen{}overlapping bins. WESTPA provides
a number of pre\sphinxhyphen{}defined types of bins that the user must parameterize within
the system.py file, which are detailed below.

Users are also free to implement their own mappers. A bin mapper must
implement, at least, an \sphinxcode{\sphinxupquote{assign(coords, mask=None, output=None)}} method,
which is responsible for mapping each of the vector of coordinate tuples
\sphinxcode{\sphinxupquote{coords}} to an integer (\sphinxcode{\sphinxupquote{numpy.uint16}}) indicating what bin that coordinate
tuple falls into. The optional \sphinxcode{\sphinxupquote{mask}} (a numpy bool array) specifies that
some coordinates are to be skipped; this is used, for instance, by the
recursive (nested) bin mapper to minimize the number of calculations required
to definitively assign a coordinate tuple to a bin. Similarly, the optional
\sphinxcode{\sphinxupquote{output}} must be an integer (\sphinxcode{\sphinxupquote{uint16}}) array of the same length as
\sphinxcode{\sphinxupquote{coords}}, into which assignments are written. The \sphinxcode{\sphinxupquote{assign()}} function must
return a reference to \sphinxcode{\sphinxupquote{output}}. (This is used to avoid allocating many
temporary output arrays in complex binning scenarios.)

A user\sphinxhyphen{}defined bin mapper must also make an \sphinxcode{\sphinxupquote{nbins}} property available,
containing the total number of bins within the mapper.


\paragraph{RectilinearBinMapper}
\label{\detokenize{users_guide/west/setup:rectilinearbinmapper}}
Creates an N\sphinxhyphen{}dimensional grid of bins. The Rectilinear bin mapper is
initialized by defining a set of bin boundaries:

\begin{sphinxVerbatim}[commandchars=\\\{\}]
\PYG{n+nb+bp}{self}\PYG{o}{.}\PYG{n}{bin\PYGZus{}mapper} \PYG{o}{=} \PYG{n}{RectilinearBinMapper}\PYG{p}{(}\PYG{n}{boundaries}\PYG{p}{)}
\end{sphinxVerbatim}

where \sphinxcode{\sphinxupquote{boundaries}} is a list or other iterable containing the bin boundaries
along each dimension. The bin boundaries must be monotonically increasing along
each dimension. It is important to note that a one\sphinxhyphen{}dimensional bin space must
still be represented as a list of lists as in the following example::

\begin{sphinxVerbatim}[commandchars=\\\{\}]
\PYG{n}{bounds} \PYG{o}{=} \PYG{p}{[}\PYG{o}{\PYGZhy{}}\PYG{n+nb}{float}\PYG{p}{(}\PYG{l+s+s1}{\PYGZsq{}}\PYG{l+s+s1}{inf}\PYG{l+s+s1}{\PYGZsq{}}\PYG{p}{)}\PYG{p}{,} \PYG{l+m+mf}{0.0}\PYG{p}{,} \PYG{l+m+mf}{1.0}\PYG{p}{,} \PYG{l+m+mf}{2.0}\PYG{p}{,} \PYG{l+m+mf}{3.0}\PYG{p}{,} \PYG{n+nb}{float}\PYG{p}{(}\PYG{l+s+s1}{\PYGZsq{}}\PYG{l+s+s1}{inf}\PYG{l+s+s1}{\PYGZsq{}}\PYG{p}{)}\PYG{p}{]}
\PYG{n+nb+bp}{self}\PYG{o}{.}\PYG{n}{bin\PYGZus{}mapper} \PYG{o}{=} \PYG{n}{RectilinearBinMapper}\PYG{p}{(}\PYG{p}{[}\PYG{n}{bounds}\PYG{p}{]}\PYG{p}{)}
\end{sphinxVerbatim}

A two\sphinxhyphen{}dimensional system might look like::

\begin{sphinxVerbatim}[commandchars=\\\{\}]
\PYG{n}{boundaries} \PYG{o}{=} \PYG{p}{[}\PYG{p}{(}\PYG{o}{\PYGZhy{}}\PYG{l+m+mi}{1}\PYG{p}{,}\PYG{o}{\PYGZhy{}}\PYG{l+m+mf}{0.5}\PYG{p}{,}\PYG{l+m+mi}{0}\PYG{p}{,}\PYG{l+m+mf}{0.5}\PYG{p}{,}\PYG{l+m+mi}{1}\PYG{p}{)}\PYG{p}{,} \PYG{p}{(}\PYG{o}{\PYGZhy{}}\PYG{l+m+mi}{1}\PYG{p}{,}\PYG{o}{\PYGZhy{}}\PYG{l+m+mf}{0.5}\PYG{p}{,}\PYG{l+m+mi}{0}\PYG{p}{,}\PYG{l+m+mf}{0.5}\PYG{p}{,}\PYG{l+m+mi}{1}\PYG{p}{)}\PYG{p}{]}
\PYG{n+nb+bp}{self}\PYG{o}{.}\PYG{n}{bin\PYGZus{}mapper} \PYG{o}{=} \PYG{n}{RectilinearBinMapper}\PYG{p}{(}\PYG{n}{boundaries}\PYG{p}{)}
\end{sphinxVerbatim}

where the first tuple in the list defines the boundaries along the first
progress coordinate, and the second tuple defines the boundaries along the
second. Of course a list of arbitrary dimensions can be defined to create an
N\sphinxhyphen{}dimensional grid discretizing the progress coordinate space.


\paragraph{VoronoiBinMapper}
\label{\detokenize{users_guide/west/setup:voronoibinmapper}}
A one\sphinxhyphen{}dimensional mapper which assigns a multidimensional progress coordinate
to the closest center based on a distance metric. The Voronoi bin mapper is
initialized with the following signature within the
\sphinxcode{\sphinxupquote{WESTSystem.initialize}}::

\begin{sphinxVerbatim}[commandchars=\\\{\}]
\PYG{n+nb+bp}{self}\PYG{o}{.}\PYG{n}{bin\PYGZus{}mapper} \PYG{o}{=} \PYG{n}{VoronoiBinMapper}\PYG{p}{(}\PYG{n}{dfunc}\PYG{p}{,} \PYG{n}{centers}\PYG{p}{,} \PYG{n}{dfargs}\PYG{o}{=}\PYG{k+kc}{None}\PYG{p}{,} \PYG{n}{dfkwargs}\PYG{o}{=}\PYG{k+kc}{None}\PYG{p}{)}
\end{sphinxVerbatim}
\begin{itemize}
\item {} 
\sphinxcode{\sphinxupquote{centers}} is a \sphinxcode{\sphinxupquote{(n\_centers, pcoord\_ndim)}} shaped numpy array defining
the generators of the Voronoi cells

\item {} 
\sphinxcode{\sphinxupquote{dfunc}} is a method written in Python that returns an \sphinxcode{\sphinxupquote{(n\_centers, )}}
shaped array containing the distance between a single set of progress
coordinates for a segment and all of the centers defining the Voronoi
tessellation. It takes the general form::

\begin{sphinxVerbatim}[commandchars=\\\{\}]
\PYG{k}{def} \PYG{n+nf}{dfunc}\PYG{p}{(}\PYG{n}{p}\PYG{p}{,} \PYG{n}{centers}\PYG{p}{,} \PYG{o}{*}\PYG{n}{dfargs}\PYG{p}{,} \PYG{o}{*}\PYG{o}{*}\PYG{n}{dfkwargs}\PYG{p}{)}\PYG{p}{:}
    \PYG{o}{.}\PYG{o}{.}\PYG{o}{.}
    \PYG{k}{return} \PYG{n}{d}
\end{sphinxVerbatim}

\end{itemize}

where \sphinxcode{\sphinxupquote{p}} is the progress coordinates of a single segment at one time slice
of shape \sphinxcode{\sphinxupquote{(pcoord\_ndim,)}}, \sphinxcode{\sphinxupquote{centers}} is the full set of centers, \sphinxcode{\sphinxupquote{dfargs}}
is a tuple or list of positional arguments and \sphinxcode{\sphinxupquote{dfwargs}} is a dictionary of
keyword arguments. The bin mapper’s \sphinxcode{\sphinxupquote{assign}} method then assigns the progress
coordinates to the closest bin (minimum distance). It is the responsibility of
the user to ensure that the distance is calculated using the appropriate
metric.
\begin{itemize}
\item {} 
\sphinxcode{\sphinxupquote{dfargs}} is an optional list or tuple of positional arguments to pass into
\sphinxcode{\sphinxupquote{dfunc}}.

\item {} 
\sphinxcode{\sphinxupquote{dfkwargs}} is an optional dict of keyword arguments to pass into \sphinxcode{\sphinxupquote{dfunc}}.

\end{itemize}


\paragraph{FuncBinMapper}
\label{\detokenize{users_guide/west/setup:funcbinmapper}}
A bin mapper that employs a set of user\sphinxhyphen{}defined function, which directly
calculate bin assignments for a number of coordinate values. The function is
responsible for iterating over the entire coordinate set. This is best used
with C/Cython/Numba methods, or intellegently\sphinxhyphen{}tuned numpy\sphinxhyphen{}based Python
functions.

The \sphinxcode{\sphinxupquote{FuncBinMapper}} is initialized as::

\begin{sphinxVerbatim}[commandchars=\\\{\}]
\PYG{n+nb+bp}{self}\PYG{o}{.}\PYG{n}{bin\PYGZus{}mapper} \PYG{o}{=} \PYG{n}{FuncBinMapper}\PYG{p}{(}\PYG{n}{func}\PYG{p}{,} \PYG{n}{nbins}\PYG{p}{,} \PYG{n}{args}\PYG{o}{=}\PYG{k+kc}{None}\PYG{p}{,} \PYG{n}{kwargs}\PYG{o}{=}\PYG{k+kc}{None}\PYG{p}{)}
\end{sphinxVerbatim}

where \sphinxcode{\sphinxupquote{func}} is the user\sphinxhyphen{}defined method to assign coordinates to bins,
\sphinxcode{\sphinxupquote{nbins}} is the number of bins in the partitioning space, and \sphinxcode{\sphinxupquote{args}} and
\sphinxcode{\sphinxupquote{kwargs}} are optional positional and keyword arguments, respectively, that
are passed into \sphinxcode{\sphinxupquote{func}} when it is called.

The user\sphinxhyphen{}defined function should have the following form::

\begin{sphinxVerbatim}[commandchars=\\\{\}]
\PYG{k}{def} \PYG{n+nf}{func}\PYG{p}{(}\PYG{n}{coords}\PYG{p}{,} \PYG{n}{mask}\PYG{p}{,} \PYG{n}{output}\PYG{p}{,} \PYG{o}{*}\PYG{n}{args}\PYG{p}{,} \PYG{o}{*}\PYG{o}{*}\PYG{n}{kwargs}\PYG{p}{)}
    \PYG{o}{.}\PYG{o}{.}\PYG{o}{.}\PYG{o}{.}
\end{sphinxVerbatim}

where the assignments returned in the \sphinxcode{\sphinxupquote{output}} array, which is modified
in\sphinxhyphen{}place.

As a contrived example, the following function would assign all segments to bin
0 if the sum of the first two progress coordinates was less than \sphinxcode{\sphinxupquote{s*0.5}}, and
to bin 1 otherwise, where \sphinxcode{\sphinxupquote{s=1.5}}::

\begin{sphinxVerbatim}[commandchars=\\\{\}]
\PYG{k}{def} \PYG{n+nf}{func}\PYG{p}{(}\PYG{n}{coords}\PYG{p}{,} \PYG{n}{mask}\PYG{p}{,} \PYG{n}{output}\PYG{p}{,} \PYG{n}{s}\PYG{p}{)}\PYG{p}{:}
    \PYG{n}{output}\PYG{p}{[}\PYG{n}{coords}\PYG{p}{[}\PYG{p}{:}\PYG{p}{,}\PYG{l+m+mi}{0}\PYG{p}{]} \PYG{o}{+} \PYG{n}{coords}\PYG{p}{[}\PYG{p}{:}\PYG{p}{,}\PYG{l+m+mi}{1}\PYG{p}{]} \PYG{o}{\PYGZlt{}} \PYG{n}{s}\PYG{o}{*}\PYG{l+m+mf}{0.5}\PYG{p}{]} \PYG{o}{=} \PYG{l+m+mi}{0}
    \PYG{n}{output}\PYG{p}{[}\PYG{n}{coords}\PYG{p}{[}\PYG{p}{:}\PYG{p}{,}\PYG{l+m+mi}{0}\PYG{p}{]} \PYG{o}{+} \PYG{n}{coords}\PYG{p}{[}\PYG{p}{:}\PYG{p}{,}\PYG{l+m+mi}{1}\PYG{p}{]} \PYG{o}{\PYGZgt{}}\PYG{o}{=} \PYG{n}{s}\PYG{o}{*}\PYG{l+m+mf}{0.5}\PYG{p}{]} \PYG{o}{=} \PYG{l+m+mi}{1}

\PYG{o}{.}\PYG{o}{.}\PYG{o}{.}\PYG{o}{.}

\PYG{n+nb+bp}{self}\PYG{o}{.}\PYG{n}{bin\PYGZus{}mapper} \PYG{o}{=} \PYG{n}{FuncBinMapper}\PYG{p}{(}\PYG{n}{func}\PYG{p}{,} \PYG{l+m+mi}{2}\PYG{p}{,} \PYG{n}{args}\PYG{o}{=}\PYG{p}{(}\PYG{l+m+mf}{1.5}\PYG{p}{,}\PYG{p}{)}\PYG{p}{)}
\end{sphinxVerbatim}


\paragraph{VectorizingFuncBinMapper}
\label{\detokenize{users_guide/west/setup:vectorizingfuncbinmapper}}
Like the \sphinxcode{\sphinxupquote{FuncBinMapper}}, the \sphinxcode{\sphinxupquote{VectorizingFuncBinMapper}} uses a
user\sphinxhyphen{}defined method to calculate bin assignments. They differ, however, in that
while the user\sphinxhyphen{}defined method passed to an instance of the \sphinxcode{\sphinxupquote{FuncBinMapper}} is
responsible for iterating over all coordinate sets passed to it, the function
associated with the \sphinxcode{\sphinxupquote{VectorizingFuncBinMapper}} is evaluated once for each
unmasked coordinate tuple provided. It is not responsible explicitly for
iterating over multiple progress coordinate sets.

The \sphinxcode{\sphinxupquote{VectorizingFuncBinMapper}} is initialized as::

\begin{sphinxVerbatim}[commandchars=\\\{\}]
\PYG{n+nb+bp}{self}\PYG{o}{.}\PYG{n}{bin\PYGZus{}mapper} \PYG{o}{=} \PYG{n}{VectorizingFuncBinMapper}\PYG{p}{(}\PYG{n}{func}\PYG{p}{,} \PYG{n}{nbins}\PYG{p}{,} \PYG{n}{args}\PYG{o}{=}\PYG{k+kc}{None}\PYG{p}{,} \PYG{n}{kwargs}\PYG{o}{=}\PYG{k+kc}{None}\PYG{p}{)}
\end{sphinxVerbatim}

where \sphinxcode{\sphinxupquote{func}} is the user\sphinxhyphen{}defined method to assign coordinates to bins,
\sphinxcode{\sphinxupquote{nbins}} is the number of bins in the partitioning space, and \sphinxcode{\sphinxupquote{args}} and
\sphinxcode{\sphinxupquote{kwargs}} are optional positional and keyword arguments, respectively, that
are passed into \sphinxcode{\sphinxupquote{func}} when it is called.

The user\sphinxhyphen{}defined function should have the following form::

\begin{sphinxVerbatim}[commandchars=\\\{\}]
\PYG{k}{def} \PYG{n+nf}{func}\PYG{p}{(}\PYG{n}{coords}\PYG{p}{,} \PYG{o}{*}\PYG{n}{args}\PYG{p}{,} \PYG{o}{*}\PYG{o}{*}\PYG{n}{kwargs}\PYG{p}{)}
    \PYG{o}{.}\PYG{o}{.}\PYG{o}{.}\PYG{o}{.}
\end{sphinxVerbatim}

Mirroring the simple example shown for the \sphinxcode{\sphinxupquote{FuncBinMapper}}, the following
should result in the same result for a given set of coordinates. Here segments
would be assigned to bin 0 if the sum of the first two progress coordinates was
less than \sphinxcode{\sphinxupquote{s*0.5}}, and to bin 1 otherwise, where \sphinxcode{\sphinxupquote{s=1.5}}::

\begin{sphinxVerbatim}[commandchars=\\\{\}]
\PYG{k}{def} \PYG{n+nf}{func}\PYG{p}{(}\PYG{n}{coords}\PYG{p}{,} \PYG{n}{s}\PYG{p}{)}\PYG{p}{:}
    \PYG{k}{if} \PYG{n}{coords}\PYG{p}{[}\PYG{l+m+mi}{0}\PYG{p}{]} \PYG{o}{+} \PYG{n}{coords}\PYG{p}{[}\PYG{l+m+mi}{1}\PYG{p}{]} \PYG{o}{\PYGZlt{}} \PYG{n}{s}\PYG{o}{*}\PYG{l+m+mf}{0.5}\PYG{p}{:}
        \PYG{k}{return} \PYG{l+m+mi}{0}
    \PYG{k}{else}\PYG{p}{:}
        \PYG{k}{return} \PYG{l+m+mi}{1}
\PYG{o}{.}\PYG{o}{.}\PYG{o}{.}\PYG{o}{.}

\PYG{n+nb+bp}{self}\PYG{o}{.}\PYG{n}{bin\PYGZus{}mapper} \PYG{o}{=} \PYG{n}{VectorizingFuncBinMapper}\PYG{p}{(}\PYG{n}{func}\PYG{p}{,} \PYG{l+m+mi}{2}\PYG{p}{,} \PYG{n}{args}\PYG{o}{=}\PYG{p}{(}\PYG{l+m+mf}{1.5}\PYG{p}{,}\PYG{p}{)}\PYG{p}{)}
\end{sphinxVerbatim}


\paragraph{PiecewiseBinMapper}
\label{\detokenize{users_guide/west/setup:piecewisebinmapper}}

\paragraph{RecursiveBinMapper}
\label{\detokenize{users_guide/west/setup:recursivebinmapper}}
The \sphinxcode{\sphinxupquote{RecursiveBinMapper}} is used for assembling more complex bin spaces from
simpler components and nesting one set of bins within another. It is
initialized as::

\begin{sphinxVerbatim}[commandchars=\\\{\}]
\PYG{n+nb+bp}{self}\PYG{o}{.}\PYG{n}{bin\PYGZus{}mapper} \PYG{o}{=} \PYG{n}{RecursiveBinMapper}\PYG{p}{(}\PYG{n}{base\PYGZus{}mapper}\PYG{p}{,} \PYG{n}{start\PYGZus{}index}\PYG{o}{=}\PYG{l+m+mi}{0}\PYG{p}{)}
\end{sphinxVerbatim}

The \sphinxcode{\sphinxupquote{base\_mapper}} is an instance of one of the other bin mappers, and
\sphinxcode{\sphinxupquote{start\_index}} is an (optional) offset for indexing the bins. Starting with
the \sphinxcode{\sphinxupquote{base\_mapper}}, additional bins can be nested into it using the
\sphinxcode{\sphinxupquote{add\_mapper(mapper, replaces\_bin\_at)}}. This method will replace the bin
containing the coordinate tuple \sphinxcode{\sphinxupquote{replaces\_bin\_at}} with the mapper specified
by \sphinxcode{\sphinxupquote{mapper}}.

As a simple example consider a bin space in which the \sphinxcode{\sphinxupquote{base\_mapper}} assigns a
segment with progress coordinate with values \textless{}1 into one bin and \textgreater{}= 1 into
another. Within the former bin, we will nest a second mapper which partitions
progress coordinate space into one bin for progress coordinate values \textless{}0.5 and
another for progress coordinates with values \textgreater{}=0.5. The bin space would look
like the following with corresponding code::

\begin{sphinxVerbatim}[commandchars=\\\{\}]
\PYG{l+s+sd}{\PYGZsq{}\PYGZsq{}\PYGZsq{}}
\PYG{l+s+sd}{             0                            1                      2}
\PYG{l+s+sd}{             +\PYGZhy{}\PYGZhy{}\PYGZhy{}\PYGZhy{}\PYGZhy{}\PYGZhy{}\PYGZhy{}\PYGZhy{}\PYGZhy{}\PYGZhy{}\PYGZhy{}\PYGZhy{}\PYGZhy{}\PYGZhy{}\PYGZhy{}\PYGZhy{}\PYGZhy{}\PYGZhy{}\PYGZhy{}\PYGZhy{}\PYGZhy{}\PYGZhy{}\PYGZhy{}\PYGZhy{}\PYGZhy{}\PYGZhy{}\PYGZhy{}\PYGZhy{}+\PYGZhy{}\PYGZhy{}\PYGZhy{}\PYGZhy{}\PYGZhy{}\PYGZhy{}\PYGZhy{}\PYGZhy{}\PYGZhy{}\PYGZhy{}\PYGZhy{}\PYGZhy{}\PYGZhy{}\PYGZhy{}\PYGZhy{}\PYGZhy{}\PYGZhy{}\PYGZhy{}\PYGZhy{}\PYGZhy{}\PYGZhy{}\PYGZhy{}+}
\PYG{l+s+sd}{             |            0.5             |                      |}
\PYG{l+s+sd}{             | +\PYGZhy{}\PYGZhy{}\PYGZhy{}\PYGZhy{}\PYGZhy{}\PYGZhy{}\PYGZhy{}\PYGZhy{}\PYGZhy{}\PYGZhy{}\PYGZhy{}+\PYGZhy{}\PYGZhy{}\PYGZhy{}\PYGZhy{}\PYGZhy{}\PYGZhy{}\PYGZhy{}\PYGZhy{}\PYGZhy{}\PYGZhy{}\PYGZhy{}\PYGZhy{}+ |                      |}
\PYG{l+s+sd}{             | |           |            | |                      |}
\PYG{l+s+sd}{             | |     1     |     2      | |          0           |}
\PYG{l+s+sd}{             | |           |            | |                      |}
\PYG{l+s+sd}{             | |           |            | |                      |}
\PYG{l+s+sd}{             | +\PYGZhy{}\PYGZhy{}\PYGZhy{}\PYGZhy{}\PYGZhy{}\PYGZhy{}\PYGZhy{}\PYGZhy{}\PYGZhy{}\PYGZhy{}\PYGZhy{}+\PYGZhy{}\PYGZhy{}\PYGZhy{}\PYGZhy{}\PYGZhy{}\PYGZhy{}\PYGZhy{}\PYGZhy{}\PYGZhy{}\PYGZhy{}\PYGZhy{}\PYGZhy{}+ |                      |prettyprint}
\PYG{l+s+sd}{             +\PYGZhy{}\PYGZhy{}\PYGZhy{}\PYGZhy{}\PYGZhy{}\PYGZhy{}\PYGZhy{}\PYGZhy{}\PYGZhy{}\PYGZhy{}\PYGZhy{}\PYGZhy{}\PYGZhy{}\PYGZhy{}\PYGZhy{}\PYGZhy{}\PYGZhy{}\PYGZhy{}\PYGZhy{}\PYGZhy{}\PYGZhy{}\PYGZhy{}\PYGZhy{}\PYGZhy{}\PYGZhy{}\PYGZhy{}\PYGZhy{}\PYGZhy{}\PYGZhy{}\PYGZhy{}\PYGZhy{}\PYGZhy{}\PYGZhy{}\PYGZhy{}\PYGZhy{}\PYGZhy{}\PYGZhy{}\PYGZhy{}\PYGZhy{}\PYGZhy{}\PYGZhy{}\PYGZhy{}\PYGZhy{}\PYGZhy{}\PYGZhy{}\PYGZhy{}\PYGZhy{}\PYGZhy{}\PYGZhy{}\PYGZhy{}\PYGZhy{}+}
\PYG{l+s+sd}{\PYGZsq{}\PYGZsq{}\PYGZsq{}}

\PYG{k}{def} \PYG{n+nf}{fn1}\PYG{p}{(}\PYG{n}{coords}\PYG{p}{,} \PYG{n}{mask}\PYG{p}{,} \PYG{n}{output}\PYG{p}{)}\PYG{p}{:}
    \PYG{n}{test} \PYG{o}{=} \PYG{n}{coords}\PYG{p}{[}\PYG{p}{:}\PYG{p}{,}\PYG{l+m+mi}{0}\PYG{p}{]} \PYG{o}{\PYGZlt{}} \PYG{l+m+mi}{1}
    \PYG{n}{output}\PYG{p}{[}\PYG{n}{mask} \PYG{o}{\PYGZam{}} \PYG{n}{test}\PYG{p}{]} \PYG{o}{=} \PYG{l+m+mi}{0}
    \PYG{n}{output}\PYG{p}{[}\PYG{n}{mask} \PYG{o}{\PYGZam{}} \PYG{o}{\PYGZti{}}\PYG{n}{test}\PYG{p}{]} \PYG{o}{=} \PYG{l+m+mi}{1}

\PYG{k}{def} \PYG{n+nf}{fn2}\PYG{p}{(}\PYG{n}{coords}\PYG{p}{,} \PYG{n}{mask}\PYG{p}{,} \PYG{n}{output}\PYG{p}{)}\PYG{p}{:}
    \PYG{n}{test} \PYG{o}{=} \PYG{n}{coords}\PYG{p}{[}\PYG{p}{:}\PYG{p}{,}\PYG{l+m+mi}{0}\PYG{p}{]} \PYG{o}{\PYGZlt{}} \PYG{l+m+mf}{0.5}
    \PYG{n}{output}\PYG{p}{[}\PYG{n}{mask} \PYG{o}{\PYGZam{}} \PYG{n}{test}\PYG{p}{]} \PYG{o}{=} \PYG{l+m+mi}{0}
    \PYG{n}{output}\PYG{p}{[}\PYG{n}{mask} \PYG{o}{\PYGZam{}} \PYG{o}{\PYGZti{}}\PYG{n}{test}\PYG{p}{]} \PYG{o}{=} \PYG{l+m+mi}{1}

\PYG{n}{outer\PYGZus{}mapper} \PYG{o}{=} \PYG{n}{FuncBinMapper}\PYG{p}{(}\PYG{n}{fn1}\PYG{p}{,}\PYG{l+m+mi}{2}\PYG{p}{)}
\PYG{n}{inner\PYGZus{}mapper} \PYG{o}{=} \PYG{n}{FuncBinMapper}\PYG{p}{(}\PYG{n}{fn2}\PYG{p}{,}\PYG{l+m+mi}{2}\PYG{p}{)}
\PYG{n}{rmapper} \PYG{o}{=} \PYG{n}{RecursiveBinMapper}\PYG{p}{(}\PYG{n}{outer\PYGZus{}mapper}\PYG{p}{)}
\PYG{n}{rmapper}\PYG{o}{.}\PYG{n}{add\PYGZus{}mapper}\PYG{p}{(}\PYG{n}{inner\PYGZus{}mapper}\PYG{p}{,} \PYG{p}{[}\PYG{l+m+mf}{0.5}\PYG{p}{]}\PYG{p}{)}
\end{sphinxVerbatim}

Examples of more complicated nesting schemes can be found in the \sphinxhref{https://github.com/westpa/westpa/blob/master/lib/west\_tools/tests/testbinning.py}{tests}
for the WESTPA binning apparatus.


\subsubsection{Initial/Basis States}
\label{\detokenize{users_guide/west/setup:initial-basis-states}}
A WESTPA simulation is initialized using \sphinxcode{\sphinxupquote{w\_init}} with an initial
distribution of replicas generated from a set of basis states. These basis
states are used to generate initial states for new trajectories, either at the
beginning of the simulation or due to recycling. Basis states are specified
when running \sphinxcode{\sphinxupquote{w\_init}} either in a file specified with \sphinxcode{\sphinxupquote{\sphinxhyphen{}\sphinxhyphen{}bstates\sphinxhyphen{}from}}, or
by one or more \sphinxcode{\sphinxupquote{\sphinxhyphen{}\sphinxhyphen{}bstate}} arguments. If neither \sphinxcode{\sphinxupquote{\sphinxhyphen{}\sphinxhyphen{}bstates\sphinxhyphen{}from}} nor at
least one \sphinxcode{\sphinxupquote{\sphinxhyphen{}\sphinxhyphen{}bstate}} argument is provided, then a default basis state of
probability one identified by the state ID zero and label “basis” will be
created (a warning will be printed in this case, to remind you of this
behavior, in case it is not what you wanted).

When using a file passed to \sphinxcode{\sphinxupquote{w\_init}} using \sphinxcode{\sphinxupquote{\sphinxhyphen{}\sphinxhyphen{}bstates\sphinxhyphen{}from}}, each line in
that file defines a state, and contains a label, the probability, and
optionally a data reference, separated by whitespace, as in::

\begin{sphinxVerbatim}[commandchars=\\\{\}]
\PYG{n}{unbound}    \PYG{l+m+mf}{1.0}
\end{sphinxVerbatim}

or:

\begin{sphinxVerbatim}[commandchars=\\\{\}]
\PYG{n}{unbound\PYGZus{}0}    \PYG{l+m+mf}{0.6}        \PYG{n}{state0}\PYG{o}{.}\PYG{n}{pdb}
\PYG{n}{unbound\PYGZus{}1}    \PYG{l+m+mf}{0.4}        \PYG{n}{state1}\PYG{o}{.}\PYG{n}{pdb}
\end{sphinxVerbatim}

Basis states can also be supplied at the command line using one or more
\sphinxcode{\sphinxupquote{\sphinxhyphen{}\sphinxhyphen{}bstate}} flags, where the argument matches the format used in the state
file above. The total probability summed over all basis states should equal
unity, however WESTPA will renormalize the distribution if this condition is
not met.

Initial states are the generated from the basis states by optionally applying
some perturbation or modification to the basis state. For example if WESTPA was
being used to simulate ligand binding, one might want to have a basis state
where the ligand was some set distance from the binding partner, and initial
states are generated by randomly orienting the ligand at that distance. When
using the executable propagator, this is done using the script specified under
the \sphinxcode{\sphinxupquote{gen\_istate}} section of the \sphinxcode{\sphinxupquote{executable}} configuration. Otherwise, if
defining a custom propagator, the user must override the \sphinxcode{\sphinxupquote{gen\_istate}} method
of \sphinxcode{\sphinxupquote{WESTPropagator}}.

When using the executable propagator, the the script specified by
\sphinxcode{\sphinxupquote{gen\_istate}} should take the data supplied by the environmental variable
\sphinxcode{\sphinxupquote{\$WEST\_BSTATE\_DATA\_REF}} and return the generated initial state to
\sphinxcode{\sphinxupquote{\$WEST\_ISTATE\_DATA\_REF}}. If no transform need be performed, the user may
simply copy the data directly without modification. This data will then be
available via \sphinxcode{\sphinxupquote{\$WEST\_PARENT\_DATA\_REF}} if \sphinxcode{\sphinxupquote{\$WEST\_CURRENT\_SEG\_INITPOINT\_TYPE}}
is \sphinxcode{\sphinxupquote{SEG\_INITPOINT\_NEWTRAJ}}.


\subsubsection{Target States}
\label{\detokenize{users_guide/west/setup:target-states}}
WESTPA can be run in a recycling mode in which replicas reaching a target state
are removed from the simulation and their weights are assigned to new replicas
created from one of the initial states. This mode creates a non\sphinxhyphen{}equilibrium
steady\sphinxhyphen{}state that isolates members of the trajectory ensemble originating in
the set of initial states and transitioning to the target states. The flux of
probability into the target state is then inversely proportional to the mean
first passage time (MFPT) of the transition.

Target states are defined when initializing a WESTPA simulation when calling
\sphinxcode{\sphinxupquote{w\_init}}. Target states are specified either in a file specified with
\sphinxcode{\sphinxupquote{\sphinxhyphen{}\sphinxhyphen{}tstates\sphinxhyphen{}from}}, or by one or more \sphinxcode{\sphinxupquote{\sphinxhyphen{}\sphinxhyphen{}tstate}} arguments. If neither
\sphinxcode{\sphinxupquote{\sphinxhyphen{}\sphinxhyphen{}tstates\sphinxhyphen{}from}} nor at least one \sphinxcode{\sphinxupquote{\sphinxhyphen{}\sphinxhyphen{}tstate}} argument is provided, then an
equilibrium simulation (without any sinks) will be performed.

Target states can be defined using a text file, where each line defines a
state, and contains a label followed by a representative progress coordinate
value, separated by whitespace, as in::

\begin{sphinxVerbatim}[commandchars=\\\{\}]
\PYG{n}{bound}     \PYG{l+m+mf}{0.02}
\end{sphinxVerbatim}

for a single target and one\sphinxhyphen{}dimensional progress coordinates or::

\begin{sphinxVerbatim}[commandchars=\\\{\}]
\PYG{n}{bound}    \PYG{l+m+mf}{2.7}    \PYG{l+m+mf}{0.0}
\PYG{n}{drift}    \PYG{l+m+mi}{100}    \PYG{l+m+mf}{50.0}
\end{sphinxVerbatim}

for two targets and a two\sphinxhyphen{}dimensional progress coordinate.

The argument associated with \sphinxcode{\sphinxupquote{\sphinxhyphen{}\sphinxhyphen{}tstate}} is a string of the form \sphinxcode{\sphinxupquote{\textquotesingle{}label,
pcoord0 {[},pcoord1{[},...{]}{]}\textquotesingle{}}}, similar to a line in the example target state
definition file above. This argument may be specified more than once, in which
case the given states are appended to the list of target states for the
simulation in the order they appear on the command line, after those that are
specified by \sphinxcode{\sphinxupquote{\sphinxhyphen{}\sphinxhyphen{}tstates\sphinxhyphen{}from}}, if any.

WESTPA uses the representative progress coordinate of a target\sphinxhyphen{}state and
converts the \sphinxstylestrong{entire} bin containing that progress coordinate into a
recycling sink.


\subsubsection{Propagators}
\label{\detokenize{users_guide/west/setup:propagators}}

\paragraph{The Executable Propagator}
\label{\detokenize{users_guide/west/setup:the-executable-propagator}}

\paragraph{Writing custom propagators}
\label{\detokenize{users_guide/west/setup:writing-custom-propagators}}
While most users will use the Executable propagator to run dynamics by calling
out to an external piece of software, it is possible to write custom
propagators that can be used to generate sampling directly through the python
interface. This is particularly useful when simulating simple systems, where
the overhead of starting up an external program is large compared to the actual
cost of computing the trajectory segment. Other use cases might include running
sampling with software that has a Python API (e.g. \sphinxhref{https://simtk.org/home/openmm}{OpenMM}).

In order to create a custom propagator, users must define a class that inherits
from \sphinxcode{\sphinxupquote{WESTPropagator}} and implement three methods:
\begin{itemize}
\item {} 
\sphinxcode{\sphinxupquote{get\_pcoord(self, state)}}: Get the progress coordinate of the given basis
or initial state.

\item {} 
\sphinxcode{\sphinxupquote{gen\_istate(self, basis\_state, initial\_state)}}: Generate a new initial
state from the given basis state. This method is optional if \sphinxcode{\sphinxupquote{gen\_istates}}
is set to \sphinxcode{\sphinxupquote{False}} in the propagation section of the configuration file,
which is the default setting.

\item {} 
\sphinxcode{\sphinxupquote{propagate(self, segments)}}: Propagate one or more segments, including any
necessary per\sphinxhyphen{}iteration setup and teardown for this propagator.

\end{itemize}

There are also two stubs that that, if overridden, provide a mechanism for
modifying the simulation before or after the iteration:
\begin{itemize}
\item {} 
\sphinxcode{\sphinxupquote{prepare\_iteration(self, n\_iter, segments)}}: Perform any necessary
per\sphinxhyphen{}iteration preparation. This is run by the work manager.

\item {} 
\sphinxcode{\sphinxupquote{finalize\_iteration(self, n\_iter, segments)}}: Perform any necessary
post\sphinxhyphen{}iteration cleanup. This is run by the work manager.

\end{itemize}

Several examples of custom propagators are available:
\begin{itemize}
\item {} 
\sphinxhref{https://github.com/westpa/westpa/blob/master/lib/examples/odld/odld\_system.py}{1D Over\sphinxhyphen{}damped Langevin dynamics}

\item {} 
\sphinxhref{https://bitbucket.org/joshua.adelman/stringmethodexamples/src/tip/examples/DicksonRingPotential/we\_base/system.py}{2D Langevin dynamics}

\item {} 
\sphinxhref{https://bitbucket.org/joshua.adelman/stringmethodexamples/src/tip/examples/ElasticNetworkModel/we\_base/system.py}{Langevin dynamics \sphinxhyphen{} CA atom Elastic Network Model}

\end{itemize}


\subsubsection{Configuration File}
\label{\detokenize{users_guide/west/setup:configuration-file}}
The configuration of a WESTPA simulation is specified using a plain text file
written in \sphinxhref{http://en.wikipedia.org/wiki/YAML}{YAML}. This file specifies,
among many other things, the length of the simulation, which modules should be
loaded for specifying the system, how external data should be organized on the
file system, and which plugins should used. YAML is a hierarchical format and
WESTPA organizes the configuration settings into blocks for each component.
While below, the configuration file will be referred to as \sphinxstylestrong{west.cfg}, the
user is free to name the configuration file something else. Most of the scripts
and tools that WESTPA provides, however, require that the name of the
configuration file be specified if the default name is not used.

The top most heading in \sphinxstyleemphasis{west.cfg} should be specified as::

\begin{sphinxVerbatim}[commandchars=\\\{\}]
\PYG{o}{\PYGZhy{}}\PYG{o}{\PYGZhy{}}\PYG{o}{\PYGZhy{}}
\PYG{n}{west}\PYG{p}{:}
    \PYG{o}{.}\PYG{o}{.}\PYG{o}{.}
\end{sphinxVerbatim}

with all sub\sphinxhyphen{}section specified below it. A complete example can be found for
the NaCl example:
\sphinxurl{https://github.com/westpa/westpa/blob/master/lib/examples/nacl\_gmx/west.cfg}

In the following section, the specifications for each section of the file can
be found, along with default parameters and descriptions. Required parameters
are indicated as REQUIRED.:

\begin{sphinxVerbatim}[commandchars=\\\{\}]
\PYG{o}{\PYGZhy{}}\PYG{o}{\PYGZhy{}}\PYG{o}{\PYGZhy{}}
\PYG{n}{west}\PYG{p}{:}
    \PYG{o}{.}\PYG{o}{.}\PYG{o}{.}
    \PYG{n}{system}\PYG{p}{:}
        \PYG{n}{driver}\PYG{p}{:} \PYG{n}{REQUIRED}
        \PYG{n}{module\PYGZus{}path}\PYG{p}{:} \PYG{p}{[}\PYG{p}{]}
\end{sphinxVerbatim}

The \sphinxcode{\sphinxupquote{driver}} parameter must be set to a subclass of \sphinxcode{\sphinxupquote{WESTSystem}}, and given
in the form \sphinxstyleemphasis{module.class}. The \sphinxcode{\sphinxupquote{module\_path}} parameter is appended to the
system path and indicates where the class is defined.:

\begin{sphinxVerbatim}[commandchars=\\\{\}]
\PYG{o}{\PYGZhy{}}\PYG{o}{\PYGZhy{}}\PYG{o}{\PYGZhy{}}
\PYG{n}{west}\PYG{p}{:}
    \PYG{o}{.}\PYG{o}{.}\PYG{o}{.}
    \PYG{n}{we}\PYG{p}{:}
        \PYG{n}{adjust\PYGZus{}counts}\PYG{p}{:} \PYG{k+kc}{True}
        \PYG{n}{weight\PYGZus{}split\PYGZus{}threshold}\PYG{p}{:} \PYG{l+m+mf}{2.0}
        \PYG{n}{weight\PYGZus{}merge\PYGZus{}cutoff}\PYG{p}{:} \PYG{l+m+mf}{1.0}
\end{sphinxVerbatim}

The \sphinxcode{\sphinxupquote{we}} section section specifies parameters related to the Huber and Kim
resampling algorithm. WESTPA implements a variation of the method, in which
setting \sphinxcode{\sphinxupquote{adust\_counts}} to \sphinxcode{\sphinxupquote{True}} strictly enforces that the number of
replicas per bin is exactly \sphinxcode{\sphinxupquote{system.bin\_target\_counts}}. Otherwise, the number
of replicas per is allowed to fluctuate as in the original implementation of
the algorithm. Adjusting the counts can improve load balancing for parallel
simulations. Replicas with weights greater than \sphinxcode{\sphinxupquote{weight\_split\_threshold}}
times the ideal weight per bin are tagged as candidates for splitting. Replicas
with weights less than \sphinxcode{\sphinxupquote{weight\_merge\_cutoff}} times the ideal weight per bin
are candidates for merging.:

\begin{sphinxVerbatim}[commandchars=\\\{\}]
\PYG{o}{\PYGZhy{}}\PYG{o}{\PYGZhy{}}\PYG{o}{\PYGZhy{}}
\PYG{n}{west}\PYG{p}{:}
    \PYG{o}{.}\PYG{o}{.}\PYG{o}{.}
    \PYG{n}{propagation}\PYG{p}{:}
        \PYG{n}{gen\PYGZus{}istates}\PYG{p}{:} \PYG{k+kc}{False}
        \PYG{n}{block\PYGZus{}size}\PYG{p}{:} \PYG{l+m+mi}{1}
        \PYG{n}{save\PYGZus{}transition\PYGZus{}matrices}\PYG{p}{:} \PYG{k+kc}{False}
        \PYG{n}{max\PYGZus{}run\PYGZus{}wallclock}\PYG{p}{:} \PYG{k+kc}{None}
        \PYG{n}{max\PYGZus{}total\PYGZus{}iterations}\PYG{p}{:} \PYG{k+kc}{None}
\end{sphinxVerbatim}
\begin{itemize}
\item {} 
\sphinxcode{\sphinxupquote{gen\_istates}}: Boolean specifying whether to generate initial states from
the basis states. The executable propagator defines a specific configuration
block (\sphinxstyleemphasis{add internal link to other section}), and custom propagators should
override the \sphinxcode{\sphinxupquote{WESTPropagator.gen\_istate()}} method.

\item {} 
\sphinxcode{\sphinxupquote{block\_size}}: An integer defining how many segments should be passed to a
worker at a time. When using the serial work manager, this value should be
set to the maximum number of segments per iteration to avoid significant
overhead incurred by the locking mechanism in the WMFutures framework.
Parallel work managers might benefit from setting this value greater than one
in some instances to decrease network communication load.

\item {} 
\sphinxcode{\sphinxupquote{save\_transition\_matrices}}:

\item {} 
\sphinxcode{\sphinxupquote{max\_run\_wallclock}}: A time in dd:hh:mm:ss or hh:mm:ss specifying the
maximum wallclock time of a particular WESTPA run. If running on a batch
queuing system, this time should be set to less than the job allocation time
to ensure that WESTPA shuts down cleanly.

\item {} 
\sphinxcode{\sphinxupquote{max\_total\_iterations}}: An integer value specifying the number of
iterations to run. This parameter is checked against the last completed
iteration stored in the HDF5 file, not the number of iterations completed for
a specific run. The default value of \sphinxcode{\sphinxupquote{None}} only stops upon external
termination of the code.:

\begin{sphinxVerbatim}[commandchars=\\\{\}]
\PYG{o}{\PYGZhy{}}\PYG{o}{\PYGZhy{}}\PYG{o}{\PYGZhy{}}
\PYG{n}{west}\PYG{p}{:}
    \PYG{o}{.}\PYG{o}{.}\PYG{o}{.}
    \PYG{n}{data}\PYG{p}{:}
        \PYG{n}{west\PYGZus{}data\PYGZus{}file}\PYG{p}{:} \PYG{n}{REQUIRED}
        \PYG{n}{aux\PYGZus{}compression\PYGZus{}threshold}\PYG{p}{:} \PYG{l+m+mi}{1048576}
        \PYG{n}{iter\PYGZus{}prec}\PYG{p}{:} \PYG{l+m+mi}{8}
        \PYG{n}{datasets}\PYG{p}{:}
            \PYG{o}{\PYGZhy{}}\PYG{n}{name}\PYG{p}{:} \PYG{n}{REQUIRED}
             \PYG{n}{h5path}\PYG{p}{:}
             \PYG{n}{store}\PYG{p}{:} \PYG{k+kc}{True}
             \PYG{n}{load}\PYG{p}{:} \PYG{k+kc}{False}
             \PYG{n}{dtype}\PYG{p}{:}
             \PYG{n}{scaleoffset}\PYG{p}{:} \PYG{k+kc}{None}
             \PYG{n}{compression}\PYG{p}{:} \PYG{k+kc}{None}
             \PYG{n}{chunks}\PYG{p}{:} \PYG{k+kc}{None}
        \PYG{n}{data\PYGZus{}refs}\PYG{p}{:}
            \PYG{n}{segment}\PYG{p}{:}
            \PYG{n}{basis\PYGZus{}state}\PYG{p}{:}
            \PYG{n}{initial\PYGZus{}state}\PYG{p}{:}
\end{sphinxVerbatim}

\item {} 
\sphinxcode{\sphinxupquote{west\_data\_file}}: The name of the main HDF5 data storage file for the
WESTPA simulation.

\item {} 
\sphinxcode{\sphinxupquote{aux\_compression\_threshold}}: The threshold in bytes for compressing the
auxiliary data in a dataset on an iteration\sphinxhyphen{}by\sphinxhyphen{}iteration basis.

\item {} 
\sphinxcode{\sphinxupquote{iter\_prec}}: The length of the iteration index with zero\sphinxhyphen{}padding. For the
default value, iteration 1 would be specified as iter\_00000001.

\item {} 
\sphinxcode{\sphinxupquote{datasets}}:

\item {} 
\sphinxcode{\sphinxupquote{data\_refs}}:

\item {} 
plugins

\item {} 
executable

\end{itemize}


\subsubsection{Environmental Variables}
\label{\detokenize{users_guide/west/setup:environmental-variables}}
There are a number of environmental variables that can be set by the user in
order to configure a WESTPA simulation:
\begin{itemize}
\item {} 
WEST\_ROOT: path to the base directory containing the WESTPA install

\item {} 
WEST\_SIM\_ROOT: path to the base directory of the WESTPA simulation

\item {} 
WEST\_PYTHON: path to python executable to run the WESTPA simulation

\item {} 
WEST\_PYTHONPATH: path to any additional modules that WESTPA will require to
run the simulation

\item {} 
WEST\_KERNPROF: path to \sphinxcode{\sphinxupquote{kernprof.py}} script to perform line\sphinxhyphen{}by\sphinxhyphen{}line
profiling of a WESTPA simulation (see \sphinxhref{http://pythonhosted.org/line\_profiler}{python line\_profiler}). This is only required for users
who need to profile specific methods in a running WESTPA simulation.

\end{itemize}

Work manager related environmental variables:
\begin{itemize}
\item {} 
WM\_WORK\_MANAGER

\item {} 
WM\_N\_WORKERS

\end{itemize}

WESTPA makes available to any script executed by it (e.g. \sphinxstylestrong{runseg.sh}), a
number of environmental variables that are set dynamically by the executable
propagator from the running simulation.


\paragraph{Programs executed for an iteration}
\label{\detokenize{users_guide/west/setup:programs-executed-for-an-iteration}}
The following environment variables are passed to programs executed on a
per\sphinxhyphen{}iteration basis, notably pre\sphinxhyphen{}iteration and post\sphinxhyphen{}iteration scripts.


\begin{savenotes}\sphinxattablestart
\centering
\begin{tabulary}{\linewidth}[t]{|T|T|T|}
\hline
\sphinxstyletheadfamily 
Variable
&\sphinxstyletheadfamily 
Possible values
&\sphinxstyletheadfamily 
Function
\\
\hline
WEST\_CURRENT\_ITER
&
Integer \textgreater{}=1
&
Current iteration number
\\
\hline
\end{tabulary}
\par
\sphinxattableend\end{savenotes}


\paragraph{Programs executed for a segment}
\label{\detokenize{users_guide/west/setup:programs-executed-for-a-segment}}
The following environment variables are passed to programs executed on a
per\sphinxhyphen{}segment basis, notably dynamics propagation.


\begin{savenotes}\sphinxattablestart
\centering
\begin{tabulary}{\linewidth}[t]{|T|T|T|}
\hline
\sphinxstyletheadfamily 
Variable
&\sphinxstyletheadfamily 
Possible values
&\sphinxstyletheadfamily 
Function
\\
\hline
WEST\_CURRENT\_ITER
&
Integer \textgreater{}=1
&
Current iteration
number
\\
\hline
WEST\_CURRENT\_SEG\_ID
&
Integer \textgreater{}=0
&
Current segment ID
\\
\hline
WEST\_CURRENT\_SEG\_DATA\_REF
&
String
&
General\sphinxhyphen{}purpose
reference, based on
current segment
information,
configured in
west.cfg. Usually
used for storage
paths
\\
\hline
WEST\_CURRENT\_SEG\_INITPOINT\_TYPE
&
Enumeration:
SEG\_INITPOINT\_CONTINUES,
SEG\_INITPOINT\_NEWTRAJ
&
Whether this
segment continues a
previous trajectory
or initiates a new
one.
\\
\hline
WEST\_PARENT\_ID
&
Integer
&
Segment ID of
parent segment.
Negative for
initial points.
\\
\hline
WEST\_PARENT\_DATA\_REF
&
String
&
General purpose
reference, based on
parent segment
information,
configured in
west.cfg. Usually
used for storage
paths
\\
\hline
WEST\_PCOORD\_RETURN
&
Filename
&
Where progress
coordinate data
must be stored
\\
\hline
WEST\_RAND16
&
Integer
&
16\sphinxhyphen{}bit random
integer
\\
\hline
WEST\_RAND32
&
Integer
&
32\sphinxhyphen{}bit random
integer
\\
\hline
WEST\_RAND64
&
Integer
&
64\sphinxhyphen{}bit random
integer
\\
\hline
WEST\_RAND128
&
Integer
&
128\sphinxhyphen{}bit random
integer
\\
\hline
WEST\_RANDFLOAT
&
Floating\sphinxhyphen{}point
&
Random number in
{[}0,1).
\\
\hline
\end{tabulary}
\par
\sphinxattableend\end{savenotes}

Additionally for any additional datasets specified in the configuration file,
WESTPA automatically provides \sphinxcode{\sphinxupquote{WEST\_X\_RETURN}}, where \sphinxcode{\sphinxupquote{X}} is the uppercase
name of the dataset. For example if the configuration file contains the
following:

\begin{sphinxVerbatim}[commandchars=\\\{\}]
\PYG{n}{data}\PYG{p}{:}
    \PYG{o}{.}\PYG{o}{.}\PYG{o}{.}
    \PYG{n}{datasets}\PYG{p}{:} \PYG{c+c1}{\PYGZsh{} dataset storage options}
      \PYG{o}{\PYGZhy{}} \PYG{n}{name}\PYG{p}{:} \PYG{n}{energy}
\end{sphinxVerbatim}

WESTPA would make \sphinxcode{\sphinxupquote{WEST\_ENERGY\_RETURN}} available.


\paragraph{Programs executed for a single point}
\label{\detokenize{users_guide/west/setup:programs-executed-for-a-single-point}}
Programs used for creating initial states from basis states (\sphinxcode{\sphinxupquote{gen\_istate.sh}})
or extracting progress coordinates from structures (e.g. \sphinxcode{\sphinxupquote{get\_pcoord.sh}}) are
provided the following environment variables:


\begin{savenotes}\sphinxattablestart
\centering
\begin{tabulary}{\linewidth}[t]{|T|T|T|T|}
\hline
\sphinxstyletheadfamily 
Variable
&\sphinxstyletheadfamily 
Available for
&\sphinxstyletheadfamily 
Possible values
&\sphinxstyletheadfamily 
Function
\\
\hline
WEST\_STRUCT\_DATA\_REF
&
All
single\sphinxhyphen{}point
calculations
&
String
&
General\sphinxhyphen{}purpose
reference, usually a
pathname, associated
with the basis/initial
state.
\\
\hline
WEST\_BSTATE\_ID
&
get\_pcoord for
basis state,
gen\_istate
&
Integer \textgreater{}= 0
&
Basis state ID
\\
\hline
WEST\_BSTATE\_DATA\_REF
&
get\_pcoord for
basis state,
gen\_istate
&
String
&
Basis state data
reference
\\
\hline
WEST\_ISTATE\_ID
&
get\_pcoord for
initial state,
gen\_istate
&
Integer \textgreater{}= 0
&
Inital state ID
\\
\hline
WEST\_ISTATE\_DATA\_REF
&
get\_pcoord for
initial state,
gen\_istate
&
String
&
Initial state data
references, usually a
pathname
\\
\hline
WEST\_PCOORD\_RETURN
&
get\_pcoord for
basis or
initial state
&
Pathname
&
Where progress
coordinate data is
expected to be found
after execution
\\
\hline
\end{tabulary}
\par
\sphinxattableend\end{savenotes}


\subsubsection{Plugins}
\label{\detokenize{users_guide/west/setup:plugins}}
WESTPA has a extensible plugin architecture that allows the user to manipulate
the simulation at specified points during an iteration.
\begin{itemize}
\item {} 
Activating plugins in the config file

\item {} 
Plugin execution order/priority

\end{itemize}


\subsubsection{Weighted Ensemble Algorithm (Resampling)}
\label{\detokenize{users_guide/west/setup:weighted-ensemble-algorithm-resampling}}

\subsection{Running}
\label{\detokenize{users_guide/west/running:running}}\label{\detokenize{users_guide/west/running:id1}}\label{\detokenize{users_guide/west/running::doc}}

\subsubsection{Overview}
\label{\detokenize{users_guide/west/running:overview}}
The \sphinxstylestrong{w\_run} command is used to run weighted ensemble simulations
\sphinxtitleref{configured \textless{}setup\textgreater{}} with \sphinxstylestrong{w\_init}.


\subsubsection{Setting simulation limits}
\label{\detokenize{users_guide/west/running:setting-simulation-limits}}

\subsubsection{Running a simulation}
\label{\detokenize{users_guide/west/running:running-a-simulation}}

\paragraph{Running on a single node}
\label{\detokenize{users_guide/west/running:running-on-a-single-node}}

\paragraph{Running on multiple nodes with MPI}
\label{\detokenize{users_guide/west/running:running-on-multiple-nodes-with-mpi}}

\paragraph{Running on multiple nodes with ZeroMQ}
\label{\detokenize{users_guide/west/running:running-on-multiple-nodes-with-zeromq}}

\subsubsection{Managing data}
\label{\detokenize{users_guide/west/running:managing-data}}

\subsubsection{Recovering from errors}
\label{\detokenize{users_guide/west/running:recovering-from-errors}}
By default, information about simulation progress is stored in
\sphinxstylestrong{west\sphinxhyphen{}JOBID.log} (where JOBID refers to the job ID given by the submission
engine); any errors will be logged here.
\begin{itemize}
\item {} 
The error “could not read pcoord from ‘tempfile’: progress coordinate has
incorrect shape” may come about from multiple causes; it is possible that the
progress coordinate length is incorrectly specified in system.py
(\sphinxstylestrong{self.pcoord\_len}), or that GROMACS (or whatever simulation package you
are using) had an error during the simulation.

\item {} 
The first case will be obvious by what comes after the message: (XX, YY)
(where XX is non\sphinxhyphen{}zero), expected (ZZ, GG) (whatever is in system.py). This
can be corrected by adjusting system.py.

\item {} 
In the second case, the progress coordinate length is 0; this
indicates that no progress coordinate data exists (null string), which
implies that the simulation software did not complete successfully. By
default, the simulation package (GROMACS or otherwise) terminal output is
stored in a log file inside of seg\_logs. Any error that occurred during the
actual simulation will be logged here, and can be corrected as needed.

\end{itemize}


\subsection{Analysis}
\label{\detokenize{users_guide/west/analysis:analysis}}\label{\detokenize{users_guide/west/analysis::doc}}

\subsubsection{Gauging simulation progress and convergence}
\label{\detokenize{users_guide/west/analysis:gauging-simulation-progress-and-convergence}}

\paragraph{Progress coordinate distribution (w\_pcpdist)}
\label{\detokenize{users_guide/west/analysis:progress-coordinate-distribution-w-pcpdist}}
w\_pcpdist and plothist


\paragraph{Kinetics for source/sink simulations}
\label{\detokenize{users_guide/west/analysis:kinetics-for-source-sink-simulations}}
w\_fluxanl


\paragraph{Kinetics for arbitrary state definitions}
\label{\detokenize{users_guide/west/analysis:kinetics-for-arbitrary-state-definitions}}
In order to calculate rate constants, it is necessary to run three different
tools:

\begin{sphinxVerbatim}[commandchars=\\\{\}]
\PYGZhy{} :ref:`w\PYGZus{}assign`
\PYGZhy{} :ref:`w\PYGZus{}kinetics`
\PYGZhy{} :ref:`w\PYGZus{}kinavg`
\end{sphinxVerbatim}

The w\_assign tool assigns trajectories to states (states which correspond to a
target bin) at a sub\sphinxhyphen{}tau resolution. This allows w\_kinetics to properly trace
the trajectories and prepare the data for further analysis.

Although the bin and state definitions can be pulled from the system, it is
frequently more convenient to specify custom bin boundaries and states; this
eliminates the need to know what constitutes a state prior to starting the
simulation. Both files must be in the YAML format, of which there are numerous
examples of online. A quick example for each file follows:

\begin{sphinxVerbatim}[commandchars=\\\{\}]
\PYG{n}{States}\PYG{p}{:}
\PYG{o}{\PYGZhy{}}\PYG{o}{\PYGZhy{}}\PYG{o}{\PYGZhy{}}
\PYG{n}{states}\PYG{p}{:}
  \PYG{o}{\PYGZhy{}} \PYG{n}{label}\PYG{p}{:} \PYG{n}{unbound}
    \PYG{n}{coords}\PYG{p}{:}
      \PYG{o}{\PYGZhy{}} \PYG{p}{[}\PYG{l+m+mi}{25}\PYG{p}{,}\PYG{l+m+mi}{0}\PYG{p}{]}
  \PYG{o}{\PYGZhy{}} \PYG{n}{label}\PYG{p}{:} \PYG{n}{boun}
    \PYG{n}{coords}\PYG{p}{:}
      \PYG{o}{\PYGZhy{}} \PYG{p}{[}\PYG{l+m+mf}{1.5}\PYG{p}{,}\PYG{l+m+mf}{33.0}\PYG{p}{]}

\PYG{n}{Bins}\PYG{p}{:}
\PYG{o}{\PYGZhy{}}\PYG{o}{\PYGZhy{}}\PYG{o}{\PYGZhy{}}
\PYG{n}{bins}\PYG{p}{:}
  \PYG{n+nb}{type}\PYG{p}{:} \PYG{n}{RectilinearBinMapper}
  \PYG{n}{boundaries}\PYG{p}{:} \PYG{p}{[}\PYG{p}{[}\PYG{l+m+mf}{0.0}\PYG{p}{,}\PYG{l+m+mf}{1.57}\PYG{p}{,}\PYG{l+m+mf}{25.0}\PYG{p}{,}\PYG{l+m+mi}{10000}\PYG{p}{]}\PYG{p}{,}\PYG{p}{[}\PYG{l+m+mf}{0.0}\PYG{p}{,}\PYG{l+m+mf}{33.0}\PYG{p}{,}\PYG{l+m+mi}{10000}\PYG{p}{]}\PYG{p}{]}
\end{sphinxVerbatim}

This system has a two dimensional progress coordinate, and two definite states,
as defined by the PMF. The binning used during the simulation was significantly
more complex; defining a smaller progress coordinate (in which we have three
regions: bound, unbound, and in between) is simply a matter of convenience.
Note that these custom bins do not change the simulation in any fashion; you
can adjust state definitions and bin boundaries at will without altering the
way the simulation runs.

The help definition, included by running:

\begin{sphinxVerbatim}[commandchars=\\\{\}]
\PYG{n}{w\PYGZus{}assign} \PYG{o}{\PYGZhy{}}\PYG{o}{\PYGZhy{}}\PYG{n}{help}
\end{sphinxVerbatim}

usually contains the most up\sphinxhyphen{}to\sphinxhyphen{}date help information, and so more
information about command line options can be obtained from there. To
run with the above YAML files, assuming they are named STATES and BINS,
you would run the following command:

\begin{sphinxVerbatim}[commandchars=\\\{\}]
\PYG{n}{w\PYGZus{}assign} \PYG{o}{\PYGZhy{}}\PYG{o}{\PYGZhy{}}\PYG{n}{states}\PYG{o}{\PYGZhy{}}\PYG{n}{from}\PYG{o}{\PYGZhy{}}\PYG{n}{file} \PYG{n}{STATES} \PYG{o}{\PYGZhy{}}\PYG{o}{\PYGZhy{}}\PYG{n}{bins}\PYG{o}{\PYGZhy{}}\PYG{n}{from}\PYG{o}{\PYGZhy{}}\PYG{n}{file} \PYG{n}{BINS}
\end{sphinxVerbatim}

By default, this produces a .h5 file (named assign.h5); this can be changed via
the command line.

The w\_kinetics tool uses the information generated from w\_assign to trace
through trajectories and calculate flux with included color information. There
are two main methods to run w\_kinetics:

\begin{sphinxVerbatim}[commandchars=\\\{\}]
\PYG{n}{w\PYGZus{}kinetics} \PYG{n}{trace}
\PYG{n}{w\PYGZus{}kinetics} \PYG{n}{matrix}
\end{sphinxVerbatim}

The matrix method is still in development; at this time, trace is the
recommended method.

Once the w\_kinetics analysis is complete, you can check for convergence of the
rate constants. WESTPA includes two tools to help you do this: w\_kinavg and
ploterr. First, begin by running the following command (keep in mind that
w\_kinavg has the same type of analysis as w\_kinetics does; whatever method you
chose (trace or matrix) in the w\_kinetics step should be used here, as well):

\begin{sphinxVerbatim}[commandchars=\\\{\}]
\PYG{n}{w\PYGZus{}kinavg} \PYG{n}{trace} \PYG{o}{\PYGZhy{}}\PYG{n}{e} \PYG{n}{cumulative}
\end{sphinxVerbatim}

This instructs w\_kinavg to produce a .h5 file with the cumulative rate
information; by then using ploterr, you can determine whether the rates
have stopped changing:

\begin{sphinxVerbatim}[commandchars=\\\{\}]
\PYG{n}{ploterr} \PYG{n}{kinavg}
\end{sphinxVerbatim}

By default, this produces a set of .pdf files, containing cumulative rate and
flux information for each state\sphinxhyphen{}to\sphinxhyphen{}state transition as a function of the WESTPA
iteration. Determine at which iteration the rate stops changing; then, rerun
w\_kinavg with the following systems:

\begin{sphinxVerbatim}[commandchars=\\\{\}]
\PYG{n}{w\PYGZus{}kinavg} \PYG{n}{trace} \PYG{o}{\PYGZhy{}}\PYG{o}{\PYGZhy{}}\PYG{n}{first}\PYG{o}{\PYGZhy{}}\PYG{n+nb}{iter} \PYG{n}{ITER}
\end{sphinxVerbatim}

where ITER is the beginning of the unchanging region. This will then
output information much like the following:

\begin{sphinxVerbatim}[commandchars=\\\{\}]
\PYG{n}{fluxes} \PYG{n}{into} \PYG{n}{macrostates}\PYG{p}{:}
\PYG{n}{unbound}\PYG{p}{:} \PYG{n}{mean}\PYG{o}{=}\PYG{l+m+mf}{1.712580005863456e\PYGZhy{}02} \PYG{n}{CI}\PYG{o}{=}\PYG{p}{(}\PYG{l+m+mf}{1.596595628304422e\PYGZhy{}02}\PYG{p}{,} \PYG{l+m+mf}{1.808249529394858e\PYGZhy{}02}\PYG{p}{)} \PYG{o}{*} \PYG{n}{tau}\PYG{o}{\PYGZca{}}\PYG{o}{\PYGZhy{}}\PYG{l+m+mi}{1}
\PYG{n}{bound}  \PYG{p}{:} \PYG{n}{mean}\PYG{o}{=}\PYG{l+m+mf}{5.944989301935855e\PYGZhy{}04} \PYG{n}{CI}\PYG{o}{=}\PYG{p}{(}\PYG{l+m+mf}{4.153556214886056e\PYGZhy{}04}\PYG{p}{,} \PYG{l+m+mf}{7.789568983584020e\PYGZhy{}04}\PYG{p}{)} \PYG{o}{*} \PYG{n}{tau}\PYG{o}{\PYGZca{}}\PYG{o}{\PYGZhy{}}\PYG{l+m+mi}{1}

\PYG{n}{fluxes} \PYG{k+kn}{from} \PYG{n+nn}{state} \PYG{n}{to} \PYG{n}{state}\PYG{p}{:}
\PYG{n}{unbound} \PYG{o}{\PYGZhy{}}\PYG{o}{\PYGZgt{}} \PYG{n}{bound}  \PYG{p}{:} \PYG{n}{mean}\PYG{o}{=}\PYG{l+m+mf}{5.944989301935855e\PYGZhy{}04} \PYG{n}{CI}\PYG{o}{=}\PYG{p}{(}\PYG{l+m+mf}{4.253003401668849e\PYGZhy{}04}\PYG{p}{,} \PYG{l+m+mf}{7.720997503648696e\PYGZhy{}04}\PYG{p}{)} \PYG{o}{*} \PYG{n}{tau}\PYG{o}{\PYGZca{}}\PYG{o}{\PYGZhy{}}\PYG{l+m+mi}{1}
\PYG{n}{bound}   \PYG{o}{\PYGZhy{}}\PYG{o}{\PYGZgt{}} \PYG{n}{unbound}\PYG{p}{:} \PYG{n}{mean}\PYG{o}{=}\PYG{l+m+mf}{1.712580005863456e\PYGZhy{}02} \PYG{n}{CI}\PYG{o}{=}\PYG{p}{(}\PYG{l+m+mf}{1.590547796439216e\PYGZhy{}02}\PYG{p}{,} \PYG{l+m+mf}{1.808154616175579e\PYGZhy{}02}\PYG{p}{)} \PYG{o}{*} \PYG{n}{tau}\PYG{o}{\PYGZca{}}\PYG{o}{\PYGZhy{}}\PYG{l+m+mi}{1}

\PYG{n}{rates} \PYG{k+kn}{from} \PYG{n+nn}{state} \PYG{n}{to} \PYG{n}{state}\PYG{p}{:}
\PYG{n}{unbound} \PYG{o}{\PYGZhy{}}\PYG{o}{\PYGZgt{}} \PYG{n}{bound}  \PYG{p}{:} \PYG{n}{mean}\PYG{o}{=}\PYG{l+m+mf}{9.972502012305491e\PYGZhy{}03} \PYG{n}{CI}\PYG{o}{=}\PYG{p}{(}\PYG{l+m+mf}{7.165030136921814e\PYGZhy{}03}\PYG{p}{,} \PYG{l+m+mf}{1.313767180582492e\PYGZhy{}02}\PYG{p}{)} \PYG{o}{*} \PYG{n}{tau}\PYG{o}{\PYGZca{}}\PYG{o}{\PYGZhy{}}\PYG{l+m+mi}{1}
\PYG{n}{bound}   \PYG{o}{\PYGZhy{}}\PYG{o}{\PYGZgt{}} \PYG{n}{unbound}\PYG{p}{:} \PYG{n}{mean}\PYG{o}{=}\PYG{l+m+mf}{1.819520888349874e\PYGZhy{}02} \PYG{n}{CI}\PYG{o}{=}\PYG{p}{(}\PYG{l+m+mf}{1.704608273094848e\PYGZhy{}02}\PYG{p}{,} \PYG{l+m+mf}{1.926165865735958e\PYGZhy{}02}\PYG{p}{)} \PYG{o}{*} \PYG{n}{tau}\PYG{o}{\PYGZca{}}\PYG{o}{\PYGZhy{}}\PYG{l+m+mi}{1}
\end{sphinxVerbatim}

Divide by tau to calculate your rate constant.


\section{WEST Tools}
\label{\detokenize{users_guide/west_tools:west-tools}}\label{\detokenize{users_guide/west_tools::doc}}
The command line tools included with the WESTPA software package are broadly
separable into two categories: \sphinxstylestrong{Tools for initializing a simulation} and
\sphinxstylestrong{tools for analyzing results}.

Command function can be user defined and modified. The particular parameters of
different command line tools are specified, in order of precedence, by:
\begin{itemize}
\item {} 
User specified command line arguments

\item {} 
User defined environmental variables

\item {} 
Package defaults

\end{itemize}

This page focuses on outlining the general functionality of the command line
tools and providing an overview of command line arguments that are shared by
multiple tools. See the {\hyperref[\detokenize{users_guide/command_line_tools:command-line-tool-index}]{\sphinxcrossref{\DUrole{std,std-ref}{index of command\sphinxhyphen{}line tools}}}} for a more comprehensive overview of each tool.


\subsection{Overview}
\label{\detokenize{users_guide/west_tools:overview}}
All tools are located in the \sphinxcode{\sphinxupquote{\$WEST\_ROOT/bin}} directory, where the shell
variable \sphinxcode{\sphinxupquote{WEST\_ROOT}} points to the path where the WESTPA package is located
on your machine.

You may wish to set this variable automatically by adding the following to your
\sphinxcode{\sphinxupquote{\textasciitilde{}/.bashrc}} or \sphinxcode{\sphinxupquote{\textasciitilde{}/.profile}} file:

\begin{sphinxVerbatim}[commandchars=\\\{\}]
\PYG{n}{export} \PYG{n}{WEST\PYGZus{}ROOT}\PYG{o}{=}\PYG{l+s+s2}{\PYGZdq{}}\PYG{l+s+s2}{\PYGZdl{}HOME/westpa}\PYG{l+s+s2}{\PYGZdq{}}
\end{sphinxVerbatim}

where the path to the westpa suite is modified accordingly.


\subsubsection{Tools for setting up and running a simulation}
\label{\detokenize{users_guide/west_tools:tools-for-setting-up-and-running-a-simulation}}
Use the following commands to initialize, configure, and run a weighted
ensemble simulation. Command line arguments or environmental variables can be
set to specify the work managers for running the simulation, where
configuration data is read from, and the \sphinxstyleemphasis{HDF5} file in which results are
stored.


\begin{savenotes}\sphinxattablestart
\centering
\begin{tabulary}{\linewidth}[t]{|T|T|}
\hline
\sphinxstyletheadfamily 
Command
&\sphinxstyletheadfamily 
Function
\\
\hline
{\hyperref[\detokenize{users_guide/command_line_tools/w_init:w-init}]{\sphinxcrossref{\DUrole{std,std-ref}{w\_init}}}}
&
Initializes simulation configuration files and environment.
Always run this command before starting a new simulation.
\\
\hline
{\hyperref[\detokenize{users_guide/command_line_tools/w_bins:w-bins}]{\sphinxcrossref{\DUrole{std,std-ref}{w\_bins}}}}
&
Set up binning, progress coordinate
\\
\hline
{\hyperref[\detokenize{users_guide/command_line_tools/w_run:w-run}]{\sphinxcrossref{\DUrole{std,std-ref}{w\_run}}}}
&
Launches a simulation. Command arguments/environmental
variables can be included to specify the work managers and
simulation parameters
\\
\hline
{\hyperref[\detokenize{users_guide/command_line_tools/w_truncate:w-truncate}]{\sphinxcrossref{\DUrole{std,std-ref}{w\_truncate}}}}
&
Truncates the weighted ensemble simulation from a given
iteration.
\\
\hline
\end{tabulary}
\par
\sphinxattableend\end{savenotes}


\subsubsection{Tools for analyzing simulation results}
\label{\detokenize{users_guide/west_tools:tools-for-analyzing-simulation-results}}
The following command line tools are provided for analysis after running a
weighted ensemble simulation (and collecting the results in an HDF5 file).

With the exception of the plotting tool \sphinxcode{\sphinxupquote{plothist}}, all analysis tools read
from and write to \sphinxstyleemphasis{HDF5} type files.


\begin{savenotes}\sphinxattablestart
\centering
\begin{tabulary}{\linewidth}[t]{|T|T|}
\hline
\sphinxstyletheadfamily 
Command
&\sphinxstyletheadfamily 
Function
\\
\hline
{\hyperref[\detokenize{users_guide/command_line_tools/w_assign:w-assign}]{\sphinxcrossref{\DUrole{std,std-ref}{w\_assign}}}}
&
Assign walkers to bins and macrostates (using simulation
output as input). Must be done before some other analysis
tools (e.g. {\hyperref[\detokenize{users_guide/command_line_tools/w_kinetics:w-kinetics}]{\sphinxcrossref{\DUrole{std,std-ref}{w\_kinetics}}}}, {\hyperref[\detokenize{users_guide/command_line_tools/w_kinavg:w-kinavg}]{\sphinxcrossref{\DUrole{std,std-ref}{w\_kinavg}}}})
\\
\hline
{\hyperref[\detokenize{users_guide/command_line_tools/w_trace:w-trace}]{\sphinxcrossref{\DUrole{std,std-ref}{w\_trace}}}}
&
Trace the path of a given walker segment over a
user\sphinxhyphen{}specified number of simulation iterations.
\\
\hline
{\hyperref[\detokenize{users_guide/command_line_tools/w_fluxanl:w-fluxanl}]{\sphinxcrossref{\DUrole{std,std-ref}{w\_fluxanl}}}}
&
Calculate average probability flux into user\sphinxhyphen{}defined
‘target’ state with relevant statistics.
\\
\hline
{\hyperref[\detokenize{users_guide/command_line_tools/w_pdist:w-pdist}]{\sphinxcrossref{\DUrole{std,std-ref}{w\_pdist}}}}
&
Construct a probability distribution of results (e.g.
progress coordinate membership) for subsequent plotting
with {\hyperref[\detokenize{users_guide/command_line_tools/plothist:plothist}]{\sphinxcrossref{\DUrole{std,std-ref}{plothist}}}}.
\\
\hline
{\hyperref[\detokenize{users_guide/command_line_tools/plothist:plothist}]{\sphinxcrossref{\DUrole{std,std-ref}{plothist}}}}
&
Tool to plot output from other analysis tools (e.g.
{\hyperref[\detokenize{users_guide/command_line_tools/w_pdist:w-pdist}]{\sphinxcrossref{\DUrole{std,std-ref}{w\_pdist}}}}).
\\
\hline
\end{tabulary}
\par
\sphinxattableend\end{savenotes}


\subsection{General Command Line Options}
\label{\detokenize{users_guide/west_tools:general-command-line-options}}
The following arguments are shared by all command line tools:

\begin{sphinxVerbatim}[commandchars=\\\{\}]
\PYG{o}{\PYGZhy{}}\PYG{n}{r} \PYG{n}{config} \PYG{n}{file}\PYG{p}{,} \PYG{o}{\PYGZhy{}}\PYG{o}{\PYGZhy{}}\PYG{n}{rcfile} \PYG{n}{config} \PYG{n}{file}
  \PYG{n}{Use} \PYG{n}{config} \PYG{n}{file} \PYG{k}{as} \PYG{n}{the} \PYG{n}{configuration} \PYG{n}{file} \PYG{p}{(}\PYG{n}{Default}\PYG{p}{:} \PYG{n}{File} \PYG{n}{named} \PYG{n}{west}\PYG{o}{.}\PYG{n}{cfg}\PYG{p}{)}
\PYG{o}{\PYGZhy{}}\PYG{o}{\PYGZhy{}}\PYG{n}{quiet}\PYG{p}{,} \PYG{o}{\PYGZhy{}}\PYG{o}{\PYGZhy{}}\PYG{n}{verbose}\PYG{p}{,} \PYG{o}{\PYGZhy{}}\PYG{o}{\PYGZhy{}}\PYG{n}{debug}
  \PYG{n}{Specify} \PYG{n}{command} \PYG{n}{tool} \PYG{n}{output} \PYG{n}{verbosity} \PYG{p}{(}\PYG{n}{Default}\PYG{p}{:} \PYG{l+s+s1}{\PYGZsq{}}\PYG{l+s+s1}{quiet}\PYG{l+s+s1}{\PYGZsq{}} \PYG{n}{mode}\PYG{p}{)}
\PYG{o}{\PYGZhy{}}\PYG{o}{\PYGZhy{}}\PYG{n}{version}
  \PYG{n}{Print} \PYG{n}{WESTPA} \PYG{n}{version} \PYG{n}{number} \PYG{o+ow}{and} \PYG{n}{exit}
\PYG{o}{\PYGZhy{}}\PYG{n}{h}\PYG{p}{,} \PYG{o}{\PYGZhy{}}\PYG{o}{\PYGZhy{}}\PYG{n}{help}
  \PYG{n}{Output} \PYG{n}{the} \PYG{n}{help} \PYG{n}{information} \PYG{k}{for} \PYG{n}{this} \PYG{n}{command} \PYG{n}{line} \PYG{n}{tool} \PYG{o+ow}{and} \PYG{n}{exit}
\end{sphinxVerbatim}


\subsubsection{A note on specifying a configuration file}
\label{\detokenize{users_guide/west_tools:a-note-on-specifying-a-configuration-file}}
A \sphinxstyleemphasis{configuration file}, which should be stored in your simulation root
directory, is read by all command line tools. The \sphinxstyleemphasis{configuration file}
specifies parameters for general simulation setup, as well as the \sphinxstyleemphasis{hdf5} file
name where simulation data is stored and read by analysis tools.

If not specified, the \sphinxstylestrong{default configuration file} is assumed to be named
\sphinxstylestrong{west.cfg}.

You can override this to use configuration file \sphinxstyleemphasis{file} by either:
\begin{itemize}
\item {} 
Setting the environmental variable \sphinxcode{\sphinxupquote{WESTRC}} equal to \sphinxstyleemphasis{file}:

\begin{sphinxVerbatim}[commandchars=\\\{\}]
\PYG{n}{export} \PYG{n}{WESTRC}\PYG{o}{=}\PYG{o}{/}\PYG{n}{path}\PYG{o}{/}\PYG{n}{to}\PYG{o}{/}\PYG{n}{westrcfile}
\end{sphinxVerbatim}

\item {} 
Including the command line argument \sphinxcode{\sphinxupquote{\sphinxhyphen{}r /path/to/westrcfile}}

\end{itemize}


\subsection{Work Manager Options}
\label{\detokenize{users_guide/west_tools:work-manager-options}}
Note: See {\hyperref[\detokenize{users_guide/wwmgr:wwmgr}]{\sphinxcrossref{\DUrole{std,std-ref}{wwmgr overview}}}} for a more detailed explanation of the
work manager framework.

Work managers a used by a number of command\sphinxhyphen{}line tools to process more complex
tasks, especially in setting up and running simulations (i.e. {\hyperref[\detokenize{users_guide/command_line_tools/w_init:w-init}]{\sphinxcrossref{\DUrole{std,std-ref}{w\_init}}}} and
{\hyperref[\detokenize{users_guide/command_line_tools/w_run:w-run}]{\sphinxcrossref{\DUrole{std,std-ref}{w\_run}}}}) \sphinxhyphen{} in general, work managers are involved in tasks that require
multiprocessing and/or tasks distributed over multiple nodes in a cluster.


\subsubsection{Overview}
\label{\detokenize{users_guide/west_tools:id1}}
The following command\sphinxhyphen{}line tools make use of work managers:
\begin{itemize}
\item {} 
{\hyperref[\detokenize{users_guide/command_line_tools/w_init:w-init}]{\sphinxcrossref{\DUrole{std,std-ref}{w\_init}}}}

\item {} 
{\hyperref[\detokenize{users_guide/command_line_tools/w_run:w-run}]{\sphinxcrossref{\DUrole{std,std-ref}{w\_run}}}}

\end{itemize}


\subsubsection{General work manager options}
\label{\detokenize{users_guide/west_tools:general-work-manager-options}}
The following are general options used for specifying the type of work
manager and number of cores:

\begin{sphinxVerbatim}[commandchars=\\\{\}]
\PYG{o}{\PYGZhy{}}\PYG{o}{\PYGZhy{}}\PYG{n}{wm}\PYG{o}{\PYGZhy{}}\PYG{n}{work}\PYG{o}{\PYGZhy{}}\PYG{n}{manager} \PYG{n}{work\PYGZus{}manager}
  \PYG{n}{Specify} \PYG{n}{which} \PYG{n+nb}{type} \PYG{n}{of} \PYG{n}{work} \PYG{n}{manager} \PYG{n}{to} \PYG{n}{use}\PYG{p}{,} \PYG{n}{where} \PYG{n}{the} \PYG{n}{possible} \PYG{n}{choices} \PYG{k}{for}
  \PYG{n}{work\PYGZus{}manager} \PYG{n}{are}\PYG{p}{:} \PYG{p}{\PYGZob{}}\PYG{n}{processes}\PYG{p}{,} \PYG{n}{gcserial}\PYG{p}{,} \PYG{n}{threads}\PYG{p}{,} \PYG{n}{mpi}\PYG{p}{,} \PYG{o+ow}{or} \PYG{n}{zmq}\PYG{p}{\PYGZcb{}}\PYG{o}{.} \PYG{n}{See} \PYG{n}{the}
  \PYG{n}{wwmgr} \PYG{n}{overview} \PYG{n}{page} \PYG{o}{\PYGZlt{}}\PYG{n}{wwmgr}\PYG{o}{\PYGZgt{}}\PYG{n}{\PYGZus{}} \PYG{k}{for} \PYG{n}{more} \PYG{n}{information} \PYG{n}{on} \PYG{n}{the} \PYG{n}{different} \PYG{n}{types} \PYG{n}{of}
  \PYG{n}{work} \PYG{n}{managers} \PYG{p}{(}\PYG{n}{Default}\PYG{p}{:} \PYG{n}{gcprocesses}\PYG{p}{)}
\PYG{o}{\PYGZhy{}}\PYG{o}{\PYGZhy{}}\PYG{n}{wm}\PYG{o}{\PYGZhy{}}\PYG{n}{n}\PYG{o}{\PYGZhy{}}\PYG{n}{workers} \PYG{n}{n\PYGZus{}workers}
  \PYG{n}{Specify} \PYG{n}{the} \PYG{n}{number} \PYG{n}{of} \PYG{n}{cores} \PYG{n}{to} \PYG{n}{use} \PYG{k}{as} \PYG{n}{gcn\PYGZus{}workers}\PYG{p}{,} \PYG{k}{if} \PYG{n}{the} \PYG{n}{work} \PYG{n}{manager} \PYG{n}{you}
  \PYG{n}{selected} \PYG{n}{supports} \PYG{n}{this} \PYG{n}{option} \PYG{p}{(}\PYG{n}{work} \PYG{n}{managers} \PYG{n}{that} \PYG{n}{do} \PYG{o+ow}{not} \PYG{n}{will} \PYG{n}{ignore} \PYG{n}{this}
  \PYG{n}{option}\PYG{p}{)}\PYG{o}{.} \PYG{n}{If} \PYG{n}{using} \PYG{n}{an} \PYG{n}{gcmpi} \PYG{o+ow}{or} \PYG{n}{zmq} \PYG{n}{work} \PYG{n}{manager}\PYG{p}{,} \PYG{n}{specify} \PYG{n}{gc}\PYG{o}{\PYGZhy{}}\PYG{o}{\PYGZhy{}}\PYG{n}{wm}\PYG{o}{\PYGZhy{}}\PYG{n}{n}\PYG{o}{\PYGZhy{}}\PYG{n}{workers}\PYG{o}{=}\PYG{l+m+mi}{0}
  \PYG{k}{for} \PYG{n}{a} \PYG{n}{dedicated} \PYG{n}{server} \PYG{p}{(}\PYG{n}{Default}\PYG{p}{:} \PYG{n}{Number} \PYG{n}{of} \PYG{n}{cores} \PYG{n}{available} \PYG{n}{on} \PYG{n}{machine}\PYG{p}{)}
\end{sphinxVerbatim}

The \sphinxcode{\sphinxupquote{mpi}} work manager is generally sufficient for most tasks that make use
of multiple nodes on a cluster. The \sphinxcode{\sphinxupquote{zmq}} work manager is preferable if the
\sphinxcode{\sphinxupquote{mpi}} work manager does not work properly on your cluster or if you prefer to
have more explicit control over the distribution of communication tasks on your
cluster.


\subsubsection{ZeroMQ (‘zmq’) work manager}
\label{\detokenize{users_guide/west_tools:zeromq-zmq-work-manager}}
The ZeroMQ work manager offers a number of additional options (all of
which are optional and have default values). All of these options focus
on whether the zmq work manager is set up as a server (i.e. task
distributor/ventilator) or client (task processor):

\begin{sphinxVerbatim}[commandchars=\\\{\}]
\PYG{o}{\PYGZhy{}}\PYG{o}{\PYGZhy{}}\PYG{n}{wm}\PYG{o}{\PYGZhy{}}\PYG{n}{zmq}\PYG{o}{\PYGZhy{}}\PYG{n}{mode} \PYG{n}{mode}
  \PYG{n}{Options}\PYG{p}{:} \PYG{p}{\PYGZob{}}\PYG{n}{server} \PYG{o+ow}{or} \PYG{n}{client}\PYG{p}{\PYGZcb{}}\PYG{o}{.} \PYG{n}{Specify} \PYG{n}{whether} \PYG{n}{the} \PYG{n}{ZMQ} \PYG{n}{work} \PYG{n}{manager} \PYG{n}{on} \PYG{n}{this}
  \PYG{n}{node} \PYG{n}{will} \PYG{n}{operate} \PYG{k}{as} \PYG{n}{a} \PYG{n}{server} \PYG{o+ow}{or} \PYG{n}{a} \PYG{n}{client} \PYG{p}{(}\PYG{n}{Default}\PYG{p}{:} \PYG{n}{server}\PYG{p}{)}

\PYG{o}{\PYGZhy{}}\PYG{o}{\PYGZhy{}}\PYG{n}{wm}\PYG{o}{\PYGZhy{}}\PYG{n}{zmq}\PYG{o}{\PYGZhy{}}\PYG{n}{info}\PYG{o}{\PYGZhy{}}\PYG{n}{file} \PYG{n}{info\PYGZus{}file}
  \PYG{n}{Specify} \PYG{n}{the} \PYG{n}{name} \PYG{n}{of} \PYG{n}{a} \PYG{n}{temporary} \PYG{n}{file} \PYG{n}{to} \PYG{n}{write} \PYG{p}{(}\PYG{k}{as} \PYG{n}{a} \PYG{n}{server}\PYG{p}{)} \PYG{o+ow}{or} \PYG{n}{read} \PYG{p}{(}\PYG{k}{as} \PYG{n}{a}
  \PYG{n}{client}\PYG{p}{)} \PYG{n}{socket} \PYG{n}{connection} \PYG{n}{endpoints} \PYG{p}{(}\PYG{n}{Default}\PYG{p}{:} \PYG{n}{server\PYGZus{}x}\PYG{o}{.}\PYG{n}{json}\PYG{p}{,} \PYG{n}{where} \PYG{n}{x} \PYG{o+ow}{is} \PYG{n}{a}
  \PYG{n}{unique} \PYG{n}{identifier} \PYG{n}{string}\PYG{p}{)}

\PYG{o}{\PYGZhy{}}\PYG{o}{\PYGZhy{}}\PYG{n}{wm}\PYG{o}{\PYGZhy{}}\PYG{n}{zmq}\PYG{o}{\PYGZhy{}}\PYG{n}{task}\PYG{o}{\PYGZhy{}}\PYG{n}{endpoint} \PYG{n}{task\PYGZus{}endpoint}
  \PYG{n}{Explicitly} \PYG{n}{use} \PYG{n}{task\PYGZus{}endpoint} \PYG{n}{to} \PYG{n}{bind} \PYG{n}{to} \PYG{p}{(}\PYG{k}{as} \PYG{n}{server}\PYG{p}{)} \PYG{o+ow}{or} \PYG{n}{connect} \PYG{n}{to} \PYG{p}{(}\PYG{k}{as}
  \PYG{n}{client}\PYG{p}{)} \PYG{k}{for} \PYG{n}{task} \PYG{n}{distribution} \PYG{p}{(}\PYG{n}{Default}\PYG{p}{:} \PYG{n}{A} \PYG{n}{randomly} \PYG{n}{determined} \PYG{n}{endpoint} \PYG{n}{that}
  \PYG{o+ow}{is} \PYG{n}{written} \PYG{o+ow}{or} \PYG{n}{read} \PYG{k+kn}{from} \PYG{n+nn}{the} \PYG{n}{specified} \PYG{n}{info\PYGZus{}file}\PYG{p}{)}

\PYG{o}{\PYGZhy{}}\PYG{o}{\PYGZhy{}}\PYG{n}{wm}\PYG{o}{\PYGZhy{}}\PYG{n}{zmq}\PYG{o}{\PYGZhy{}}\PYG{n}{result}\PYG{o}{\PYGZhy{}}\PYG{n}{endpoint} \PYG{n}{result\PYGZus{}endpoint}
  \PYG{n}{Explicitly} \PYG{n}{use} \PYG{n}{result\PYGZus{}endpoint} \PYG{n}{to} \PYG{n}{bind} \PYG{n}{to} \PYG{p}{(}\PYG{k}{as} \PYG{n}{server}\PYG{p}{)} \PYG{o+ow}{or} \PYG{n}{connect} \PYG{n}{to} \PYG{p}{(}\PYG{k}{as}
  \PYG{n}{client}\PYG{p}{)} \PYG{n}{to} \PYG{n}{distribute} \PYG{o+ow}{and} \PYG{n}{collect} \PYG{n}{task} \PYG{n}{results} \PYG{p}{(}\PYG{n}{Default}\PYG{p}{:} \PYG{n}{A} \PYG{n}{randomly}
  \PYG{n}{determined} \PYG{n}{endpoint} \PYG{n}{that} \PYG{o+ow}{is} \PYG{n}{written} \PYG{n}{to} \PYG{o+ow}{or} \PYG{n}{read} \PYG{k+kn}{from} \PYG{n+nn}{the} \PYG{n}{specified}
  \PYG{n}{info\PYGZus{}file}\PYG{p}{)}

\PYG{o}{\PYGZhy{}}\PYG{o}{\PYGZhy{}}\PYG{n}{wm}\PYG{o}{\PYGZhy{}}\PYG{n}{zmq}\PYG{o}{\PYGZhy{}}\PYG{n}{announce}\PYG{o}{\PYGZhy{}}\PYG{n}{endpoint} \PYG{n}{announce\PYGZus{}endpoint}
  \PYG{n}{Explicitly} \PYG{n}{use} \PYG{n}{announce\PYGZus{}endpoint} \PYG{n}{to} \PYG{n}{bind} \PYG{n}{to} \PYG{p}{(}\PYG{k}{as} \PYG{n}{server}\PYG{p}{)} \PYG{o+ow}{or} \PYG{n}{connect} \PYG{n}{to} \PYG{p}{(}\PYG{k}{as}
  \PYG{n}{client}\PYG{p}{)} \PYG{n}{to} \PYG{n}{distribute} \PYG{n}{central} \PYG{n}{announcements} \PYG{p}{(}\PYG{n}{Default}\PYG{p}{:} \PYG{n}{A} \PYG{n}{randomly} \PYG{n}{determined}
  \PYG{n}{endpoint} \PYG{n}{that} \PYG{o+ow}{is} \PYG{n}{written} \PYG{n}{to} \PYG{o+ow}{or} \PYG{n}{read} \PYG{k+kn}{from} \PYG{n+nn}{the} \PYG{n}{specified} \PYG{n}{info\PYGZus{}file}\PYG{p}{)}

\PYG{o}{\PYGZhy{}}\PYG{o}{\PYGZhy{}}\PYG{n}{wm}\PYG{o}{\PYGZhy{}}\PYG{n}{zmq}\PYG{o}{\PYGZhy{}}\PYG{n}{heartbeat}\PYG{o}{\PYGZhy{}}\PYG{n}{interval} \PYG{n}{interval}
  \PYG{n}{If} \PYG{n}{a} \PYG{n}{server}\PYG{p}{,} \PYG{n}{send} \PYG{n}{an} \PYG{n}{Im} \PYG{n}{alive} \PYG{n}{ping} \PYG{n}{to} \PYG{n}{connected} \PYG{n}{clients} \PYG{n}{every} \PYG{n}{interval}
  \PYG{n}{seconds}\PYG{p}{;} \PYG{n}{If} \PYG{n}{a} \PYG{n}{client}\PYG{p}{,} \PYG{n}{expect} \PYG{n}{to} \PYG{n}{hear} \PYG{n}{a} \PYG{n}{server} \PYG{n}{ping} \PYG{n}{every} \PYG{n}{approximately}
  \PYG{n}{interval} \PYG{n}{seconds}\PYG{p}{,} \PYG{o+ow}{or} \PYG{k}{else} \PYG{n}{assume} \PYG{n}{the} \PYG{n}{server} \PYG{n}{has} \PYG{n}{crashed} \PYG{o+ow}{and} \PYG{n}{shutdown}
  \PYG{p}{(}\PYG{n}{Default}\PYG{p}{:} \PYG{l+m+mi}{600} \PYG{n}{seconds}\PYG{p}{)}

\PYG{o}{\PYGZhy{}}\PYG{o}{\PYGZhy{}}\PYG{n}{wm}\PYG{o}{\PYGZhy{}}\PYG{n}{zmq}\PYG{o}{\PYGZhy{}}\PYG{n}{task}\PYG{o}{\PYGZhy{}}\PYG{n}{timeout} \PYG{n}{timeout}
  \PYG{n}{Kill} \PYG{n}{worker} \PYG{n}{processes}\PYG{o}{/}\PYG{n}{jobs} \PYG{n}{after} \PYG{n}{that} \PYG{n}{take} \PYG{n}{longer} \PYG{n}{than} \PYG{n}{timeout} \PYG{n}{seconds} \PYG{n}{to}
  \PYG{n}{complete} \PYG{p}{(}\PYG{n}{Default}\PYG{p}{:} \PYG{n}{no} \PYG{n}{time} \PYG{n}{limit}\PYG{p}{)}

\PYG{o}{\PYGZhy{}}\PYG{o}{\PYGZhy{}}\PYG{n}{wm}\PYG{o}{\PYGZhy{}}\PYG{n}{zmq}\PYG{o}{\PYGZhy{}}\PYG{n}{client}\PYG{o}{\PYGZhy{}}\PYG{n}{comm}\PYG{o}{\PYGZhy{}}\PYG{n}{mode} \PYG{n}{mode}
  \PYG{n}{Use} \PYG{n}{the} \PYG{n}{communication} \PYG{n}{mode}\PYG{p}{,} \PYG{n}{mode}\PYG{p}{,} \PYG{p}{(}\PYG{n}{options}\PYG{p}{:} \PYG{p}{\PYGZob{}}\PYG{n}{ipc} \PYG{k}{for} \PYG{n}{Unix} \PYG{n}{sockets}\PYG{p}{,} \PYG{o+ow}{or} \PYG{n}{tcp}
  \PYG{k}{for} \PYG{n}{TCP}\PYG{o}{/}\PYG{n}{IP} \PYG{n}{sockets}\PYG{p}{\PYGZcb{}}\PYG{p}{)} \PYG{n}{to} \PYG{n}{communicate} \PYG{k}{with} \PYG{n}{worker} \PYG{n}{processes} \PYG{p}{(}\PYG{n}{Default}\PYG{p}{:} \PYG{n}{ipc}\PYG{p}{)}
\end{sphinxVerbatim}


\subsection{Initializing/Running Simulations}
\label{\detokenize{users_guide/west_tools:initializing-running-simulations}}
For a more complete overview of all the files necessary for setting up a
simulation, see the {\hyperref[\detokenize{users_guide/west/setup:setup}]{\sphinxcrossref{\DUrole{std,std-ref}{user guide for setting up a simulation}}}}


\section{WEST Work Manager}
\label{\detokenize{users_guide/wwmgr:west-work-manager}}\label{\detokenize{users_guide/wwmgr:wwmgr}}\label{\detokenize{users_guide/wwmgr::doc}}

\subsection{Introduction}
\label{\detokenize{users_guide/wwmgr:introduction}}
WWMGR is the parallel task distribution framework originally included as part
of the WEMD source. It was extracted to permit independent development, and
(more importantly) independent testing. A number of different schemes can be
selected at run\sphinxhyphen{}time for distributing work across multiple cores/nodes, as
follows:


\begin{savenotes}\sphinxattablestart
\centering
\begin{tabulary}{\linewidth}[t]{|T|T|T|T|T|}
\hline
\sphinxstyletheadfamily 
Name
&\sphinxstyletheadfamily 
Implementation
&\sphinxstyletheadfamily 
Multi\sphinxhyphen{}Core
&\sphinxstyletheadfamily 
Multi\sphinxhyphen{}Node
&\sphinxstyletheadfamily 
Appropriate For
\\
\hline
serial
&
None
&
No
&
No
&
Testing, minimizing overhead
when dynamics is inexpensive
\\
\hline
threads
&
Python “threading” module
&
Yes
&
No
&
Dynamics propagated by external
executables, large amounts of
data transferred per segment
\\
\hline
processes
&
Python “multiprocessing” module
&
Yes
&
No
&
Dynamics propagated by Python
routines, modest amounts of
data transferred per segment
\\
\hline
mpi
&
\sphinxhref{http://mpi4py.scipy.org/}{mpi4py}
compiled and linked against system MPI
&
Yes
&
Yes
&
Distributing calculations
across multiple nodes. Start
with this on your cluster of
choice.
\\
\hline
zmq
&
\sphinxhref{http://www.zeromq.org/}{ZeroMQ}
and \sphinxhref{http://zeromq.github.com/pyzmq/}{PyZMQ}
&
Yes
&
Yes
&
Distributing calculations
across multiple nodes. Use this
if MPI does not work properly
on your cluster (particularly
for spawning child processes).
\\
\hline
\end{tabulary}
\par
\sphinxattableend\end{savenotes}


\subsection{Environment variables}
\label{\detokenize{users_guide/wwmgr:environment-variables}}

\subsubsection{For controlling task distribution}
\label{\detokenize{users_guide/wwmgr:for-controlling-task-distribution}}
While the original WEMD work managers were controlled by command\sphinxhyphen{}line options
and entries in wemd.cfg, the new work manager is controlled using command\sphinxhyphen{}line
options or environment variables (much like OpenMP). These variables are as
follow:


\begin{savenotes}\sphinxattablestart
\centering
\begin{tabulary}{\linewidth}[t]{|T|T|T|T|}
\hline
\sphinxstyletheadfamily 
Variable
&\sphinxstyletheadfamily 
Applicable to
&\sphinxstyletheadfamily 
Default
&\sphinxstyletheadfamily 
Meaning
\\
\hline
WM\_WORK\_MANAGER
&
(none)
&
processes
&
Use the given task distribution
system: “serial”, “threads”,
“processes”, or “zmq”
\\
\hline
WM\_N\_WORKERS
&
threads, processes, zmq
&
number of cores in machine
&
Use this number of workers. In
the case of zmq, use this many
workers on the current machine
only (can be set independently
on different nodes).
\\
\hline
WM\_ZMQ\_MODE
&
zmq
&
server
&
Start as a server (“server”) or
a client (“client”). Servers
coordinate a given calculation,
and clients execute tasks
related to that calculation.
\\
\hline
WM\_ZMQ\_TASK\_TIMEOUT
&
zmq
&
60
&
Time (in seconds) after which a
worker will be considered hung,
terminated, and restarted. This
\sphinxstylestrong{must} be updated for
long\sphinxhyphen{}running dynamics segments.
Set to zero to disable hang
checks entirely.
\\
\hline
WM\_ZMQ\_TASK\_ENDPOINT
&
zmq
&
Random port
&
Master distributes tasks at
this address
\\
\hline
WM\_ZMQ\_RESULT\_ENDPOINT
&
zmq
&
Random port
&
Master receives task results at
this address                                                                                                                                                           |
\\
\hline
WM\_ZMQ\_ANNOUNCE\_ENDPOINT
&
zmq
&
Random port
&
Master publishes announcements
(such as “shut down now”) at
this address
\\
\hline
WM\_ZMQ\_SERVER\_INFO
&
zmq
&
\sphinxcode{\sphinxupquote{zmq\_server\_info\_PID\_ID.json}}
(where PID is a process ID and
ID is a nearly random hex number)
&
A file describing the above
endpoints can be found here (to
ease cluster\sphinxhyphen{}wide startup)
\\
\hline
\end{tabulary}
\par
\sphinxattableend\end{savenotes}


\subsubsection{For passing information to workers}
\label{\detokenize{users_guide/wwmgr:for-passing-information-to-workers}}
One environment variable is made available by multi\sphinxhyphen{}process work managers
(processes and ZMQ) to help clients configure themselves (e.g. select an
appropriate GPU on a multi\sphinxhyphen{}GPU node):


\begin{savenotes}\sphinxattablestart
\centering
\begin{tabulary}{\linewidth}[t]{|T|T|T|}
\hline
\sphinxstyletheadfamily 
Variable
&\sphinxstyletheadfamily 
Applicable to
&\sphinxstyletheadfamily 
Meaning
\\
\hline
WM\_PROCESS\_ID
&
processes, zmq
&
Contains an integer, 0 based, identifying the
process among the set of processes started on a
given node.
\\
\hline
\end{tabulary}
\par
\sphinxattableend\end{savenotes}


\subsection{The ZeroMQ work manager for clusters}
\label{\detokenize{users_guide/wwmgr:the-zeromq-work-manager-for-clusters}}
The ZeroMQ (“zmq”) work manager can be used for both single\sphinxhyphen{}machine and
cluster\sphinxhyphen{}wide communication. Communication occurs over sockets using the \sphinxhref{http://www.zeromq.org/}{ZeroMQ} messaging protocol. Within nodes, \sphinxhref{http://en.wikipedia.org/wiki/UNIX\_socket}{Unix sockets} are used for efficient
communication, while between nodes, TCP sockets are used. This also minimizes
the number of open sockets on the master node.

The quick and dirty guide to using this on a cluster is as follows:

\begin{sphinxVerbatim}[commandchars=\\\{\}]
source env.sh
export WM\PYGZus{}WORK\PYGZus{}MANAGER=zmq
export WM\PYGZus{}ZMQ\PYGZus{}COMM\PYGZus{}MODE=tcp
export WM\PYGZus{}ZMQ\PYGZus{}SERVER\PYGZus{}INFO=\PYGZdl{}WEST\PYGZus{}SIM\PYGZus{}ROOT/wemd\PYGZus{}server\PYGZus{}info.json

w\PYGZus{}run \PYGZam{}

\PYGZsh{} manually run w\PYGZus{}run on each client node, as appropriate for your batch system
\PYGZsh{} e.g. qrsh \PYGZhy{}inherit for Grid Engine, or maybe just simple SSH

for host in \PYGZdl{}(cat \PYGZdl{}TMPDIR/machines | sort | uniq); do
   qrsh \PYGZhy{}inherit \PYGZhy{}V \PYGZdl{}host \PYGZdl{}PWD/node\PYGZhy{}ltc1.sh \PYGZam{}
done
\end{sphinxVerbatim}


\section{WEST Extensions}
\label{\detokenize{users_guide/westext:west-extensions}}\label{\detokenize{users_guide/westext::doc}}

\subsection{Post\sphinxhyphen{}Analysis Reweighting}
\label{\detokenize{users_guide/westext:post-analysis-reweighting}}

\subsection{String Method}
\label{\detokenize{users_guide/westext:string-method}}

\subsection{Weighted Ensemble Equilibrium Dynamics}
\label{\detokenize{users_guide/westext:weighted-ensemble-equilibrium-dynamics}}

\subsection{Weighted Ensemble Steady State}
\label{\detokenize{users_guide/westext:weighted-ensemble-steady-state}}

\section{Command Line Tool Index}
\label{\detokenize{users_guide/command_line_tools:command-line-tool-index}}\label{\detokenize{users_guide/command_line_tools:id1}}\label{\detokenize{users_guide/command_line_tools::doc}}

\subsection{w\_assign}
\label{\detokenize{users_guide/command_line_tools/w_assign:w-assign}}\label{\detokenize{users_guide/command_line_tools/w_assign:id1}}\label{\detokenize{users_guide/command_line_tools/w_assign::doc}}
\sphinxcode{\sphinxupquote{w\_assign}} uses simulation output to assign walkers to user\sphinxhyphen{}specified bins
and macrostates. These assignments are required for some other simulation
tools, namely \sphinxcode{\sphinxupquote{w\_kinetics}} and \sphinxcode{\sphinxupquote{w\_kinavg}}.

\sphinxcode{\sphinxupquote{w\_assign}} supports parallelization (see \sphinxhref{UserGuide:ToolRefs\#Work\_Manager\_Options}{general work manager options} for more on command line options
to specify a work manager).


\subsubsection{Overview}
\label{\detokenize{users_guide/command_line_tools/w_assign:overview}}
Usage:

\begin{sphinxVerbatim}[commandchars=\\\{\}]
\PYG{n}{w\PYGZus{}assign} \PYG{p}{[}\PYG{o}{\PYGZhy{}}\PYG{n}{h}\PYG{p}{]} \PYG{p}{[}\PYG{o}{\PYGZhy{}}\PYG{n}{r} \PYG{n}{RCFILE}\PYG{p}{]} \PYG{p}{[}\PYG{o}{\PYGZhy{}}\PYG{o}{\PYGZhy{}}\PYG{n}{quiet} \PYG{o}{|} \PYG{o}{\PYGZhy{}}\PYG{o}{\PYGZhy{}}\PYG{n}{verbose} \PYG{o}{|} \PYG{o}{\PYGZhy{}}\PYG{o}{\PYGZhy{}}\PYG{n}{debug}\PYG{p}{]} \PYG{p}{[}\PYG{o}{\PYGZhy{}}\PYG{o}{\PYGZhy{}}\PYG{n}{version}\PYG{p}{]}
               \PYG{p}{[}\PYG{o}{\PYGZhy{}}\PYG{n}{W} \PYG{n}{WEST\PYGZus{}H5FILE}\PYG{p}{]} \PYG{p}{[}\PYG{o}{\PYGZhy{}}\PYG{n}{o} \PYG{n}{OUTPUT}\PYG{p}{]}
               \PYG{p}{[}\PYG{o}{\PYGZhy{}}\PYG{o}{\PYGZhy{}}\PYG{n}{bins}\PYG{o}{\PYGZhy{}}\PYG{n}{from}\PYG{o}{\PYGZhy{}}\PYG{n}{system} \PYG{o}{|} \PYG{o}{\PYGZhy{}}\PYG{o}{\PYGZhy{}}\PYG{n}{bins}\PYG{o}{\PYGZhy{}}\PYG{n}{from}\PYG{o}{\PYGZhy{}}\PYG{n}{expr} \PYG{n}{BINS\PYGZus{}FROM\PYGZus{}EXPR} \PYG{o}{|} \PYG{o}{\PYGZhy{}}\PYG{o}{\PYGZhy{}}\PYG{n}{bins}\PYG{o}{\PYGZhy{}}\PYG{n}{from}\PYG{o}{\PYGZhy{}}\PYG{n}{function} \PYG{n}{BINS\PYGZus{}FROM\PYGZus{}FUNCTION}\PYG{p}{]}
               \PYG{p}{[}\PYG{o}{\PYGZhy{}}\PYG{n}{p} \PYG{n}{MODULE}\PYG{o}{.}\PYG{n}{FUNCTION}\PYG{p}{]}
               \PYG{p}{[}\PYG{o}{\PYGZhy{}}\PYG{o}{\PYGZhy{}}\PYG{n}{states} \PYG{n}{STATEDEF} \PYG{p}{[}\PYG{n}{STATEDEF} \PYG{o}{.}\PYG{o}{.}\PYG{o}{.}\PYG{p}{]} \PYG{o}{|} \PYG{o}{\PYGZhy{}}\PYG{o}{\PYGZhy{}}\PYG{n}{states}\PYG{o}{\PYGZhy{}}\PYG{n}{from}\PYG{o}{\PYGZhy{}}\PYG{n}{file} \PYG{n}{STATEFILE} \PYG{o}{|} \PYG{o}{\PYGZhy{}}\PYG{o}{\PYGZhy{}}\PYG{n}{states}\PYG{o}{\PYGZhy{}}\PYG{n}{from}\PYG{o}{\PYGZhy{}}\PYG{n}{function} \PYG{n}{STATEFUNC}\PYG{p}{]}
               \PYG{p}{[}\PYG{o}{\PYGZhy{}}\PYG{o}{\PYGZhy{}}\PYG{n}{wm}\PYG{o}{\PYGZhy{}}\PYG{n}{work}\PYG{o}{\PYGZhy{}}\PYG{n}{manager} \PYG{n}{WORK\PYGZus{}MANAGER}\PYG{p}{]} \PYG{p}{[}\PYG{o}{\PYGZhy{}}\PYG{o}{\PYGZhy{}}\PYG{n}{wm}\PYG{o}{\PYGZhy{}}\PYG{n}{n}\PYG{o}{\PYGZhy{}}\PYG{n}{workers} \PYG{n}{N\PYGZus{}WORKERS}\PYG{p}{]}
               \PYG{p}{[}\PYG{o}{\PYGZhy{}}\PYG{o}{\PYGZhy{}}\PYG{n}{wm}\PYG{o}{\PYGZhy{}}\PYG{n}{zmq}\PYG{o}{\PYGZhy{}}\PYG{n}{mode} \PYG{n}{MODE}\PYG{p}{]} \PYG{p}{[}\PYG{o}{\PYGZhy{}}\PYG{o}{\PYGZhy{}}\PYG{n}{wm}\PYG{o}{\PYGZhy{}}\PYG{n}{zmq}\PYG{o}{\PYGZhy{}}\PYG{n}{info} \PYG{n}{INFO\PYGZus{}FILE}\PYG{p}{]}
               \PYG{p}{[}\PYG{o}{\PYGZhy{}}\PYG{o}{\PYGZhy{}}\PYG{n}{wm}\PYG{o}{\PYGZhy{}}\PYG{n}{zmq}\PYG{o}{\PYGZhy{}}\PYG{n}{task}\PYG{o}{\PYGZhy{}}\PYG{n}{endpoint} \PYG{n}{TASK\PYGZus{}ENDPOINT}\PYG{p}{]}
               \PYG{p}{[}\PYG{o}{\PYGZhy{}}\PYG{o}{\PYGZhy{}}\PYG{n}{wm}\PYG{o}{\PYGZhy{}}\PYG{n}{zmq}\PYG{o}{\PYGZhy{}}\PYG{n}{result}\PYG{o}{\PYGZhy{}}\PYG{n}{endpoint} \PYG{n}{RESULT\PYGZus{}ENDPOINT}\PYG{p}{]}
               \PYG{p}{[}\PYG{o}{\PYGZhy{}}\PYG{o}{\PYGZhy{}}\PYG{n}{wm}\PYG{o}{\PYGZhy{}}\PYG{n}{zmq}\PYG{o}{\PYGZhy{}}\PYG{n}{announce}\PYG{o}{\PYGZhy{}}\PYG{n}{endpoint} \PYG{n}{ANNOUNCE\PYGZus{}ENDPOINT}\PYG{p}{]}
               \PYG{p}{[}\PYG{o}{\PYGZhy{}}\PYG{o}{\PYGZhy{}}\PYG{n}{wm}\PYG{o}{\PYGZhy{}}\PYG{n}{zmq}\PYG{o}{\PYGZhy{}}\PYG{n}{listen}\PYG{o}{\PYGZhy{}}\PYG{n}{endpoint} \PYG{n}{ANNOUNCE\PYGZus{}ENDPOINT}\PYG{p}{]}
               \PYG{p}{[}\PYG{o}{\PYGZhy{}}\PYG{o}{\PYGZhy{}}\PYG{n}{wm}\PYG{o}{\PYGZhy{}}\PYG{n}{zmq}\PYG{o}{\PYGZhy{}}\PYG{n}{heartbeat}\PYG{o}{\PYGZhy{}}\PYG{n}{interval} \PYG{n}{INTERVAL}\PYG{p}{]}
               \PYG{p}{[}\PYG{o}{\PYGZhy{}}\PYG{o}{\PYGZhy{}}\PYG{n}{wm}\PYG{o}{\PYGZhy{}}\PYG{n}{zmq}\PYG{o}{\PYGZhy{}}\PYG{n}{task}\PYG{o}{\PYGZhy{}}\PYG{n}{timeout} \PYG{n}{TIMEOUT}\PYG{p}{]}
               \PYG{p}{[}\PYG{o}{\PYGZhy{}}\PYG{o}{\PYGZhy{}}\PYG{n}{wm}\PYG{o}{\PYGZhy{}}\PYG{n}{zmq}\PYG{o}{\PYGZhy{}}\PYG{n}{client}\PYG{o}{\PYGZhy{}}\PYG{n}{comm}\PYG{o}{\PYGZhy{}}\PYG{n}{mode} \PYG{n}{MODE}\PYG{p}{]}
\end{sphinxVerbatim}


\subsubsection{Command\sphinxhyphen{}Line Options}
\label{\detokenize{users_guide/command_line_tools/w_assign:command-line-options}}
See the \sphinxhref{UserGuide:ToolRefs}{general command\sphinxhyphen{}line tool reference} for
more information on the general options.


\subsubsection{Input/output Options}
\label{\detokenize{users_guide/command_line_tools/w_assign:input-output-options}}
\begin{sphinxVerbatim}[commandchars=\\\{\}]
\PYG{o}{\PYGZhy{}}\PYG{n}{W}\PYG{p}{,} \PYG{o}{\PYGZhy{}}\PYG{o}{\PYGZhy{}}\PYG{n}{west}\PYG{o}{\PYGZhy{}}\PYG{n}{data} \PYG{o}{/}\PYG{n}{path}\PYG{o}{/}\PYG{n}{to}\PYG{o}{/}\PYG{n}{file}

    \PYG{n}{Read} \PYG{n}{simulation} \PYG{n}{result} \PYG{n}{data} \PYG{k+kn}{from} \PYG{n+nn}{file} \PYG{o}{*}\PYG{n}{file}\PYG{o}{*}\PYG{o}{.} \PYG{p}{(}\PYG{o}{*}\PYG{o}{*}\PYG{n}{Default}\PYG{p}{:}\PYG{o}{*}\PYG{o}{*} \PYG{n}{The}
    \PYG{o}{*}\PYG{n}{hdf5}\PYG{o}{*} \PYG{n}{file} \PYG{n}{specified} \PYG{o+ow}{in} \PYG{n}{the} \PYG{n}{configuration} \PYG{n}{file}\PYG{p}{,} \PYG{n}{by} \PYG{n}{default}
    \PYG{o}{*}\PYG{o}{*}\PYG{n}{west}\PYG{o}{.}\PYG{n}{h5}\PYG{o}{*}\PYG{o}{*}\PYG{p}{)}

\PYG{o}{\PYGZhy{}}\PYG{n}{o}\PYG{p}{,} \PYG{o}{\PYGZhy{}}\PYG{o}{\PYGZhy{}}\PYG{n}{output} \PYG{o}{/}\PYG{n}{path}\PYG{o}{/}\PYG{n}{to}\PYG{o}{/}\PYG{n}{file}
    \PYG{n}{Write} \PYG{n}{assignment} \PYG{n}{results} \PYG{n}{to} \PYG{n}{file} \PYG{o}{*}\PYG{n}{outfile}\PYG{o}{*}\PYG{o}{.} \PYG{p}{(}\PYG{o}{*}\PYG{o}{*}\PYG{n}{Default}\PYG{p}{:}\PYG{o}{*}\PYG{o}{*} \PYG{o}{*}\PYG{n}{hdf5}\PYG{o}{*}
    \PYG{n}{file} \PYG{o}{*}\PYG{o}{*}\PYG{n}{assign}\PYG{o}{.}\PYG{n}{h5}\PYG{o}{*}\PYG{o}{*}\PYG{p}{)}
\end{sphinxVerbatim}


\subsubsection{Binning Options}
\label{\detokenize{users_guide/command_line_tools/w_assign:binning-options}}
Specify how binning is to be assigned to the dataset.:

\begin{sphinxVerbatim}[commandchars=\\\{\}]
\PYGZhy{}\PYGZhy{}bins\PYGZhy{}from\PYGZhy{}system
  Use binning scheme specified by the system driver; system driver can be
  found in the west configuration file, by default named **west.cfg**
  (**Default binning**)

\PYGZhy{}\PYGZhy{}bins\PYGZhy{}from\PYGZhy{}expr bin\PYGZus{}expr
  Use binning scheme specified in *``bin\PYGZus{}expr``*, which takes the form a
  Python list of lists, where each inner list corresponds to the binning a
  given dimension. (for example, \PYGZdq{}[[0,1,2,4,inf],[\PYGZhy{}inf,0,inf]]\PYGZdq{} specifies bin
  boundaries for two dimensional progress coordinate. Note that this option
  accepts the special symbol \PYGZsq{}inf\PYGZsq{} for floating point infinity

\PYGZhy{}\PYGZhy{}bins\PYGZhy{}from\PYGZhy{}function bin\PYGZus{}func
  Bins specified by calling an external function *``bin\PYGZus{}func``*.
  *``bin\PYGZus{}func``* should be formatted as \PYGZsq{}[PATH:]module.function\PYGZsq{}, where the
  function \PYGZsq{}function\PYGZsq{} in module \PYGZsq{}module\PYGZsq{} will be used
\end{sphinxVerbatim}


\subsubsection{Macrostate Options}
\label{\detokenize{users_guide/command_line_tools/w_assign:macrostate-options}}
You can optionally specify how to assign user\sphinxhyphen{}defined macrostates. Note
that macrostates must be assigned for subsequent analysis tools, namely
\sphinxcode{\sphinxupquote{w\_kinetics}} and \sphinxcode{\sphinxupquote{w\_kinavg}}.:

\begin{sphinxVerbatim}[commandchars=\\\{\}]
\PYGZhy{}\PYGZhy{}states statedef [statedef ...]
  Specify a macrostate for a single bin as *``statedef``*, formatted
  as a coordinate tuple where each coordinate specifies the bin to
  which it belongs, for instance:
  \PYGZsq{}[1.0, 2.0]\PYGZsq{} assigns a macrostate corresponding to the bin that
  contains the (two\PYGZhy{}dimensional) progress coordinates 1.0 and 2.0.
  Note that a macrostate label can optionally by specified, for
  instance: \PYGZsq{}bound:[1.0, 2.0]\PYGZsq{} assigns the corresponding bin
  containing the given coordinates the macrostate named \PYGZsq{}bound\PYGZsq{}. Note
  that multiple assignments can be specified with this command, but
  only one macrostate per bin is possible \PYGZhy{} if you wish to specify
  multiple bins in a single macrostate, use the
  *``\PYGZhy{}\PYGZhy{}states\PYGZhy{}from\PYGZhy{}file``* option.

\PYGZhy{}\PYGZhy{}states\PYGZhy{}from\PYGZhy{}file statefile
  Read macrostate assignments from *yaml* file *``statefile``*. This
  option allows you to assign multiple bins to a single macrostate.
  The following example shows the contents of *``statefile``* that
  specify two macrostates, bound and unbound, over multiple bins with
  a two\PYGZhy{}dimensional progress coordinate:

\PYGZhy{}\PYGZhy{}\PYGZhy{}
states:
  \PYGZhy{} label: unbound
    coords:
      \PYGZhy{} [9.0, 1.0]
      \PYGZhy{} [9.0, 2.0]
  \PYGZhy{} label: bound
    coords:
      \PYGZhy{} [0.1, 0.0]
\end{sphinxVerbatim}


\subsubsection{Specifying Progress Coordinate}
\label{\detokenize{users_guide/command_line_tools/w_assign:specifying-progress-coordinate}}
By default, progress coordinate information for each iteration is taken from
\sphinxstyleemphasis{pcoord} dataset in the specified input file (which, by default is \sphinxstyleemphasis{west.h5}).
Optionally, you can specify a function to construct the progress coordinate for
each iteration \sphinxhyphen{} this may be useful to consolidate data from several sources or
otherwise preprocess the progress coordinate data.:

\begin{sphinxVerbatim}[commandchars=\\\{\}]
\PYG{o}{\PYGZhy{}}\PYG{o}{\PYGZhy{}}\PYG{n}{construct}\PYG{o}{\PYGZhy{}}\PYG{n}{pcoord} \PYG{n}{module}\PYG{o}{.}\PYG{n}{function}\PYG{p}{,} \PYG{o}{\PYGZhy{}}\PYG{n}{p} \PYG{n}{module}\PYG{o}{.}\PYG{n}{function}
  \PYG{n}{Use} \PYG{n}{the} \PYG{n}{function} \PYG{o}{*}\PYG{n}{module}\PYG{o}{.}\PYG{n}{function}\PYG{o}{*} \PYG{n}{to} \PYG{n}{construct} \PYG{n}{the} \PYG{n}{progress}
  \PYG{n}{coordinate} \PYG{k}{for} \PYG{n}{each} \PYG{n}{iteration}\PYG{o}{.} \PYG{n}{This} \PYG{n}{will} \PYG{n}{be} \PYG{n}{called} \PYG{n}{once} \PYG{n}{per}
  \PYG{n}{iteration} \PYG{k}{as} \PYG{o}{*}\PYG{n}{function}\PYG{p}{(}\PYG{n}{n\PYGZus{}iter}\PYG{p}{,} \PYG{n}{iter\PYGZus{}group}\PYG{p}{)}\PYG{o}{*} \PYG{o+ow}{and} \PYG{n}{should} \PYG{k}{return} \PYG{n}{an}
  \PYG{n}{array} \PYG{n}{indexable} \PYG{k}{as} \PYG{p}{[}\PYG{n}{seg\PYGZus{}id}\PYG{p}{]}\PYG{p}{[}\PYG{n}{timepoint}\PYG{p}{]}\PYG{p}{[}\PYG{n}{dimension}\PYG{p}{]}\PYG{o}{.} \PYG{n}{The}
  \PYG{o}{*}\PYG{o}{*}\PYG{n}{default}\PYG{o}{*}\PYG{o}{*} \PYG{n}{function} \PYG{n}{returns} \PYG{n}{the} \PYG{l+s+s1}{\PYGZsq{}}\PYG{l+s+s1}{pcoord}\PYG{l+s+s1}{\PYGZsq{}} \PYG{n}{dataset} \PYG{k}{for} \PYG{n}{that} \PYG{n}{iteration}
  \PYG{p}{(}\PYG{n}{i}\PYG{o}{.}\PYG{n}{e}\PYG{o}{.} \PYG{n}{the} \PYG{n}{function} \PYG{n}{executes} \PYG{k}{return} \PYG{n}{iter\PYGZus{}group}\PYG{p}{[}\PYG{l+s+s1}{\PYGZsq{}}\PYG{l+s+s1}{pcoord}\PYG{l+s+s1}{\PYGZsq{}}\PYG{p}{]}\PYG{p}{[}\PYG{o}{.}\PYG{o}{.}\PYG{o}{.}\PYG{p}{]}\PYG{p}{)}
\end{sphinxVerbatim}


\subsubsection{Examples}
\label{\detokenize{users_guide/command_line_tools/w_assign:examples}}

\subsection{w\_bins}
\label{\detokenize{users_guide/command_line_tools/w_bins:w-bins}}\label{\detokenize{users_guide/command_line_tools/w_bins:id1}}\label{\detokenize{users_guide/command_line_tools/w_bins::doc}}
\sphinxcode{\sphinxupquote{w\_bins}} deals with binning modification and statistics


\subsubsection{Overview}
\label{\detokenize{users_guide/command_line_tools/w_bins:overview}}
Usage:

\begin{sphinxVerbatim}[commandchars=\\\{\}]
\PYGZdl{}WEST\PYGZus{}ROOT/bin/w\PYGZus{}bins [\PYGZhy{}h] [\PYGZhy{}r RCFILE] [\PYGZhy{}\PYGZhy{}quiet | \PYGZhy{}\PYGZhy{}verbose | \PYGZhy{}\PYGZhy{}debug] [\PYGZhy{}\PYGZhy{}version]
             [\PYGZhy{}W WEST\PYGZus{}H5FILE]
             \PYGZob{}info,rebin\PYGZcb{} ...
\end{sphinxVerbatim}

Display information and statistics about binning in a WEST simulation, or
modify the binning for the current iteration of a WEST simulation.


\subsubsection{Command\sphinxhyphen{}Line Options}
\label{\detokenize{users_guide/command_line_tools/w_bins:command-line-options}}
See the \sphinxhref{UserGuide:ToolRefs}{general command\sphinxhyphen{}line tool reference} for
more information on the general options.


\paragraph{Options Under ‘info’}
\label{\detokenize{users_guide/command_line_tools/w_bins:options-under-info}}
Usage:

\begin{sphinxVerbatim}[commandchars=\\\{\}]
\PYGZdl{}WEST\PYGZus{}ROOT/bin/w\PYGZus{}bins info [\PYGZhy{}h] [\PYGZhy{}n N\PYGZus{}ITER] [\PYGZhy{}\PYGZhy{}detail]
                  [\PYGZhy{}\PYGZhy{}bins\PYGZhy{}from\PYGZhy{}system | \PYGZhy{}\PYGZhy{}bins\PYGZhy{}from\PYGZhy{}expr BINS\PYGZus{}FROM\PYGZus{}EXPR | \PYGZhy{}\PYGZhy{}bins\PYGZhy{}from\PYGZhy{}function BINS\PYGZus{}FROM\PYGZus{}FUNCTION | \PYGZhy{}\PYGZhy{}bins\PYGZhy{}from\PYGZhy{}file]
\end{sphinxVerbatim}

Positional options:

\begin{sphinxVerbatim}[commandchars=\\\{\}]
\PYG{n}{info}
  \PYG{n}{Display} \PYG{n}{information} \PYG{n}{about} \PYG{n}{binning}\PYG{o}{.}
\end{sphinxVerbatim}

Options for ‘info’:

\begin{sphinxVerbatim}[commandchars=\\\{\}]
\PYG{o}{\PYGZhy{}}\PYG{n}{n} \PYG{n}{N\PYGZus{}ITER}\PYG{p}{,} \PYG{o}{\PYGZhy{}}\PYG{o}{\PYGZhy{}}\PYG{n}{n}\PYG{o}{\PYGZhy{}}\PYG{n+nb}{iter} \PYG{n}{N\PYGZus{}ITER}
  \PYG{n}{Consider} \PYG{n}{initial} \PYG{n}{points} \PYG{n}{of} \PYG{n}{segment} \PYG{n}{N\PYGZus{}ITER} \PYG{p}{(}\PYG{n}{default}\PYG{p}{:} \PYG{n}{current}
  \PYG{n}{iteration}\PYG{p}{)}\PYG{o}{.}

\PYG{o}{\PYGZhy{}}\PYG{o}{\PYGZhy{}}\PYG{n}{detail}
  \PYG{n}{Display} \PYG{n}{detailed} \PYG{n}{per}\PYG{o}{\PYGZhy{}}\PYG{n+nb}{bin} \PYG{n}{information} \PYG{o+ow}{in} \PYG{n}{addition} \PYG{n}{to} \PYG{n}{summary}
  \PYG{n}{information}\PYG{o}{.}
\end{sphinxVerbatim}

Binning options for ‘info’:

\begin{sphinxVerbatim}[commandchars=\\\{\}]
\PYG{o}{\PYGZhy{}}\PYG{o}{\PYGZhy{}}\PYG{n}{bins}\PYG{o}{\PYGZhy{}}\PYG{n}{from}\PYG{o}{\PYGZhy{}}\PYG{n}{system}
  \PYG{n}{Bins} \PYG{n}{are} \PYG{n}{constructed} \PYG{n}{by} \PYG{n}{the} \PYG{n}{system} \PYG{n}{driver} \PYG{n}{specified} \PYG{o+ow}{in} \PYG{n}{the} \PYG{n}{WEST}
  \PYG{n}{configuration} \PYG{n}{file} \PYG{p}{(}\PYG{n}{default} \PYG{n}{where} \PYG{n}{stored} \PYG{n+nb}{bin} \PYG{n}{definitions} \PYG{o+ow}{not}
  \PYG{n}{available}\PYG{p}{)}\PYG{o}{.}

\PYG{o}{\PYGZhy{}}\PYG{o}{\PYGZhy{}}\PYG{n}{bins}\PYG{o}{\PYGZhy{}}\PYG{n}{from}\PYG{o}{\PYGZhy{}}\PYG{n}{expr} \PYG{n}{BINS\PYGZus{}FROM\PYGZus{}EXPR}\PYG{p}{,} \PYG{o}{\PYGZhy{}}\PYG{o}{\PYGZhy{}}\PYG{n}{binbounds} \PYG{n}{BINS\PYGZus{}FROM\PYGZus{}EXPR}
  \PYG{n}{Construct} \PYG{n}{bins} \PYG{n}{on} \PYG{n}{a} \PYG{n}{rectilinear} \PYG{n}{grid} \PYG{n}{according} \PYG{n}{to} \PYG{n}{the} \PYG{n}{given} \PYG{n}{BINEXPR}\PYG{o}{.}
  \PYG{n}{This} \PYG{n}{must} \PYG{n}{be} \PYG{n}{a} \PYG{n+nb}{list} \PYG{n}{of} \PYG{n}{lists} \PYG{n}{of} \PYG{n+nb}{bin} \PYG{n}{boundaries} \PYG{p}{(}\PYG{n}{one} \PYG{n+nb}{list} \PYG{n}{of} \PYG{n+nb}{bin}
  \PYG{n}{boundaries} \PYG{k}{for} \PYG{n}{each} \PYG{n}{dimension} \PYG{n}{of} \PYG{n}{the} \PYG{n}{progress} \PYG{n}{coordinate}\PYG{p}{)}\PYG{p}{,} \PYG{n}{formatted}
  \PYG{k}{as} \PYG{n}{a} \PYG{n}{Python} \PYG{n}{expression}\PYG{o}{.} \PYG{n}{E}\PYG{o}{.}\PYG{n}{g}\PYG{o}{.} \PYG{l+s+s2}{\PYGZdq{}}\PYG{l+s+s2}{[[0,1,2,4,inf],[\PYGZhy{}inf,0,inf]]}\PYG{l+s+s2}{\PYGZdq{}}\PYG{o}{.} \PYG{n}{The}
  \PYG{n}{numpy} \PYG{n}{module} \PYG{o+ow}{and} \PYG{n}{the} \PYG{n}{special} \PYG{n}{symbol} \PYG{l+s+s2}{\PYGZdq{}}\PYG{l+s+s2}{inf}\PYG{l+s+s2}{\PYGZdq{}} \PYG{p}{(}\PYG{k}{for} \PYG{n}{floating}\PYG{o}{\PYGZhy{}}\PYG{n}{point}
  \PYG{n}{infinity}\PYG{p}{)} \PYG{n}{are} \PYG{n}{available} \PYG{k}{for} \PYG{n}{use} \PYG{n}{within} \PYG{n}{BINEXPR}\PYG{o}{.}

\PYG{o}{\PYGZhy{}}\PYG{o}{\PYGZhy{}}\PYG{n}{bins}\PYG{o}{\PYGZhy{}}\PYG{n}{from}\PYG{o}{\PYGZhy{}}\PYG{n}{function} \PYG{n}{BINS\PYGZus{}FROM\PYGZus{}FUNCTION}\PYG{p}{,} \PYG{o}{\PYGZhy{}}\PYG{o}{\PYGZhy{}}\PYG{n}{binfunc} \PYG{n}{BINS\PYGZus{}FROM\PYGZus{}FUNCTION}
  \PYG{n}{Supply} \PYG{n}{an} \PYG{n}{external} \PYG{n}{function} \PYG{n}{which}\PYG{p}{,} \PYG{n}{when} \PYG{n}{called}\PYG{p}{,} \PYG{n}{returns} \PYG{n}{a} \PYG{n}{properly}
  \PYG{n}{constructed} \PYG{n+nb}{bin} \PYG{n}{mapper} \PYG{n}{which} \PYG{n}{will} \PYG{n}{then} \PYG{n}{be} \PYG{n}{used} \PYG{k}{for} \PYG{n+nb}{bin} \PYG{n}{assignments}\PYG{o}{.}
  \PYG{n}{This} \PYG{n}{should} \PYG{n}{be} \PYG{n}{formatted} \PYG{k}{as} \PYG{l+s+s2}{\PYGZdq{}}\PYG{l+s+s2}{[PATH:]MODULE.FUNC}\PYG{l+s+s2}{\PYGZdq{}}\PYG{p}{,} \PYG{n}{where} \PYG{n}{the} \PYG{n}{function}
  \PYG{n}{FUNC} \PYG{o+ow}{in} \PYG{n}{module} \PYG{n}{MODULE} \PYG{n}{will} \PYG{n}{be} \PYG{n}{used}\PYG{p}{;} \PYG{n}{the} \PYG{n}{optional} \PYG{n}{PATH} \PYG{n}{will} \PYG{n}{be}
  \PYG{n}{prepended} \PYG{n}{to} \PYG{n}{the} \PYG{n}{module} \PYG{n}{search} \PYG{n}{path} \PYG{n}{when} \PYG{n}{loading} \PYG{n}{MODULE}\PYG{o}{.}

\PYG{o}{\PYGZhy{}}\PYG{o}{\PYGZhy{}}\PYG{n}{bins}\PYG{o}{\PYGZhy{}}\PYG{n}{from}\PYG{o}{\PYGZhy{}}\PYG{n}{file}
  \PYG{n}{Load} \PYG{n+nb}{bin} \PYG{n}{specification} \PYG{k+kn}{from} \PYG{n+nn}{the} \PYG{n}{data} \PYG{n}{file} \PYG{n}{being} \PYG{n}{examined} \PYG{p}{(}\PYG{n}{default}
  \PYG{n}{where} \PYG{n}{stored} \PYG{n+nb}{bin} \PYG{n}{definitions} \PYG{n}{available}\PYG{p}{)}\PYG{o}{.}
\end{sphinxVerbatim}


\paragraph{Options Under ‘rebin’}
\label{\detokenize{users_guide/command_line_tools/w_bins:options-under-rebin}}
Usage:

\begin{sphinxVerbatim}[commandchars=\\\{\}]
\PYGZdl{}WEST\PYGZus{}ROOT/bin/w\PYGZus{}bins rebin [\PYGZhy{}h] [\PYGZhy{}\PYGZhy{}confirm] [\PYGZhy{}\PYGZhy{}detail]
                   [\PYGZhy{}\PYGZhy{}bins\PYGZhy{}from\PYGZhy{}system | \PYGZhy{}\PYGZhy{}bins\PYGZhy{}from\PYGZhy{}expr BINS\PYGZus{}FROM\PYGZus{}EXPR | \PYGZhy{}\PYGZhy{}bins\PYGZhy{}from\PYGZhy{}function BINS\PYGZus{}FROM\PYGZus{}FUNCTION]
                   [\PYGZhy{}\PYGZhy{}target\PYGZhy{}counts TARGET\PYGZus{}COUNTS | \PYGZhy{}\PYGZhy{}target\PYGZhy{}counts\PYGZhy{}from FILENAME]
\end{sphinxVerbatim}

Positional option:

\begin{sphinxVerbatim}[commandchars=\\\{\}]
\PYG{n}{rebin}
  \PYG{n}{Rebuild} \PYG{n}{current} \PYG{n}{iteration} \PYG{k}{with} \PYG{n}{new} \PYG{n}{binning}\PYG{o}{.}
\end{sphinxVerbatim}

Options for ‘rebin’:

\begin{sphinxVerbatim}[commandchars=\\\{\}]
\PYG{o}{\PYGZhy{}}\PYG{o}{\PYGZhy{}}\PYG{n}{confirm}
  \PYG{n}{Commit} \PYG{n}{the} \PYG{n}{revised} \PYG{n}{iteration} \PYG{n}{to} \PYG{n}{HDF5}\PYG{p}{;} \PYG{n}{without} \PYG{n}{this} \PYG{n}{option}\PYG{p}{,} \PYG{n}{the}
  \PYG{n}{effects} \PYG{n}{of} \PYG{n}{the} \PYG{n}{new} \PYG{n}{binning} \PYG{n}{are} \PYG{n}{only} \PYG{n}{calculated} \PYG{o+ow}{and} \PYG{n}{printed}\PYG{o}{.}

\PYG{o}{\PYGZhy{}}\PYG{o}{\PYGZhy{}}\PYG{n}{detail}
  \PYG{n}{Display} \PYG{n}{detailed} \PYG{n}{per}\PYG{o}{\PYGZhy{}}\PYG{n+nb}{bin} \PYG{n}{information} \PYG{o+ow}{in} \PYG{n}{addition} \PYG{n}{to} \PYG{n}{summary}
  \PYG{n}{information}\PYG{o}{.}
\end{sphinxVerbatim}

Binning options for ‘rebin’;

Same as the binning options for ‘info’.

Bin target count options for ‘rebin’;:

\begin{sphinxVerbatim}[commandchars=\\\{\}]
\PYG{o}{\PYGZhy{}}\PYG{o}{\PYGZhy{}}\PYG{n}{target}\PYG{o}{\PYGZhy{}}\PYG{n}{counts} \PYG{n}{TARGET\PYGZus{}COUNTS}
  \PYG{n}{Use} \PYG{n}{TARGET\PYGZus{}COUNTS} \PYG{n}{instead} \PYG{n}{of} \PYG{n}{stored} \PYG{o+ow}{or} \PYG{n}{system} \PYG{n}{driver} \PYG{n}{target} \PYG{n}{counts}\PYG{o}{.}
  \PYG{n}{TARGET\PYGZus{}COUNTS} \PYG{o+ow}{is} \PYG{n}{a} \PYG{n}{comma}\PYG{o}{\PYGZhy{}}\PYG{n}{separated} \PYG{n+nb}{list} \PYG{n}{of} \PYG{n}{integers}\PYG{o}{.} \PYG{n}{As} \PYG{n}{a} \PYG{n}{special}
  \PYG{n}{case}\PYG{p}{,} \PYG{n}{a} \PYG{n}{single} \PYG{n}{integer} \PYG{o+ow}{is} \PYG{n}{acceptable}\PYG{p}{,} \PYG{o+ow}{in} \PYG{n}{which} \PYG{n}{case} \PYG{n}{the} \PYG{n}{same} \PYG{n}{target}
  \PYG{n}{count} \PYG{o+ow}{is} \PYG{n}{used} \PYG{k}{for} \PYG{n+nb}{all} \PYG{n}{bins}\PYG{o}{.}

\PYG{o}{\PYGZhy{}}\PYG{o}{\PYGZhy{}}\PYG{n}{target}\PYG{o}{\PYGZhy{}}\PYG{n}{counts}\PYG{o}{\PYGZhy{}}\PYG{k+kn}{from} \PYG{n+nn}{FILENAME}
  \PYG{n}{Read} \PYG{n}{target} \PYG{n}{counts} \PYG{k+kn}{from} \PYG{n+nn}{the} \PYG{n}{text} \PYG{n}{file} \PYG{n}{FILENAME} \PYG{n}{instead} \PYG{n}{of} \PYG{n}{using}
  \PYG{n}{stored} \PYG{o+ow}{or} \PYG{n}{system} \PYG{n}{driver} \PYG{n}{target} \PYG{n}{counts}\PYG{o}{.} \PYG{n}{FILENAME} \PYG{n}{must} \PYG{n}{contain} \PYG{n}{a} \PYG{n+nb}{list}
  \PYG{n}{of} \PYG{n}{integers}\PYG{p}{,} \PYG{n}{separated} \PYG{n}{by} \PYG{n}{arbitrary} \PYG{n}{whitespace} \PYG{p}{(}\PYG{n}{including} \PYG{n}{newlines}\PYG{p}{)}\PYG{o}{.}
\end{sphinxVerbatim}


\subsubsection{Input Options}
\label{\detokenize{users_guide/command_line_tools/w_bins:input-options}}
\begin{sphinxVerbatim}[commandchars=\\\{\}]
\PYG{o}{\PYGZhy{}}\PYG{n}{W} \PYG{n}{WEST\PYGZus{}H5FILE}\PYG{p}{,} \PYG{o}{\PYGZhy{}}\PYG{o}{\PYGZhy{}}\PYG{n}{west\PYGZus{}data} \PYG{n}{WEST\PYGZus{}H5FILE}
  \PYG{n}{Take} \PYG{n}{WEST} \PYG{n}{data} \PYG{k+kn}{from} \PYG{n+nn}{WEST\PYGZus{}H5FILE} \PYG{p}{(}\PYG{n}{default}\PYG{p}{:} \PYG{n}{read} \PYG{k+kn}{from} \PYG{n+nn}{the} \PYG{n}{HDF5} \PYG{n}{file}
  \PYG{n}{specified} \PYG{o+ow}{in} \PYG{n}{west}\PYG{o}{.}\PYG{n}{cfg}\PYG{p}{)}\PYG{o}{.}
\end{sphinxVerbatim}


\subsubsection{Examples}
\label{\detokenize{users_guide/command_line_tools/w_bins:examples}}
(TODO: Write up an example)


\subsection{w\_crawl}
\label{\detokenize{users_guide/command_line_tools/w_crawl:w-crawl}}\label{\detokenize{users_guide/command_line_tools/w_crawl:id1}}\label{\detokenize{users_guide/command_line_tools/w_crawl::doc}}

\subsection{w\_eddist}
\label{\detokenize{users_guide/command_line_tools/w_eddist:w-eddist}}\label{\detokenize{users_guide/command_line_tools/w_eddist:id1}}\label{\detokenize{users_guide/command_line_tools/w_eddist::doc}}

\subsection{w\_fluxanl}
\label{\detokenize{users_guide/command_line_tools/w_fluxanl:w-fluxanl}}\label{\detokenize{users_guide/command_line_tools/w_fluxanl:id1}}\label{\detokenize{users_guide/command_line_tools/w_fluxanl::doc}}
\sphinxcode{\sphinxupquote{w\_fluxanl}} calculates the probability flux of a weighted ensemble simulation
based on a pre\sphinxhyphen{}defined target state. Also calculates confidence interval of
average flux. Monte Carlo bootstrapping techniques are used to account for
autocorrelation between fluxes and/or errors that are not normally distributed.


\subsubsection{Overview}
\label{\detokenize{users_guide/command_line_tools/w_fluxanl:overview}}
usage:

\begin{sphinxVerbatim}[commandchars=\\\{\}]
\PYGZdl{}WEST\PYGZus{}ROOT/bin/w\PYGZus{}fluxanl [\PYGZhy{}h] [\PYGZhy{}r RCFILE] [\PYGZhy{}\PYGZhy{}quiet | \PYGZhy{}\PYGZhy{}verbose | \PYGZhy{}\PYGZhy{}debug] [\PYGZhy{}\PYGZhy{}version]
                         [\PYGZhy{}W WEST\PYGZus{}H5FILE] [\PYGZhy{}o OUTPUT]
                         [\PYGZhy{}\PYGZhy{}first\PYGZhy{}iter N\PYGZus{}ITER] [\PYGZhy{}\PYGZhy{}last\PYGZhy{}iter N\PYGZus{}ITER]
                         [\PYGZhy{}a ALPHA] [\PYGZhy{}\PYGZhy{}autocorrel\PYGZhy{}alpha ACALPHA] [\PYGZhy{}N NSETS] [\PYGZhy{}\PYGZhy{}evol] [\PYGZhy{}\PYGZhy{}evol\PYGZhy{}step ESTEP]
\end{sphinxVerbatim}

Note: All command line arguments are optional for \sphinxcode{\sphinxupquote{w\_fluxanl}}.


\subsubsection{Command\sphinxhyphen{}Line Options}
\label{\detokenize{users_guide/command_line_tools/w_fluxanl:command-line-options}}
See the \sphinxhref{UserGuide:ToolRefs}{general command\sphinxhyphen{}line tool reference} for more
information on the general options.


\paragraph{Input/output options}
\label{\detokenize{users_guide/command_line_tools/w_fluxanl:input-output-options}}
These arguments allow the user to specify where to read input simulation result
data and where to output calculated progress coordinate probability
distribution data.

Both input and output files are \sphinxstyleemphasis{hdf5} format.:

\begin{sphinxVerbatim}[commandchars=\\\{\}]
\PYG{o}{\PYGZhy{}}\PYG{n}{W}\PYG{p}{,} \PYG{o}{\PYGZhy{}}\PYG{o}{\PYGZhy{}}\PYG{n}{west}\PYG{o}{\PYGZhy{}}\PYG{n}{data} \PYG{n}{file}
  \PYG{n}{Read} \PYG{n}{simulation} \PYG{n}{result} \PYG{n}{data} \PYG{k+kn}{from} \PYG{n+nn}{file} \PYG{o}{*}\PYG{n}{file}\PYG{o}{*}\PYG{o}{.} \PYG{p}{(}\PYG{o}{*}\PYG{o}{*}\PYG{n}{Default}\PYG{p}{:}\PYG{o}{*}\PYG{o}{*} \PYG{n}{The}
  \PYG{o}{*}\PYG{n}{hdf5}\PYG{o}{*} \PYG{n}{file} \PYG{n}{specified} \PYG{o+ow}{in} \PYG{n}{the} \PYG{n}{configuration} \PYG{n}{file}\PYG{p}{)}

\PYG{o}{\PYGZhy{}}\PYG{n}{o}\PYG{p}{,} \PYG{o}{\PYGZhy{}}\PYG{o}{\PYGZhy{}}\PYG{n}{output} \PYG{n}{file}
  \PYG{n}{Store} \PYG{n}{this} \PYG{n}{tool}\PYG{l+s+s1}{\PYGZsq{}}\PYG{l+s+s1}{s output in *file*. (**Default:** The *hdf5* file}
  \PYG{o}{*}\PYG{o}{*}\PYG{n}{pcpdist}\PYG{o}{.}\PYG{n}{h5}\PYG{o}{*}\PYG{o}{*}\PYG{p}{)}
\end{sphinxVerbatim}


\paragraph{Iteration range options}
\label{\detokenize{users_guide/command_line_tools/w_fluxanl:iteration-range-options}}
Specify the range of iterations over which to construct the progress
coordinate probability distribution.:

\begin{sphinxVerbatim}[commandchars=\\\{\}]
\PYG{o}{\PYGZhy{}}\PYG{o}{\PYGZhy{}}\PYG{n}{first}\PYG{o}{\PYGZhy{}}\PYG{n+nb}{iter} \PYG{n}{n\PYGZus{}iter}
  \PYG{n}{Construct} \PYG{n}{probability} \PYG{n}{distribution} \PYG{n}{starting} \PYG{k}{with} \PYG{n}{iteration} \PYG{o}{*}\PYG{n}{n\PYGZus{}iter}\PYG{o}{*}
  \PYG{p}{(}\PYG{o}{*}\PYG{o}{*}\PYG{n}{Default}\PYG{p}{:}\PYG{o}{*}\PYG{o}{*} \PYG{l+m+mi}{1}\PYG{p}{)}

\PYG{o}{\PYGZhy{}}\PYG{o}{\PYGZhy{}}\PYG{n}{last}\PYG{o}{\PYGZhy{}}\PYG{n+nb}{iter} \PYG{n}{n\PYGZus{}iter}
  \PYG{n}{Construct} \PYG{n}{probability} \PYG{n}{distribution}\PYG{l+s+s1}{\PYGZsq{}}\PYG{l+s+s1}{s time evolution up to (and}
  \PYG{n}{including}\PYG{p}{)} \PYG{n}{iteration} \PYG{o}{*}\PYG{n}{n\PYGZus{}iter}\PYG{o}{*} \PYG{p}{(}\PYG{o}{*}\PYG{o}{*}\PYG{n}{Default}\PYG{p}{:}\PYG{o}{*}\PYG{o}{*} \PYG{n}{Last} \PYG{n}{completed}
  \PYG{n}{iteration}\PYG{p}{)}
\end{sphinxVerbatim}


\paragraph{Confidence interval and bootstrapping options}
\label{\detokenize{users_guide/command_line_tools/w_fluxanl:confidence-interval-and-bootstrapping-options}}
Specify alpha values of constructed confidence intervals.:

\begin{sphinxVerbatim}[commandchars=\\\{\}]
\PYGZhy{}a alpha
  Calculate a (1 \PYGZhy{} *alpha*) confidence interval for the mean flux
  (**Default:** 0.05)

\PYGZhy{}\PYGZhy{}autocorrel\PYGZhy{}alpha ACalpha
  Identify autocorrelation of fluxes at *ACalpha* significance level.
  Note: Specifying an *ACalpha* level that is too small may result in
  failure to find autocorrelation in noisy flux signals (**Default:**
  Same level as *alpha*)

\PYGZhy{}N n\PYGZus{}sets, \PYGZhy{}\PYGZhy{}nsets n\PYGZus{}sets
  Use *n\PYGZus{}sets* samples for bootstrapping (**Default:** Chosen based
  on *alpha*)

\PYGZhy{}\PYGZhy{}evol
  Calculate the time evolution of flux confidence intervals
  (**Warning:** computationally expensive calculation)

\PYGZhy{}\PYGZhy{}evol\PYGZhy{}step estep
  (if ``\PYGZsq{}\PYGZhy{}\PYGZhy{}evol\PYGZsq{}`` specified) Calculate the time evolution of flux
  confidence intervals for every *estep* iterations (**Default:** 1)
\end{sphinxVerbatim}


\subsubsection{Examples}
\label{\detokenize{users_guide/command_line_tools/w_fluxanl:examples}}
Calculate the time evolution flux every 5 iterations:

\begin{sphinxVerbatim}[commandchars=\\\{\}]
\PYGZdl{}WEST\PYGZus{}ROOT/bin/w\PYGZus{}fluxanl \PYGZhy{}\PYGZhy{}evol \PYGZhy{}\PYGZhy{}evol\PYGZhy{}step 5
\end{sphinxVerbatim}

Calculate mean flux confidence intervals at 0.01 signicance level and
calculate autocorrelations at 0.05 significance:

\begin{sphinxVerbatim}[commandchars=\\\{\}]
\PYGZdl{}WEST\PYGZus{}ROOT/bin/w\PYGZus{}fluxanl \PYGZhy{}\PYGZhy{}alpha 0.01 \PYGZhy{}\PYGZhy{}autocorrel\PYGZhy{}alpha 0.05
\end{sphinxVerbatim}

Calculate the mean flux confidence intervals using a custom bootstrap
sample size of 500:

\begin{sphinxVerbatim}[commandchars=\\\{\}]
\PYGZdl{}WEST\PYGZus{}ROOT/bin/w\PYGZus{}fluxanl \PYGZhy{}\PYGZhy{}n\PYGZhy{}sets 500
\end{sphinxVerbatim}


\subsection{w\_fork}
\label{\detokenize{users_guide/command_line_tools/w_fork:w-fork}}\label{\detokenize{users_guide/command_line_tools/w_fork:id1}}\label{\detokenize{users_guide/command_line_tools/w_fork::doc}}

\subsection{w\_init}
\label{\detokenize{users_guide/command_line_tools/w_init:w-init}}\label{\detokenize{users_guide/command_line_tools/w_init:id1}}\label{\detokenize{users_guide/command_line_tools/w_init::doc}}
\sphinxcode{\sphinxupquote{w\_init}} initializes the weighted ensemble simulation, creates the
main HDF5 file and prepares the first iteration.


\subsubsection{Overview}
\label{\detokenize{users_guide/command_line_tools/w_init:overview}}
Usage:

\begin{sphinxVerbatim}[commandchars=\\\{\}]
\PYGZdl{}WEST\PYGZus{}ROOT/bin/w\PYGZus{}init [\PYGZhy{}h] [\PYGZhy{}r RCFILE] [\PYGZhy{}\PYGZhy{}quiet | \PYGZhy{}\PYGZhy{}verbose | \PYGZhy{}\PYGZhy{}debug] [\PYGZhy{}\PYGZhy{}version]
             [\PYGZhy{}\PYGZhy{}force] [\PYGZhy{}\PYGZhy{}bstate\PYGZhy{}file BSTATE\PYGZus{}FILE] [\PYGZhy{}\PYGZhy{}bstate BSTATES]
             [\PYGZhy{}\PYGZhy{}tstate\PYGZhy{}file TSTATE\PYGZus{}FILE] [\PYGZhy{}\PYGZhy{}tstate TSTATES]
             [\PYGZhy{}\PYGZhy{}segs\PYGZhy{}per\PYGZhy{}state N] [\PYGZhy{}\PYGZhy{}no\PYGZhy{}we] [\PYGZhy{}\PYGZhy{}wm\PYGZhy{}work\PYGZhy{}manager WORK\PYGZus{}MANAGER]
             [\PYGZhy{}\PYGZhy{}wm\PYGZhy{}n\PYGZhy{}workers N\PYGZus{}WORKERS] [\PYGZhy{}\PYGZhy{}wm\PYGZhy{}zmq\PYGZhy{}mode MODE]
             [\PYGZhy{}\PYGZhy{}wm\PYGZhy{}zmq\PYGZhy{}info INFO\PYGZus{}FILE] [\PYGZhy{}\PYGZhy{}wm\PYGZhy{}zmq\PYGZhy{}task\PYGZhy{}endpoint TASK\PYGZus{}ENDPOINT]
             [\PYGZhy{}\PYGZhy{}wm\PYGZhy{}zmq\PYGZhy{}result\PYGZhy{}endpoint RESULT\PYGZus{}ENDPOINT]
             [\PYGZhy{}\PYGZhy{}wm\PYGZhy{}zmq\PYGZhy{}announce\PYGZhy{}endpoint ANNOUNCE\PYGZus{}ENDPOINT]
             [\PYGZhy{}\PYGZhy{}wm\PYGZhy{}zmq\PYGZhy{}heartbeat\PYGZhy{}interval INTERVAL]
             [\PYGZhy{}\PYGZhy{}wm\PYGZhy{}zmq\PYGZhy{}task\PYGZhy{}timeout TIMEOUT] [\PYGZhy{}\PYGZhy{}wm\PYGZhy{}zmq\PYGZhy{}client\PYGZhy{}comm\PYGZhy{}mode MODE]
\end{sphinxVerbatim}

Initialize a new WEST simulation, creating the WEST HDF5 file and preparing the
first iteration’s segments. Initial states are generated from one or more
“basis states” which are specified either in a file specified with
\sphinxcode{\sphinxupquote{\sphinxhyphen{}\sphinxhyphen{}bstates\sphinxhyphen{}from}}, or by one or more \sphinxcode{\sphinxupquote{\sphinxhyphen{}\sphinxhyphen{}bstate}} arguments. If neither
\sphinxcode{\sphinxupquote{\sphinxhyphen{}\sphinxhyphen{}bstates\sphinxhyphen{}from}} nor at least one \sphinxcode{\sphinxupquote{\sphinxhyphen{}\sphinxhyphen{}bstate}} argument is provided, then a
default basis state of probability one identified by the state ID zero and
label “basis” will be created (a warning will be printed in this case, to
remind you of this behavior, in case it is not what you wanted). Target states
for (non\sphinxhyphen{} equilibrium) steady\sphinxhyphen{}state simulations are specified either in a file
specified with \sphinxcode{\sphinxupquote{\sphinxhyphen{}\sphinxhyphen{}tstates\sphinxhyphen{}from}}, or by one or more \sphinxcode{\sphinxupquote{\sphinxhyphen{}\sphinxhyphen{}tstate}} arguments. If
neither \sphinxcode{\sphinxupquote{\sphinxhyphen{}\sphinxhyphen{}tstates\sphinxhyphen{}from}} nor at least one \sphinxcode{\sphinxupquote{\sphinxhyphen{}\sphinxhyphen{}tstate}} argument is provided,
then an equilibrium simulation (without any sinks) will be performed.


\subsubsection{Command\sphinxhyphen{}Line Options}
\label{\detokenize{users_guide/command_line_tools/w_init:command-line-options}}
See the \sphinxhref{UserGuide:ToolRefs}{general command\sphinxhyphen{}line tool reference} for more
information on the general options.


\paragraph{State Options}
\label{\detokenize{users_guide/command_line_tools/w_init:state-options}}
\begin{sphinxVerbatim}[commandchars=\\\{\}]
\PYG{o}{\PYGZhy{}}\PYG{o}{\PYGZhy{}}\PYG{n}{force}
  \PYG{n}{Overwrites} \PYG{n+nb}{any} \PYG{n}{existing} \PYG{n}{simulation} \PYG{n}{data}

\PYG{o}{\PYGZhy{}}\PYG{o}{\PYGZhy{}}\PYG{n}{bstate} \PYG{n}{BSTATES}
  \PYG{n}{Add} \PYG{n}{the} \PYG{n}{given} \PYG{n}{basis} \PYG{n}{state} \PYG{p}{(}\PYG{n}{specified} \PYG{k}{as} \PYG{n}{a} \PYG{n}{string}
  \PYG{l+s+s1}{\PYGZsq{}}\PYG{l+s+s1}{label,probability[,auxref]}\PYG{l+s+s1}{\PYGZsq{}}\PYG{p}{)} \PYG{n}{to} \PYG{n}{the} \PYG{n+nb}{list} \PYG{n}{of} \PYG{n}{basis} \PYG{n}{states} \PYG{p}{(}\PYG{n}{after}
  \PYG{n}{those} \PYG{n}{specified} \PYG{o+ow}{in} \PYG{o}{\PYGZhy{}}\PYG{o}{\PYGZhy{}}\PYG{n}{bstates}\PYG{o}{\PYGZhy{}}\PYG{n}{from}\PYG{p}{,} \PYG{k}{if} \PYG{n+nb}{any}\PYG{p}{)}\PYG{o}{.} \PYG{n}{This} \PYG{n}{argument} \PYG{n}{may} \PYG{n}{be}
  \PYG{n}{specified} \PYG{n}{more} \PYG{n}{than} \PYG{n}{once}\PYG{p}{,} \PYG{o+ow}{in} \PYG{n}{which} \PYG{n}{case} \PYG{n}{the} \PYG{n}{given} \PYG{n}{states} \PYG{n}{are}
  \PYG{n}{appended} \PYG{o+ow}{in} \PYG{n}{the} \PYG{n}{order} \PYG{n}{they} \PYG{n}{are} \PYG{n}{given} \PYG{n}{on} \PYG{n}{the} \PYG{n}{command} \PYG{n}{line}\PYG{o}{.}

\PYG{o}{\PYGZhy{}}\PYG{o}{\PYGZhy{}}\PYG{n}{bstate}\PYG{o}{\PYGZhy{}}\PYG{n}{file} \PYG{n}{BSTATE\PYGZus{}FILE}\PYG{p}{,} \PYG{o}{\PYGZhy{}}\PYG{o}{\PYGZhy{}}\PYG{n}{bstates}\PYG{o}{\PYGZhy{}}\PYG{k+kn}{from} \PYG{n+nn}{BSTATE\PYGZus{}FILE}
  \PYG{n}{Read} \PYG{n}{basis} \PYG{n}{state} \PYG{n}{names}\PYG{p}{,} \PYG{n}{probabilities}\PYG{p}{,} \PYG{o+ow}{and} \PYG{p}{(}\PYG{n}{optionally}\PYG{p}{)} \PYG{n}{data}
  \PYG{n}{references} \PYG{k+kn}{from} \PYG{n+nn}{BSTATE\PYGZus{}FILE}\PYG{n+nn}{.}

\PYG{o}{\PYGZhy{}}\PYG{o}{\PYGZhy{}}\PYG{n}{tstate} \PYG{n}{TSTATES}
  \PYG{n}{Add} \PYG{n}{the} \PYG{n}{given} \PYG{n}{target} \PYG{n}{state} \PYG{p}{(}\PYG{n}{specified} \PYG{k}{as} \PYG{n}{a} \PYG{n}{string}
  \PYG{l+s+s1}{\PYGZsq{}}\PYG{l+s+s1}{label,pcoord0[,pcoord1[,...]]}\PYG{l+s+s1}{\PYGZsq{}}\PYG{p}{)} \PYG{n}{to} \PYG{n}{the} \PYG{n+nb}{list} \PYG{n}{of} \PYG{n}{target} \PYG{n}{states} \PYG{p}{(}\PYG{n}{after}
  \PYG{n}{those} \PYG{n}{specified} \PYG{o+ow}{in} \PYG{n}{the} \PYG{n}{file} \PYG{n}{given} \PYG{n}{by} \PYG{o}{\PYGZhy{}}\PYG{o}{\PYGZhy{}}\PYG{n}{tstates}\PYG{o}{\PYGZhy{}}\PYG{n}{from}\PYG{p}{,} \PYG{k}{if} \PYG{n+nb}{any}\PYG{p}{)}\PYG{o}{.} \PYG{n}{This}
  \PYG{n}{argument} \PYG{n}{may} \PYG{n}{be} \PYG{n}{specified} \PYG{n}{more} \PYG{n}{than} \PYG{n}{once}\PYG{p}{,} \PYG{o+ow}{in} \PYG{n}{which} \PYG{n}{case} \PYG{n}{the} \PYG{n}{given}
  \PYG{n}{states} \PYG{n}{are} \PYG{n}{appended} \PYG{o+ow}{in} \PYG{n}{the} \PYG{n}{order} \PYG{n}{they} \PYG{n}{appear} \PYG{n}{on} \PYG{n}{the} \PYG{n}{command} \PYG{n}{line}\PYG{o}{.}

\PYG{o}{\PYGZhy{}}\PYG{o}{\PYGZhy{}}\PYG{n}{tstate}\PYG{o}{\PYGZhy{}}\PYG{n}{file} \PYG{n}{TSTATE\PYGZus{}FILE}\PYG{p}{,} \PYG{o}{\PYGZhy{}}\PYG{o}{\PYGZhy{}}\PYG{n}{tstates}\PYG{o}{\PYGZhy{}}\PYG{k+kn}{from} \PYG{n+nn}{TSTATE\PYGZus{}FILE}
  \PYG{n}{Read} \PYG{n}{target} \PYG{n}{state} \PYG{n}{names} \PYG{o+ow}{and} \PYG{n}{representative} \PYG{n}{progress} \PYG{n}{coordinates} \PYG{k+kn}{from}
  \PYG{n+nn}{TSTATE\PYGZus{}FILE}\PYG{n+nn}{.}

\PYG{o}{\PYGZhy{}}\PYG{o}{\PYGZhy{}}\PYG{n}{segs}\PYG{o}{\PYGZhy{}}\PYG{n}{per}\PYG{o}{\PYGZhy{}}\PYG{n}{state} \PYG{n}{N}
  \PYG{n}{Initialize} \PYG{n}{N} \PYG{n}{segments} \PYG{k+kn}{from} \PYG{n+nn}{each} \PYG{n}{basis} \PYG{n}{state} \PYG{p}{(}\PYG{n}{default}\PYG{p}{:} \PYG{l+m+mi}{1}\PYG{p}{)}\PYG{o}{.}

\PYG{o}{\PYGZhy{}}\PYG{o}{\PYGZhy{}}\PYG{n}{no}\PYG{o}{\PYGZhy{}}\PYG{n}{we}\PYG{p}{,} \PYG{o}{\PYGZhy{}}\PYG{o}{\PYGZhy{}}\PYG{n}{shotgun}
  \PYG{n}{Do} \PYG{o+ow}{not} \PYG{n}{run} \PYG{n}{the} \PYG{n}{weighted} \PYG{n}{ensemble} \PYG{n+nb}{bin}\PYG{o}{/}\PYG{n}{split}\PYG{o}{/}\PYG{n}{merge} \PYG{n}{algorithm} \PYG{n}{on}
  \PYG{n}{newly}\PYG{o}{\PYGZhy{}}\PYG{n}{created} \PYG{n}{segments}\PYG{o}{.}
\end{sphinxVerbatim}


\subsubsection{Examples}
\label{\detokenize{users_guide/command_line_tools/w_init:examples}}
(TODO: write 3 examples; Setting up the basis states, explanation of
bstates and istates. Setting up an equilibrium simulation, w/o target(s)
for recycling. Setting up a simulation with one/multiple target states.)


\subsection{w\_kinavg}
\label{\detokenize{users_guide/command_line_tools/w_kinavg:w-kinavg}}\label{\detokenize{users_guide/command_line_tools/w_kinavg:id1}}\label{\detokenize{users_guide/command_line_tools/w_kinavg::doc}}

\subsection{w\_kinetics}
\label{\detokenize{users_guide/command_line_tools/w_kinetics:w-kinetics}}\label{\detokenize{users_guide/command_line_tools/w_kinetics:id1}}\label{\detokenize{users_guide/command_line_tools/w_kinetics::doc}}

\subsection{w\_ntop}
\label{\detokenize{users_guide/command_line_tools/w_ntop:w-ntop}}\label{\detokenize{users_guide/command_line_tools/w_ntop:id1}}\label{\detokenize{users_guide/command_line_tools/w_ntop::doc}}

\subsection{w\_pdist}
\label{\detokenize{users_guide/command_line_tools/w_pdist:w-pdist}}\label{\detokenize{users_guide/command_line_tools/w_pdist:id1}}\label{\detokenize{users_guide/command_line_tools/w_pdist::doc}}
\sphinxcode{\sphinxupquote{w\_pcpdist}} constructs and calculates the progress coordinate probability
distribution’s evolution over a user\sphinxhyphen{}specified number of simulation iterations.
\sphinxcode{\sphinxupquote{w\_pcpdist}} supports progress coordinates with dimensionality ≥ 1.

The resulting distribution can be viewed with the {\hyperref[\detokenize{users_guide/command_line_tools/plothist:plothist}]{\sphinxcrossref{\DUrole{std,std-ref}{plothist}}}} tool.


\subsubsection{Overview}
\label{\detokenize{users_guide/command_line_tools/w_pdist:overview}}
Usage:

\begin{sphinxVerbatim}[commandchars=\\\{\}]
\PYGZdl{}WEST\PYGZus{}ROOT/bin/w\PYGZus{}pdist [\PYGZhy{}h] [\PYGZhy{}r RCFILE] [\PYGZhy{}\PYGZhy{}quiet | \PYGZhy{}\PYGZhy{}verbose | \PYGZhy{}\PYGZhy{}debug] [\PYGZhy{}\PYGZhy{}version]
                       [\PYGZhy{}W WEST\PYGZus{}H5FILE] [\PYGZhy{}\PYGZhy{}first\PYGZhy{}iter N\PYGZus{}ITER] [\PYGZhy{}\PYGZhy{}last\PYGZhy{}iter N\PYGZus{}ITER]
                       [\PYGZhy{}b BINEXPR] [\PYGZhy{}o OUTPUT]
                       [\PYGZhy{}\PYGZhy{}construct\PYGZhy{}dataset CONSTRUCT\PYGZus{}DATASET | \PYGZhy{}\PYGZhy{}dsspecs DSSPEC [DSSPEC ...]]
                       [\PYGZhy{}\PYGZhy{}serial | \PYGZhy{}\PYGZhy{}parallel | \PYGZhy{}\PYGZhy{}work\PYGZhy{}manager WORK\PYGZus{}MANAGER]
                       [\PYGZhy{}\PYGZhy{}n\PYGZhy{}workers N\PYGZus{}WORKERS] [\PYGZhy{}\PYGZhy{}zmq\PYGZhy{}mode MODE]
                       [\PYGZhy{}\PYGZhy{}zmq\PYGZhy{}info INFO\PYGZus{}FILE] [\PYGZhy{}\PYGZhy{}zmq\PYGZhy{}task\PYGZhy{}endpoint TASK\PYGZus{}ENDPOINT]
                       [\PYGZhy{}\PYGZhy{}zmq\PYGZhy{}result\PYGZhy{}endpoint RESULT\PYGZus{}ENDPOINT]
                       [\PYGZhy{}\PYGZhy{}zmq\PYGZhy{}announce\PYGZhy{}endpoint ANNOUNCE\PYGZus{}ENDPOINT]
                       [\PYGZhy{}\PYGZhy{}zmq\PYGZhy{}listen\PYGZhy{}endpoint ANNOUNCE\PYGZus{}ENDPOINT]
                       [\PYGZhy{}\PYGZhy{}zmq\PYGZhy{}heartbeat\PYGZhy{}interval INTERVAL]
                       [\PYGZhy{}\PYGZhy{}zmq\PYGZhy{}task\PYGZhy{}timeout TIMEOUT] [\PYGZhy{}\PYGZhy{}zmq\PYGZhy{}client\PYGZhy{}comm\PYGZhy{}mode MODE]
\end{sphinxVerbatim}

Note: This tool supports parallelization, which may be more efficient for
especially large datasets.


\subsubsection{Command\sphinxhyphen{}Line Options}
\label{\detokenize{users_guide/command_line_tools/w_pdist:command-line-options}}
See the \sphinxhref{UserGuide:ToolRefs}{general command\sphinxhyphen{}line tool reference} for more
information on the general options.


\paragraph{Input/output options}
\label{\detokenize{users_guide/command_line_tools/w_pdist:input-output-options}}
These arguments allow the user to specify where to read input simulation result
data and where to output calculated progress coordinate probability
distribution data.

Both input and output files are \sphinxstyleemphasis{hdf5} format:

\begin{sphinxVerbatim}[commandchars=\\\{\}]
\PYG{o}{\PYGZhy{}}\PYG{n}{W}\PYG{p}{,} \PYG{o}{\PYGZhy{}}\PYG{o}{\PYGZhy{}}\PYG{n}{WEST\PYGZus{}H5FILE} \PYG{n}{file}
  \PYG{n}{Read} \PYG{n}{simulation} \PYG{n}{result} \PYG{n}{data} \PYG{k+kn}{from} \PYG{n+nn}{file} \PYG{o}{*}\PYG{n}{file}\PYG{o}{*}\PYG{o}{.} \PYG{p}{(}\PYG{o}{*}\PYG{o}{*}\PYG{n}{Default}\PYG{p}{:}\PYG{o}{*}\PYG{o}{*} \PYG{n}{The}
  \PYG{o}{*}\PYG{n}{hdf5}\PYG{o}{*} \PYG{n}{file} \PYG{n}{specified} \PYG{o+ow}{in} \PYG{n}{the} \PYG{n}{configuration} \PYG{n}{file} \PYG{p}{(}\PYG{n}{default} \PYG{n}{config} \PYG{n}{file}
  \PYG{o+ow}{is} \PYG{o}{*}\PYG{n}{west}\PYG{o}{.}\PYG{n}{h5}\PYG{o}{*}\PYG{p}{)}\PYG{p}{)}

\PYG{o}{\PYGZhy{}}\PYG{n}{o}\PYG{p}{,} \PYG{o}{\PYGZhy{}}\PYG{o}{\PYGZhy{}}\PYG{n}{output} \PYG{n}{file}
  \PYG{n}{Store} \PYG{n}{this} \PYG{n}{tool}\PYG{l+s+s1}{\PYGZsq{}}\PYG{l+s+s1}{s output in *file*. (**Default:** The *hdf5* file}
  \PYG{o}{*}\PYG{o}{*}\PYG{n}{pcpdist}\PYG{o}{.}\PYG{n}{h5}\PYG{o}{*}\PYG{o}{*}\PYG{p}{)}
\end{sphinxVerbatim}


\paragraph{Iteration range options}
\label{\detokenize{users_guide/command_line_tools/w_pdist:iteration-range-options}}
Specify the range of iterations over which to construct the progress
coordinate probability distribution.:

\begin{sphinxVerbatim}[commandchars=\\\{\}]
\PYG{o}{\PYGZhy{}}\PYG{o}{\PYGZhy{}}\PYG{n}{first}\PYG{o}{\PYGZhy{}}\PYG{n+nb}{iter} \PYG{n}{n\PYGZus{}iter}
  \PYG{n}{Construct} \PYG{n}{probability} \PYG{n}{distribution} \PYG{n}{starting} \PYG{k}{with} \PYG{n}{iteration} \PYG{o}{*}\PYG{n}{n\PYGZus{}iter}\PYG{o}{*}
  \PYG{p}{(}\PYG{o}{*}\PYG{o}{*}\PYG{n}{Default}\PYG{p}{:}\PYG{o}{*}\PYG{o}{*} \PYG{l+m+mi}{1}\PYG{p}{)}

\PYG{o}{\PYGZhy{}}\PYG{o}{\PYGZhy{}}\PYG{n}{last}\PYG{o}{\PYGZhy{}}\PYG{n+nb}{iter} \PYG{n}{n\PYGZus{}iter}
  \PYG{n}{Construct} \PYG{n}{probability} \PYG{n}{distribution}\PYG{l+s+s1}{\PYGZsq{}}\PYG{l+s+s1}{s time evolution up to (and}
  \PYG{n}{including}\PYG{p}{)} \PYG{n}{iteration} \PYG{o}{*}\PYG{n}{n\PYGZus{}iter}\PYG{o}{*} \PYG{p}{(}\PYG{o}{*}\PYG{o}{*}\PYG{n}{Default}\PYG{p}{:}\PYG{o}{*}\PYG{o}{*} \PYG{n}{Last} \PYG{n}{completed}
  \PYG{n}{iteration}\PYG{p}{)}
\end{sphinxVerbatim}


\paragraph{Probability distribution binning options}
\label{\detokenize{users_guide/command_line_tools/w_pdist:probability-distribution-binning-options}}
Specify the number of bins to use when constructing the progress
coordinate probability distribution. If using a multidimensional
progress coordinate, different binning schemes can be used for the
probability distribution for each progress coordinate.:

\begin{sphinxVerbatim}[commandchars=\\\{\}]
\PYG{o}{\PYGZhy{}}\PYG{n}{b} \PYG{n}{binexpr}
  \PYG{o}{*}\PYG{n}{binexpr}\PYG{o}{*} \PYG{n}{specifies} \PYG{n}{the} \PYG{n}{number} \PYG{o+ow}{and} \PYG{n}{formatting} \PYG{n}{of} \PYG{n}{the} \PYG{n}{bins}\PYG{o}{.} \PYG{n}{Its}
  \PYG{n+nb}{format} \PYG{n}{can} \PYG{n}{be} \PYG{k}{as} \PYG{n}{follows}\PYG{p}{:}

      \PYG{l+m+mf}{1.} \PYG{n}{an} \PYG{n}{integer}\PYG{p}{,} \PYG{o+ow}{in} \PYG{n}{which} \PYG{n}{case} \PYG{n+nb}{all} \PYG{n}{distributions} \PYG{n}{have} \PYG{n}{that} \PYG{n}{many}
      \PYG{n}{equal} \PYG{n}{sized} \PYG{n}{bins}
      \PYG{l+m+mf}{2.} \PYG{n}{a} \PYG{n}{python}\PYG{o}{\PYGZhy{}}\PYG{n}{style} \PYG{n+nb}{list} \PYG{n}{of} \PYG{n}{integers}\PYG{p}{,} \PYG{n}{of} \PYG{n}{length} \PYG{n}{corresponding} \PYG{n}{to}
      \PYG{n}{the} \PYG{n}{number} \PYG{n}{of} \PYG{n}{dimensions} \PYG{n}{of} \PYG{n}{the} \PYG{n}{progress} \PYG{n}{coordinate}\PYG{p}{,} \PYG{o+ow}{in} \PYG{n}{which}
      \PYG{n}{case} \PYG{n}{each} \PYG{n}{progress} \PYG{n}{coordinate}\PYG{l+s+s1}{\PYGZsq{}}\PYG{l+s+s1}{s probability distribution has the}
      \PYG{n}{corresponding} \PYG{n}{number} \PYG{n}{of} \PYG{n}{bins}
      \PYG{l+m+mf}{3.} \PYG{n}{a} \PYG{n}{python}\PYG{o}{\PYGZhy{}}\PYG{n}{style} \PYG{n+nb}{list} \PYG{n}{of} \PYG{n}{lists} \PYG{n}{of} \PYG{n}{scalars}\PYG{p}{,} \PYG{n}{where} \PYG{n}{the} \PYG{n+nb}{list} \PYG{n}{at}
      \PYG{n}{each} \PYG{n}{index} \PYG{n}{corresponds} \PYG{n}{to} \PYG{n}{each} \PYG{n}{dimension} \PYG{n}{of} \PYG{n}{the} \PYG{n}{progress}
      \PYG{n}{coordinate} \PYG{o+ow}{and} \PYG{n}{specifies} \PYG{n}{specific} \PYG{n+nb}{bin} \PYG{n}{boundaries} \PYG{k}{for} \PYG{n}{that}
      \PYG{n}{progress} \PYG{n}{coordinate}\PYG{l+s+s1}{\PYGZsq{}}\PYG{l+s+s1}{s probability distribution.}

  \PYG{p}{(}\PYG{o}{*}\PYG{o}{*}\PYG{n}{Default}\PYG{p}{:}\PYG{o}{*}\PYG{o}{*} \PYG{l+m+mi}{100} \PYG{n}{bins} \PYG{k}{for} \PYG{n+nb}{all} \PYG{n}{progress} \PYG{n}{coordinates}\PYG{p}{)}
\end{sphinxVerbatim}


\subsubsection{Examples}
\label{\detokenize{users_guide/command_line_tools/w_pdist:examples}}
Assuming simulation results are stored in \sphinxstyleemphasis{west.h5} (which is specified in the
configuration file named \sphinxstyleemphasis{west.cfg}), for a simulation with a 1\sphinxhyphen{}dimensional
progress coordinate:

Calculate a probability distribution histogram using all default options
(output file: \sphinxstyleemphasis{pdist.h5}; histogram binning: 100 equal sized bins; probability
distribution over the lowest reached progress coordinate to the largest; work
is parallelized over all available local cores using the ‘processes’ work
manager):

\begin{sphinxVerbatim}[commandchars=\\\{\}]
\PYGZdl{}WEST\PYGZus{}ROOT/bin/w\PYGZus{}pdist
\end{sphinxVerbatim}

Same as above, except using the serial work manager (which may be more
efficient for smaller datasets):

\begin{sphinxVerbatim}[commandchars=\\\{\}]
\PYGZdl{}WEST\PYGZus{}ROOT/bin/w\PYGZus{}pdist \PYGZhy{}\PYGZhy{}serial
\end{sphinxVerbatim}


\subsection{w\_run}
\label{\detokenize{users_guide/command_line_tools/w_run:w-run}}\label{\detokenize{users_guide/command_line_tools/w_run:id1}}\label{\detokenize{users_guide/command_line_tools/w_run::doc}}
\sphinxcode{\sphinxupquote{w\_run}} starts or continues a weighted ensemble simualtion.


\subsubsection{Overview}
\label{\detokenize{users_guide/command_line_tools/w_run:overview}}
Usage:

\begin{sphinxVerbatim}[commandchars=\\\{\}]
\PYGZdl{}WEST\PYGZus{}ROOT/bin/w\PYGZus{}run [\PYGZhy{}h] [\PYGZhy{}r RCFILE] [\PYGZhy{}\PYGZhy{}quiet | \PYGZhy{}\PYGZhy{}verbose | \PYGZhy{}\PYGZhy{}debug] [\PYGZhy{}\PYGZhy{}version]
             [\PYGZhy{}\PYGZhy{}oneseg ] [\PYGZhy{}\PYGZhy{}wm\PYGZhy{}work\PYGZhy{}manager WORK\PYGZus{}MANAGER]
             [\PYGZhy{}\PYGZhy{}wm\PYGZhy{}n\PYGZhy{}workers N\PYGZus{}WORKERS] [\PYGZhy{}\PYGZhy{}wm\PYGZhy{}zmq\PYGZhy{}mode MODE]
             [\PYGZhy{}\PYGZhy{}wm\PYGZhy{}zmq\PYGZhy{}info INFO\PYGZus{}FILE] [\PYGZhy{}\PYGZhy{}wm\PYGZhy{}zmq\PYGZhy{}task\PYGZhy{}endpoint TASK\PYGZus{}ENDPOINT]
             [\PYGZhy{}\PYGZhy{}wm\PYGZhy{}zmq\PYGZhy{}result\PYGZhy{}endpoint RESULT\PYGZus{}ENDPOINT]
             [\PYGZhy{}\PYGZhy{}wm\PYGZhy{}zmq\PYGZhy{}announce\PYGZhy{}endpoint ANNOUNCE\PYGZus{}ENDPOINT]
             [\PYGZhy{}\PYGZhy{}wm\PYGZhy{}zmq\PYGZhy{}heartbeat\PYGZhy{}interval INTERVAL]
             [\PYGZhy{}\PYGZhy{}wm\PYGZhy{}zmq\PYGZhy{}task\PYGZhy{}timeout TIMEOUT] [\PYGZhy{}\PYGZhy{}wm\PYGZhy{}zmq\PYGZhy{}client\PYGZhy{}comm\PYGZhy{}mode MODE]
\end{sphinxVerbatim}


\subsubsection{Command\sphinxhyphen{}Line Options}
\label{\detokenize{users_guide/command_line_tools/w_run:command-line-options}}
See the {\hyperref[\detokenize{users_guide/command_line_tools:command-line-tool-index}]{\sphinxcrossref{\DUrole{std,std-ref}{command\sphinxhyphen{}line tool index}}}} for
more information on the general options.


\paragraph{Segment Options}
\label{\detokenize{users_guide/command_line_tools/w_run:segment-options}}\begin{description}
\item[{::}] \leavevmode\begin{optionlist}{3cm}
\item [\sphinxhyphen{}\sphinxhyphen{}oneseg]  
Only propagate one segment (useful for debugging propagators)
\end{optionlist}

\end{description}


\subsubsection{Example}
\label{\detokenize{users_guide/command_line_tools/w_run:example}}
A simple example for using w\_run (mostly taken from odld example that
is available in the main WESTPA distribution):

\begin{sphinxVerbatim}[commandchars=\\\{\}]
\PYGZdl{}WEST\PYGZus{}ROOT/bin/w\PYGZus{}run \PYGZam{}\PYGZgt{} west.log
\end{sphinxVerbatim}

This commands starts up a serial weighted ensemble run and pipes the results
into the west.log file. As a side note \sphinxcode{\sphinxupquote{\sphinxhyphen{}\sphinxhyphen{}debug}} option is very useful for
debugging the code if something goes wrong.


\subsection{w\_select}
\label{\detokenize{users_guide/command_line_tools/w_select:w-select}}\label{\detokenize{users_guide/command_line_tools/w_select:id1}}\label{\detokenize{users_guide/command_line_tools/w_select::doc}}

\subsection{w\_stateprobs}
\label{\detokenize{users_guide/command_line_tools/w_stateprobs:w-stateprobs}}\label{\detokenize{users_guide/command_line_tools/w_stateprobs:id1}}\label{\detokenize{users_guide/command_line_tools/w_stateprobs::doc}}

\subsection{w\_states}
\label{\detokenize{users_guide/command_line_tools/w_states:w-states}}\label{\detokenize{users_guide/command_line_tools/w_states:id1}}\label{\detokenize{users_guide/command_line_tools/w_states::doc}}

\subsection{w\_succ}
\label{\detokenize{users_guide/command_line_tools/w_succ:w-succ}}\label{\detokenize{users_guide/command_line_tools/w_succ:id1}}\label{\detokenize{users_guide/command_line_tools/w_succ::doc}}

\subsection{w\_trace}
\label{\detokenize{users_guide/command_line_tools/w_trace:w-trace}}\label{\detokenize{users_guide/command_line_tools/w_trace:id1}}\label{\detokenize{users_guide/command_line_tools/w_trace::doc}}

\subsection{w\_truncate}
\label{\detokenize{users_guide/command_line_tools/w_truncate:w-truncate}}\label{\detokenize{users_guide/command_line_tools/w_truncate:id1}}\label{\detokenize{users_guide/command_line_tools/w_truncate::doc}}
\sphinxcode{\sphinxupquote{w\_truncate}} removes all iterations after a certain point


\subsubsection{Overview}
\label{\detokenize{users_guide/command_line_tools/w_truncate:overview}}
Usage:

\begin{sphinxVerbatim}[commandchars=\\\{\}]
\PYGZdl{}WEST\PYGZus{}ROOT/bin/w\PYGZus{}truncate [\PYGZhy{}h] [\PYGZhy{}r RCFILE] [\PYGZhy{}\PYGZhy{}quiet | \PYGZhy{}\PYGZhy{}verbose | \PYGZhy{}\PYGZhy{}debug] [\PYGZhy{}\PYGZhy{}version]
                 [\PYGZhy{}n N\PYGZus{}ITER]
\end{sphinxVerbatim}

Remove all iterations after a certain point in a


\subsubsection{Command\sphinxhyphen{}Line Options}
\label{\detokenize{users_guide/command_line_tools/w_truncate:command-line-options}}
See the \sphinxtitleref{command\sphinxhyphen{}line tool index \textless{}command\_line\_tool\_index\textgreater{}} for more
information on the general options.


\paragraph{Iteration Options}
\label{\detokenize{users_guide/command_line_tools/w_truncate:iteration-options}}
\begin{sphinxVerbatim}[commandchars=\\\{\}]
\PYG{o}{\PYGZhy{}}\PYG{n}{n} \PYG{n}{N\PYGZus{}ITER}\PYG{p}{,} \PYG{o}{\PYGZhy{}}\PYG{o}{\PYGZhy{}}\PYG{n+nb}{iter} \PYG{n}{N\PYGZus{}ITER}
  \PYG{n}{Truncate} \PYG{n}{this} \PYG{n}{iteration} \PYG{o+ow}{and} \PYG{n}{those} \PYG{n}{following}\PYG{o}{.}
\end{sphinxVerbatim}


\subsubsection{Examples}
\label{\detokenize{users_guide/command_line_tools/w_truncate:examples}}
(TODO: Write up an example)


\subsection{ploterr}
\label{\detokenize{users_guide/command_line_tools/ploterr:ploterr}}\label{\detokenize{users_guide/command_line_tools/ploterr:id1}}\label{\detokenize{users_guide/command_line_tools/ploterr::doc}}

\subsection{plothist}
\label{\detokenize{users_guide/command_line_tools/plothist:plothist}}\label{\detokenize{users_guide/command_line_tools/plothist:id1}}\label{\detokenize{users_guide/command_line_tools/plothist::doc}}
Use the \sphinxcode{\sphinxupquote{plothist}} tool to plot the results of {\hyperref[\detokenize{users_guide/command_line_tools/w_pdist:w-pdist}]{\sphinxcrossref{\DUrole{std,std-ref}{w\_pdist}}}}. This tool uses
an \sphinxstyleemphasis{hdf5} file as its input (i.e. the output of another analysis tool), and
outputs a \sphinxstyleemphasis{pdf} image.

The \sphinxcode{\sphinxupquote{plothist}} tool operates in one of three (mutually exclusive)
plotting modes:
\begin{itemize}
\item {} 
\sphinxstylestrong{\textasciigrave{}\textasciigrave{}evolution\textasciigrave{}\textasciigrave{}}: Plots the relevant data as a time evolution over
specified number of simulation iterations

\item {} 
\sphinxstylestrong{\textasciigrave{}\textasciigrave{}average\textasciigrave{}\textasciigrave{}}: Plots the relevant data as a time average over a
specified number of iterations

\item {} 
\sphinxstylestrong{\textasciigrave{}\textasciigrave{}instant\textasciigrave{}\textasciigrave{}}: Plots the relevant data for a single specified
iteration

\end{itemize}


\subsubsection{Overview}
\label{\detokenize{users_guide/command_line_tools/plothist:overview}}
The basic usage, independent of plotting mode, is as follows:

usage:

\begin{DUlineblock}{0em}
\item[] \sphinxcode{\sphinxupquote{\$WEST\_ROOT/bin/plothist {[}\sphinxhyphen{}h{]} {[}\sphinxhyphen{}r RCFILE{]} {[}\sphinxhyphen{}\sphinxhyphen{}quiet | \sphinxhyphen{}\sphinxhyphen{}verbose | \sphinxhyphen{}\sphinxhyphen{}debug{]} {[}\sphinxhyphen{}\sphinxhyphen{}version{]}}}
\item[] \textasciigrave{}\textasciigrave{}                        \{instant,average,evolution\} input …\textasciigrave{}\textasciigrave{}
\end{DUlineblock}

Note that the user must specify a plotting mode (i.e. ‘\sphinxcode{\sphinxupquote{instant}}‘,
‘\sphinxcode{\sphinxupquote{average}}‘, or ‘\sphinxcode{\sphinxupquote{evolution}}‘) and an input file, \sphinxcode{\sphinxupquote{input}}.

Therefore, this tool is always called as:

\sphinxcode{\sphinxupquote{\$WEST\_ROOT/bin/plothist mode input\_file {[}}}\sphinxstyleemphasis{\textasciigrave{}\textasciigrave{}other\textasciigrave{}\textasciigrave{}
\textasciigrave{}\textasciigrave{}options\textasciigrave{}\textasciigrave{}}\sphinxcode{\sphinxupquote{{]}}}


\paragraph{‘\sphinxstyleliteralintitle{\sphinxupquote{instant}}‘ mode}
\label{\detokenize{users_guide/command_line_tools/plothist:instant-mode}}
usage:

\begin{DUlineblock}{0em}
\item[] \sphinxcode{\sphinxupquote{\$WEST\_ROOT/bin/plothist instant {[}\sphinxhyphen{}h{]} input {[}\sphinxhyphen{}o PLOT\_OUTPUT{]}}}
\item[] \textasciigrave{}\textasciigrave{}                                {[}\textendash{}hdf5\sphinxhyphen{}output HDF5\_OUTPUT{]} {[}\textendash{}text\sphinxhyphen{}output TEXT\_OUTPUT{]}\textasciigrave{}\textasciigrave{}
\item[] \textasciigrave{}\textasciigrave{}                                {[}\textendash{}title TITLE{]} {[}\textendash{}range RANGE{]} {[}\textendash{}linear | \textendash{}energy | \textendash{}log10{]}\textasciigrave{}\textasciigrave{}
\item[] \textasciigrave{}\textasciigrave{}                                {[}\textendash{}iter N\_ITER{]} \textasciigrave{}\textasciigrave{}
\item[] \textasciigrave{}\textasciigrave{}                                {[}DIMENSION{]} {[}ADDTLDIM{]}\textasciigrave{}\textasciigrave{}
\end{DUlineblock}


\paragraph{‘\sphinxstyleliteralintitle{\sphinxupquote{average}}‘ mode}
\label{\detokenize{users_guide/command_line_tools/plothist:average-mode}}
usage:

\begin{DUlineblock}{0em}
\item[] \sphinxcode{\sphinxupquote{\$WEST\_ROOT/bin/plothist average {[}\sphinxhyphen{}h{]} input {[}\sphinxhyphen{}o PLOT\_OUTPUT{]}}}
\item[] \textasciigrave{}\textasciigrave{}                                {[}\textendash{}hdf5\sphinxhyphen{}output HDF5\_OUTPUT{]} {[}\textendash{}text\sphinxhyphen{}output TEXT\_OUTPUT{]}\textasciigrave{}\textasciigrave{}
\item[] \textasciigrave{}\textasciigrave{}                                {[}\textendash{}title TITLE{]} {[}\textendash{}range RANGE{]} {[}\textendash{}linear | \textendash{}energy | \textendash{}log10{]}\textasciigrave{}\textasciigrave{}
\item[] \textasciigrave{}\textasciigrave{}                                {[}\textendash{}first\sphinxhyphen{}iter N\_ITER{]} {[}\textendash{}last\sphinxhyphen{}iter N\_ITER{]}                           \textasciigrave{}\textasciigrave{}
\item[] \textasciigrave{}\textasciigrave{}                                {[}DIMENSION{]} {[}ADDTLDIM{]}\textasciigrave{}\textasciigrave{}
\end{DUlineblock}


\paragraph{‘\sphinxstyleliteralintitle{\sphinxupquote{evolution}}‘ mode}
\label{\detokenize{users_guide/command_line_tools/plothist:evolution-mode}}
usage:

\begin{DUlineblock}{0em}
\item[] \sphinxcode{\sphinxupquote{\$WEST\_ROOT/bin/plothist evolution {[}\sphinxhyphen{}h{]} input {[}\sphinxhyphen{}o PLOT\_OUTPUT{]}}}
\item[] \textasciigrave{}\textasciigrave{}                                  {[}\textendash{}hdf5\sphinxhyphen{}output HDF5\_OUTPUT{]}\textasciigrave{}\textasciigrave{}
\item[] \textasciigrave{}\textasciigrave{}                                  {[}\textendash{}title TITLE{]} {[}\textendash{}range RANGE{]} {[}\textendash{}linear | \textendash{}energy | \textendash{}log10{]}\textasciigrave{}\textasciigrave{}
\item[] \textasciigrave{}\textasciigrave{}                                  {[}\textendash{}first\sphinxhyphen{}iter N\_ITER{]} {[}\textendash{}last\sphinxhyphen{}iter N\_ITER{]}\textasciigrave{}\textasciigrave{}
\item[] \textasciigrave{}\textasciigrave{}                                  {[}\textendash{}step\sphinxhyphen{}iter STEP{]}                                   \textasciigrave{}\textasciigrave{}
\item[] \textasciigrave{}\textasciigrave{}                                  {[}DIMENSION{]}\textasciigrave{}\textasciigrave{}
\end{DUlineblock}


\subsubsection{Command\sphinxhyphen{}Line Options}
\label{\detokenize{users_guide/command_line_tools/plothist:command-line-options}}
See the {\hyperref[\detokenize{users_guide/command_line_tools:command-line-tool-index}]{\sphinxcrossref{\DUrole{std,std-ref}{command\sphinxhyphen{}line tool index}}}} for more
information on the general options.

Unless specified (as a \sphinxstylestrong{Note} in the command\sphinxhyphen{}line option description), the
command\sphinxhyphen{}line options below are shared for all three plotting modes


\paragraph{Input/output options}
\label{\detokenize{users_guide/command_line_tools/plothist:input-output-options}}
No matter the mode, an input \sphinxstyleemphasis{hdf5} file must be specified. There are
three possible outputs that are mode or user\sphinxhyphen{}specified: A text file, an
\sphinxstyleemphasis{hdf5} file, and a pdf image.


\subparagraph{Specifying input file}
\label{\detokenize{users_guide/command_line_tools/plothist:specifying-input-file}}\begin{description}
\item[{\sphinxstylestrong{*\textasciigrave{}\textasciigrave{}input\textasciigrave{}\textasciigrave{}*}}] \leavevmode
Specify the input \sphinxstyleemphasis{hdf5} file ‘’\sphinxcode{\sphinxupquote{input}}. This is the output file
from a previous analysis tool (e.g. ‘pcpdist.h5’)

\end{description}


\subparagraph{Output plot pdf file}
\label{\detokenize{users_guide/command_line_tools/plothist:output-plot-pdf-file}}\begin{description}
\item[{\sphinxstylestrong{\textasciigrave{}\textasciigrave{}\sphinxhyphen{}o ‘’plot\_output’’, \textendash{}plot\_output ‘’plot\_output’’\textasciigrave{}\textasciigrave{}}}] \leavevmode
Specify the name of the pdf plot image output (\sphinxstylestrong{Default:}
‘hist.pdf’).
\sphinxstylestrong{Note:} You can suppress plotting entirely by specifying an empty string
as \sphinxstyleemphasis{plot\_output} (i.e. \sphinxcode{\sphinxupquote{\sphinxhyphen{}o \textquotesingle{}\textquotesingle{}}} or \sphinxcode{\sphinxupquote{\sphinxhyphen{}\sphinxhyphen{}plot\_output \textquotesingle{}\textquotesingle{}}})

\end{description}


\subparagraph{Additional output options}
\label{\detokenize{users_guide/command_line_tools/plothist:additional-output-options}}
Note: \sphinxcode{\sphinxupquote{plothist}} provides additional, optional arguments to output the
data points used to construct the plot:
\begin{description}
\item[{\sphinxstylestrong{\textasciigrave{}\textasciigrave{}\textendash{}hdf5\sphinxhyphen{}output ‘’hdf5\_output’’\textasciigrave{}\textasciigrave{}}}] \leavevmode
Output plot data \sphinxstyleemphasis{hdf5} file \sphinxcode{\sphinxupquote{\textquotesingle{}hdf5\_output\textquotesingle{}}} (\sphinxstylestrong{Default:} No
\sphinxstyleemphasis{hdf5} output file)

\item[{\sphinxstylestrong{\textasciigrave{}\textasciigrave{}\textendash{}text\sphinxhyphen{}output ‘’text\_output’’\textasciigrave{}\textasciigrave{}}}] \leavevmode
Output plot data as a text file named \sphinxcode{\sphinxupquote{\textquotesingle{}text\_output\textquotesingle{}}}
(\sphinxstylestrong{Default:} No text output file)
\sphinxstylestrong{Note:} This option is only available for 1 dimensional histogram
plots (that is, \sphinxcode{\sphinxupquote{\textquotesingle{}average\textquotesingle{}}} and \sphinxcode{\sphinxupquote{\textquotesingle{}instant\textquotesingle{}}} modes only)

\end{description}


\paragraph{Plotting options}
\label{\detokenize{users_guide/command_line_tools/plothist:plotting-options}}
The following options allow the user to specify a plot title, the type
of plot (i.e. energy or probability distribution), whether to apply a
log transformation to the data, and the range of data values to include.
\begin{description}
\item[{\sphinxstylestrong{\textasciigrave{}\textasciigrave{}\textendash{}title ‘’title’’ \textasciigrave{}\textasciigrave{}}}] \leavevmode
Optionally specify a title, \sphinxstyleemphasis{\textasciigrave{}\textasciigrave{}title\textasciigrave{}\textasciigrave{}}, for the plot (\sphinxstylestrong{Default:}
No title)

\item[{\sphinxstylestrong{\textasciigrave{}\textasciigrave{}\textendash{}range ‘’\textless{}nowiki\textgreater{}’\textless{}/nowiki\textgreater{}LB, UB\textless{}nowiki\textgreater{}’\textless{}/nowiki\textgreater{}’’\textasciigrave{}\textasciigrave{}}}] \leavevmode
Optionally specify the data range to be plotted as “\sphinxcode{\sphinxupquote{LB, UB}}”
(e.g. \sphinxcode{\sphinxupquote{\textquotesingle{} \sphinxhyphen{}\sphinxhyphen{}range "\sphinxhyphen{}1, 10" \textquotesingle{}}} \sphinxhyphen{} note that the quotation marks are
necessary if specifying a negative bound). For 1 dimensional
histograms, the range affects the y axis. For 2 dimensional plots
(e.g. evolution plot with 1 dimensional progress coordinate), it
corresponds to the range of the color bar

\end{description}


\subparagraph{Mutually exclusive plotting options}
\label{\detokenize{users_guide/command_line_tools/plothist:mutually-exclusive-plotting-options}}
The following three options determine how the plotted data is
represented (\sphinxstylestrong{Default:} \sphinxcode{\sphinxupquote{\textquotesingle{}\sphinxhyphen{}\sphinxhyphen{}energy\textquotesingle{}}})
\begin{description}
\item[{\sphinxstylestrong{\textasciigrave{}\textasciigrave{}\textendash{}energy \textasciigrave{}\textasciigrave{}}}] \leavevmode
Plots the probability distribution on an inverted natural log scale
(i.e. \sphinxhyphen{}ln{[}P(x){]} ), corresponding to the free energy (\sphinxstylestrong{Default})

\item[{\sphinxstylestrong{\textasciigrave{}\textasciigrave{}\textendash{}linear \textasciigrave{}\textasciigrave{}}}] \leavevmode
Plots the probability distribution function as a linear scale

\item[{\sphinxstylestrong{\textasciigrave{}\textasciigrave{}\textendash{}log10 \textasciigrave{}\textasciigrave{}}}] \leavevmode
Plots the (base\sphinxhyphen{}10) logarithm of the probability distribution

\end{description}


\paragraph{Iteration selection options}
\label{\detokenize{users_guide/command_line_tools/plothist:iteration-selection-options}}
Depending on plotting mode, you can select either a range or a single
iteration to plot.

\sphinxstylestrong{\textasciigrave{}\textasciigrave{}’instant’\textasciigrave{}\textasciigrave{}} mode only:
\begin{description}
\item[{\sphinxstylestrong{\textasciigrave{}\textasciigrave{}\textendash{}iter ‘’n\_iter’’ \textasciigrave{}\textasciigrave{}}}] \leavevmode
Plot the distribution for iteration \sphinxcode{\sphinxupquote{\textquotesingle{}\textquotesingle{}n\_iter\textquotesingle{}\textquotesingle{}}} (\sphinxstylestrong{Default:}
Last completed iteration)

\end{description}

\sphinxstylestrong{\textasciigrave{}\textasciigrave{}’average’\textasciigrave{}\textasciigrave{}} and \sphinxstylestrong{\textasciigrave{}\textasciigrave{}’evolution’\textasciigrave{}\textasciigrave{}} modes only:
\begin{description}
\item[{\sphinxstylestrong{\textasciigrave{}\textasciigrave{}\textendash{}first\sphinxhyphen{}iter ‘’first\_iter’’ \textasciigrave{}\textasciigrave{}}}] \leavevmode
Begin averaging or plotting at iteration \sphinxstyleemphasis{\textasciigrave{}\textasciigrave{}first\_iter\textasciigrave{}\textasciigrave{}}
(\sphinxstylestrong{Default:} 1)

\item[{\sphinxstylestrong{\textasciigrave{}\textasciigrave{}\textendash{}last\sphinxhyphen{}iter ‘’last\_iter’’ \textasciigrave{}\textasciigrave{}}}] \leavevmode
Average or plot up to and including \sphinxstyleemphasis{\textasciigrave{}\textasciigrave{}last\_iter\textasciigrave{}\textasciigrave{}} (\sphinxstylestrong{Default:}
Last completed iteration)

\end{description}

\sphinxstylestrong{\textasciigrave{}\textasciigrave{}’evolution’\textasciigrave{}\textasciigrave{}} mode only:
\begin{description}
\item[{\sphinxstylestrong{\textasciigrave{}\textasciigrave{}\textendash{}iter\_step ‘’n\_step’’ \textasciigrave{}\textasciigrave{}}}] \leavevmode
Average every \sphinxstyleemphasis{\textasciigrave{}\textasciigrave{}n\_step\textasciigrave{}\textasciigrave{}} iterations together when plotting in
\sphinxcode{\sphinxupquote{\textquotesingle{}evolution\textquotesingle{}}} mode (\sphinxstylestrong{Default:} 1 \sphinxhyphen{} i.e. plot each iteration)

\end{description}


\paragraph{Specifying progress coordinate dimension}
\label{\detokenize{users_guide/command_line_tools/plothist:specifying-progress-coordinate-dimension}}
For progress coordinates with dimensions greater than 1, you can specify
the dimension of the progress coordinate to use, the of progress
coordinate values to include, and the progress coordinate axis label
with a single positional argument:
\begin{description}
\item[{\sphinxstylestrong{\textasciigrave{}\textasciigrave{}dimension \textasciigrave{}\textasciigrave{}}}] \leavevmode
Specify \sphinxcode{\sphinxupquote{\textquotesingle{}dimension\textquotesingle{}}} as ‘\sphinxcode{\sphinxupquote{int{[}:{[}LB,UB{]}:label{]}}}‘, where
‘\sphinxcode{\sphinxupquote{int}}‘ specifies the dimension (starting at 0), and,
optionally, ‘\sphinxcode{\sphinxupquote{LB,UB}}‘ specifies the lower and upper range
bounds, and/or ‘\sphinxcode{\sphinxupquote{label}}‘ specifies the axis label (\sphinxstylestrong{Default:}
\sphinxcode{\sphinxupquote{int}} = 0, full range, default label is ‘dimension \sphinxcode{\sphinxupquote{int}}’; e.g
‘dimension 0’)

\end{description}

For \sphinxcode{\sphinxupquote{\textquotesingle{}average\textquotesingle{}}} and \sphinxcode{\sphinxupquote{\textquotesingle{}instant\textquotesingle{}}} modes, you can plot two dimensions
at once using a color map if this positional argument is specified:
\begin{description}
\item[{\sphinxstylestrong{\textasciigrave{}\textasciigrave{}addtl\_dimension \textasciigrave{}\textasciigrave{}}}] \leavevmode
Specify the other dimension to include as \sphinxcode{\sphinxupquote{\textquotesingle{}addtl\_dimension\textquotesingle{}}}

\end{description}


\subsubsection{Examples}
\label{\detokenize{users_guide/command_line_tools/plothist:examples}}
These examples assume the input file is created using w\_pcpdist and is
named ‘pcpdist.h5’


\paragraph{Basic plotting}
\label{\detokenize{users_guide/command_line_tools/plothist:basic-plotting}}
Plot the energy ( \sphinxhyphen{}ln(P(x)) ) for the last iteration

\sphinxcode{\sphinxupquote{\$WEST\_ROOT/bin/plothist instant pcpdist.h5}}

Plot the evolution of the log10 of the probability distribution over all
iterations

{\color{red}\bfseries{}\textasciigrave{}\textasciigrave{}}\$WEST\_ROOT/bin/plothist evolution pcpdist.h5 \textendash{}log10 \textasciigrave{}\textasciigrave{}

Plot the average linear probability distribution over all iterations

\sphinxcode{\sphinxupquote{\$WEST\_ROOT/bin/plothist average pcpdist.h5 \sphinxhyphen{}\sphinxhyphen{}linear}}


\paragraph{Specifying progress coordinate}
\label{\detokenize{users_guide/command_line_tools/plothist:specifying-progress-coordinate}}
Plot the average probability distribution as the energy, label the
x\sphinxhyphen{}axis ‘pcoord’, over the entire range of the progress coordinate

\sphinxcode{\sphinxupquote{\$WEST\_ROOT/bin/plothist average pcpdist.h5 0::pcoord}}

Same as above, but only plot the energies for with progress coordinate
between 0 and 10

\sphinxcode{\sphinxupquote{\$WEST\_ROOT/bin/plothist average pcpdist.h5 \textquotesingle{}0:0,10:pcoord\textquotesingle{}}}

(Note: the quotes are needed if specifying a range that includes a
negative bound)

(For a simulation that uses at least 2 progress coordinates) plot the
probability distribution for the 5th iteration, representing the first
two progress coordinates as a heatmap

\sphinxcode{\sphinxupquote{\$WEST\_ROOT/bin/plothist instant pcpdist.h5 0 1 \sphinxhyphen{}\sphinxhyphen{}iter 5 \sphinxhyphen{}\sphinxhyphen{}linear}}


\section{HDF5 File Schema}
\label{\detokenize{users_guide/hdf5:hdf5-file-schema}}\label{\detokenize{users_guide/hdf5::doc}}
WESTPA stores all of its simulation data in the cross\sphinxhyphen{}platform, self\sphinxhyphen{}describing
\sphinxhref{http://www.hdfgroup.org/HDF5}{HDF5} file format. This file format can be
read and written by a variety of languages and toolkits, including C/C++,
Fortran, Python, Java, and \sphinxhref{http://www.mathworks.com/help/matlab/ref/hdf5read.html}{Matlab} so that analysis of
weighted ensemble simulations is not tied to using the WESTPA framework. HDF5
files are organized like a filesystem, where arbitrarily\sphinxhyphen{}nested groups (i.e.
directories) are used to organize datasets (i.e. files). The excellent \sphinxhref{http://www.hdfgroup.org/hdf-java-html/hdfview/}{HDFView} program may be used to
explore WEST data files.

The canonical file format reference for a given version of the WEST code is
described in \sphinxhref{https://github.com/westpa/westpa/blob/master/src/west/data\_manager.py}{src/west/data\_manager.py}.


\subsection{Overall structure}
\label{\detokenize{users_guide/hdf5:overall-structure}}
\begin{sphinxVerbatim}[commandchars=\\\{\}]
\PYG{o}{/}
    \PYG{c+c1}{\PYGZsh{}ibstates/}
        \PYG{n}{index}
        \PYG{n}{naming}
            \PYG{n}{bstate\PYGZus{}index}
            \PYG{n}{bstate\PYGZus{}pcoord}
            \PYG{n}{istate\PYGZus{}index}
            \PYG{n}{istate\PYGZus{}pcoord}
    \PYG{c+c1}{\PYGZsh{}tstates/}
        \PYG{n}{index}
    \PYG{n}{bin\PYGZus{}topologies}\PYG{o}{/}
        \PYG{n}{index}
        \PYG{n}{pickles}
    \PYG{n}{iterations}\PYG{o}{/}
        \PYG{n}{iter\PYGZus{}XXXXXXXX}\PYG{o}{/}\PYGZbs{}\PYG{o}{|}\PYG{n}{iter\PYGZus{}XXXXXXXX}\PYG{o}{/}
            \PYG{n}{auxdata}\PYG{o}{/}
            \PYG{n}{bin\PYGZus{}target\PYGZus{}counts}
            \PYG{n}{ibstates}\PYG{o}{/}
                \PYG{n}{bstate\PYGZus{}index}
                \PYG{n}{bstate\PYGZus{}pcoord}
                \PYG{n}{istate\PYGZus{}index}
                \PYG{n}{istate\PYGZus{}pcoord}
            \PYG{n}{pcoord}
            \PYG{n}{seg\PYGZus{}index}
            \PYG{n}{wtgraph}
        \PYG{o}{.}\PYG{o}{.}\PYG{o}{.}
    \PYG{n}{summary}
\end{sphinxVerbatim}


\subsection{The root group (/)}
\label{\detokenize{users_guide/hdf5:the-root-group}}
The root of the WEST HDF5 file contains the following entries (where a
trailing “/” denotes a group):


\begin{savenotes}\sphinxattablestart
\centering
\begin{tabulary}{\linewidth}[t]{|T|T|T|}
\hline
\sphinxstyletheadfamily 
Name
&\sphinxstyletheadfamily 
Type
&\sphinxstyletheadfamily 
Description
\\
\hline
ibstates/
&
Group
&
Initial and basis states for this
simulation
\\
\hline
tstates/
&
Group
&
Target (recycling) states for this
simulation; may be empty
\\
\hline
bin\_topologies/
&
Group
&
Data pertaining to the binning scheme
used in each iteration
\\
\hline
iterations/
&
Group
&
Iteration data
\\
\hline
summary
&
Dataset (1\sphinxhyphen{}dimensional,
compound)
&
Summary data by iteration
\\
\hline
\end{tabulary}
\par
\sphinxattableend\end{savenotes}


\subsubsection{The iteration summary table (/summary)}
\label{\detokenize{users_guide/hdf5:the-iteration-summary-table-summary}}

\begin{savenotes}\sphinxattablestart
\centering
\begin{tabulary}{\linewidth}[t]{|T|T|}
\hline
\sphinxstyletheadfamily 
Field
&\sphinxstyletheadfamily 
Description
\\
\hline
n\_particles
&
the total number of walkers in this iteration
\\
\hline
norm
&
total probability, for stability monitoring
\\
\hline
min\_bin\_prob
&
smallest probability contained in a bin
\\
\hline
max\_bin\_prob
&
largest probability contained in a bin
\\
\hline
min\_seg\_prob
&
smallest probability carried by a walker
\\
\hline
max\_seg\_prob
&
largest probability carried by a walker
\\
\hline
cputime
&
total CPU time (in seconds) spent on propagation for this
iteration
\\
\hline
walltime
&
total wallclock time (in seconds) spent on this iteration
\\
\hline
binhash
&
a hex string identifying the binning used in this iteration
\\
\hline
\end{tabulary}
\par
\sphinxattableend\end{savenotes}


\subsection{Per iteration data (/iterations/iter\_XXXXXXXX)}
\label{\detokenize{users_guide/hdf5:per-iteration-data-iterations-iter-xxxxxxxx}}
Data for each iteration is stored in its own group, named according to the
iteration number and zero\sphinxhyphen{}padded out to 8 digits, as in
\sphinxcode{\sphinxupquote{/iterations/iter\_00000001}} for iteration 1. This is done solely for
convenience in dealing with the data in external utilities that sort output by
group name lexicographically. The field width is in fact configurable via the
\sphinxcode{\sphinxupquote{iter\_prec}} configuration entry under \sphinxcode{\sphinxupquote{data}} section of the WESTPA
configuration file.

The HDF5 group for each iteration contains the following elements:


\begin{savenotes}\sphinxattablestart
\centering
\begin{tabulary}{\linewidth}[t]{|T|T|T|}
\hline
\sphinxstyletheadfamily 
Name
&\sphinxstyletheadfamily 
Type
&\sphinxstyletheadfamily 
Description
\\
\hline
auxdata/
&
Group
&
All user\sphinxhyphen{}defined auxiliary data0
sets
\\
\hline
bin\_target\_counts
&
Dataset (1\sphinxhyphen{}dimensional)
&
The per\sphinxhyphen{}bin target count for the
iteration
\\
\hline
ibstates/
&
Group
&
Initial and basis state data for
the iteration
\\
\hline
pcoord
&
Dataset (3\sphinxhyphen{}dimensional)
&
Progress coordinate data for the
iteration stored as a (num of
segments, pcoord\_len, pcoord\_ndim)
array
\\
\hline
seg\_index
&
Dataset (1\sphinxhyphen{}dimensional,
compound)
&
Summary data for each segment
\\
\hline
wtgraph
&
Dataset (1\sphinxhyphen{}dimensional)
&\\
\hline
\end{tabulary}
\par
\sphinxattableend\end{savenotes}


\subsubsection{The segment summary table (/iterations/iter\_XXXXXXXX/seg\_index)}
\label{\detokenize{users_guide/hdf5:the-segment-summary-table-iterations-iter-xxxxxxxx-seg-index}}

\begin{savenotes}\sphinxattablestart
\centering
\begin{tabulary}{\linewidth}[t]{|T|T|}
\hline
\sphinxstyletheadfamily 
Field
&\sphinxstyletheadfamily 
Description
\\
\hline
weight
&
Segment weight
\\
\hline
parent\_id
&
Index of parent
\\
\hline
wtg\_n\_parents
&\\
\hline
wtg\_offset
&\\
\hline
cputime
&
Total cpu time required to run the segment
\\
\hline
walltime
&
Total walltime required to run the segment
\\
\hline
endpoint\_type
&\\
\hline
status
&\\
\hline
\end{tabulary}
\par
\sphinxattableend\end{savenotes}


\subsection{Bin Topologies group (/bin\_topologies)}
\label{\detokenize{users_guide/hdf5:bin-topologies-group-bin-topologies}}
Bin topologies used during a WE simulation are stored as a unique hash
identifier and a serialized \sphinxcode{\sphinxupquote{BinMapper}} object in \sphinxhref{http://docs.python.org/2/library/pickle.html}{python pickle} format. This group contains
two datasets:
\begin{itemize}
\item {} 
\sphinxcode{\sphinxupquote{index}}: Compound array containing the bin hash and pickle length

\item {} 
\sphinxcode{\sphinxupquote{pickle}}: The pickled \sphinxcode{\sphinxupquote{BinMapper}} objects for each unique mapper stored
in a (num unique mappers, max pickled size) array

\end{itemize}


\chapter{For Developers}
\label{\detokenize{sphinx_index:for-developers}}

\section{Overview}
\label{\detokenize{development/overview:overview}}\label{\detokenize{development/overview::doc}}

\section{Style Guide}
\label{\detokenize{development/style_guide:style-guide}}\label{\detokenize{development/style_guide::doc}}

\subsection{Preface}
\label{\detokenize{development/style_guide:preface}}
The WESTPA documentation should help the user to understand how WESTPA works
and how to use it. To aid in effective communication, a number of guidelines
appear below.

When writing in the WESTPA documentation, please be:
\begin{itemize}
\item {} 
Correct

\item {} 
Clear

\item {} 
Consistent

\item {} 
Concise

\end{itemize}

Articles in this documentation should follow the guidelines on this page.
However, there may be cases when following these guidelines will make an
article confusing: when in doubt, use your best judgment and ask for the
opinions of those around you.


\subsection{Style and Usage}
\label{\detokenize{development/style_guide:style-and-usage}}

\subsubsection{Acronyms and abbreviations}
\label{\detokenize{development/style_guide:acronyms-and-abbreviations}}\begin{itemize}
\item {} 
Software documentation often involves extensive use of acronyms and
abbreviations.

Acronym: A word formed from the initial letter or letters of each or most of
the parts of a compound term

Abbreviation: A shortened form of a written word or name that is used in
place of the full word or name

\item {} 
Define non\sphinxhyphen{}standard acronyms and abbreviations on their first use by using
the full\sphinxhyphen{}length term, followed by the acronym or abbreviation in parentheses.

A potential of mean force (PMF) diagram may aid the user in visuallizing the
energy landscape of the simulation.

\item {} 
Only use acronyms and abbreviations when they make an idea more clear than
spelling out the full term. Consider clarity from the point of view of a new
user who is intelligent but may have little experience with computers.

Correct: The WESTPA wiki supports HyperText Markup Language (HTML). For
example, the user may use HTML tags to give text special formatting. However,
be sure to test that the HTML tag gives the desired effect by previewing
edits before saving.

Avoid: The WESTPA wiki supports HyperText Markup Language. For example, the
user may use HyperText Markup Language tags to give text special formatting.
However, be sure to test that the HyperText Markup Language tag gives the
desired effect by previewing edits before saving.

Avoid: For each iter, make sure to return the pcoord and any auxdata.

\item {} 
Use all capital letters for abbreviating file types. File extensions should
be lowercase.

HDF5, PNG, MP4, GRO, XTC

west.h5, bound.png, unfolding.mp4, protein.gro, segment.xtc

\item {} 
Provide pronunciations for acronyms that may be difficult to sound out.

\item {} 
Do not use periods in acronyms and abbreviations except where it is
customary:

Correct: HTML, U.S.

Avoid: H.T.M.L., US

\end{itemize}


\subsubsection{Capitalization}
\label{\detokenize{development/style_guide:capitalization}}\begin{itemize}
\item {} 
Capitalize at the beginning of each sentence.

\item {} 
Do not capitalize after a semicolon.

\item {} 
Do not capitalize after a colon, unless multiple sentences follow the colon.

\item {} 
In this case, capitalize each sentence.

\item {} 
Preserve the capitalization of computer language elements (commands,

\item {} 
utilities, variables, modules, classes, and arguments).

\item {} 
Capitilize generic Python variables according to the

\item {} 
\sphinxhref{http://www.python.org/dev/peps/pep-0008/\#class-names}{PEP 0008 Python Style Guide}. For example,
generic class names should follow the \sphinxstyleemphasis{CapWords} convention, such as
\sphinxcode{\sphinxupquote{GenericClass}}.

\end{itemize}


\subsubsection{Contractions}
\label{\detokenize{development/style_guide:contractions}}\begin{itemize}
\item {} 
Do not use contractions. Contractions are a shortened version of word
characterized by the omission of internal letters.

Avoid: can’t, don’t, shouldn’t

\item {} 
Possessive nouns are not contractions. Use possessive nouns freely.

\end{itemize}


\subsubsection{Internationalization}
\label{\detokenize{development/style_guide:internationalization}}\begin{itemize}
\item {} 
Use short sentences (less than 25 words). Although we do not maintain
WESTPA documentation in languages other than English, some users may use
automatic translation programs. These programs function best with short
sentences.

\item {} 
Do not use technical terms where a common term would be equally or more
clear.

\item {} 
Use multiple simple sentences in place of a single complicated sentence.

\end{itemize}


\subsubsection{Italics}
\label{\detokenize{development/style_guide:italics}}\begin{itemize}
\item {} 
Use italics (surround the word with * * on each side) to highlight words
that are not part of a sentence’s normal grammer.

Correct: The word \sphinxstyleemphasis{istates} refers to the initial states that WESTPA uses to
begin trajectories.

\end{itemize}


\subsubsection{Non\sphinxhyphen{}English words}
\label{\detokenize{development/style_guide:non-english-words}}\begin{itemize}
\item {} 
Avoid Latin words and abbreviations.

Avoid: etc., et cetera, e.g., i.e.

\end{itemize}


\subsubsection{Specially formatted characters}
\label{\detokenize{development/style_guide:specially-formatted-characters}}\begin{itemize}
\item {} 
Never begin a sentence with a specially formatted character. This includes
abbreviations, variable names, and anything else this guide instructs to use
with special tags. Sentences may begin with \sphinxstyleemphasis{WESTPA}.

Correct: The program \sphinxcode{\sphinxupquote{ls}} allows the user to see the contents of a
directory.

Avoid: \sphinxcode{\sphinxupquote{ls}} allows the user to see the contents of a directory.

\item {} 
Use the word \sphinxstyleemphasis{and} rather than an \sphinxcode{\sphinxupquote{\&}} ampersand .

\item {} 
When a special character has a unique meaning to a program, first use the
character surrounded by \textasciigrave{}\textasciigrave{} tags and then spell it out.

Correct: Append an \sphinxcode{\sphinxupquote{\&}} ampersand to a command to let it run in the
background.

Avoid: Append an “\&” to a command… Append an \sphinxcode{\sphinxupquote{\&}} to a command… Append
an ampersand to a command…

\item {} 
There are many names for the \sphinxcode{\sphinxupquote{\#}} hash mark, including hash tag, number
sign, pound sign, and octothorpe. Refer to this symbol as a “hash mark”.

\end{itemize}


\subsubsection{Subject}
\label{\detokenize{development/style_guide:subject}}\begin{itemize}
\item {} 
Refer to the end WESTPA user as \sphinxstyleemphasis{the user} in software documentation.

Correct: The user should use the \sphinxcode{\sphinxupquote{processes}} work manager to run segments
in parallel on a single node.

\item {} 
Refer to the end WESTPA user as \sphinxstyleemphasis{you} in tutorials (you is the implied
subject of commands). It is also acceptable to use personal pronouns such as
\sphinxstyleemphasis{we} and \sphinxstyleemphasis{our}. Be consistent within the tutorial.

Correct: You should have two files in this directory, named \sphinxcode{\sphinxupquote{system.py}} and
\sphinxcode{\sphinxupquote{west.cfg}}.

\end{itemize}


\subsubsection{Tense}
\label{\detokenize{development/style_guide:tense}}\begin{itemize}
\item {} 
Use \sphinxstyleemphasis{should} to specify proper usage.

Correct: The user should run \sphinxcode{\sphinxupquote{w\_truncate \sphinxhyphen{}n \textless{}var\textgreater{}iter\textless{}/var\textgreater{}}} to remove
iterations after and including iter from the HDF5 file specified in the
WESTPA configuration file.

\item {} 
Use \sphinxstyleemphasis{will} to specify expected results and output.

Correct: WESTPA will create a HDF5 file when the user runs \sphinxcode{\sphinxupquote{w\_init}}.

\end{itemize}


\subsubsection{Voice}
\label{\detokenize{development/style_guide:voice}}\begin{itemize}
\item {} 
Use active voice. Passive voice can obscure a sentence and add unnecessary
words.

Correct: WESTPA will return an error if the sum of the weights of segments
does not equal one.

Avoid: An error will be returned if the sum of the weights of segments does
not equal one.

\end{itemize}


\subsubsection{Weighted ensemble}
\label{\detokenize{development/style_guide:weighted-ensemble}}\begin{itemize}
\item {} 
Refer to weighted ensemble in all lowercase, unless at the beginning of a
sentence. Do not hyphenate.

Correct: WESTPA is an implementation of the weighted ensemble algorithm.

Avoid: WESTPA is an implementation of the weighted\sphinxhyphen{}ensemble algorithm.

Avoid: WESTPA is an implementation of the Weighted Ensemble algorithm.

\end{itemize}


\subsubsection{WESTPA}
\label{\detokenize{development/style_guide:westpa}}\begin{itemize}
\item {} 
Refer to WESTPA in all capitals. Do not use bold, italics, or other special
formatting except when another guideline from this style guide applies.

Correct: Install the WESTPA software package.

\item {} 
The word \sphinxstyleemphasis{WESTPA} may refer to the software package or a entity of running
software.

Correct: WESTPA includes a number of analysis utilities.

Correct: WESTPA will return an error if the user does not supply a
configuration file.

\end{itemize}


\subsection{Computer Language Elements}
\label{\detokenize{development/style_guide:computer-language-elements}}

\subsubsection{Classes, modules, and libraries}
\label{\detokenize{development/style_guide:classes-modules-and-libraries}}\begin{itemize}
\item {} 
Display class names in fixed\sphinxhyphen{}width font using the \sphinxcode{\sphinxupquote{\textasciigrave{}\textasciigrave{}}} tag.

Correct: \sphinxcode{\sphinxupquote{WESTPropagator}}

Correct: The \sphinxcode{\sphinxupquote{numpy}} library provides access to various low\sphinxhyphen{}level
mathematical and scientific calculation routines.

\item {} 
Generic class names should be relevant to the properties of the class; do not
use \sphinxstyleemphasis{foo} or \sphinxstyleemphasis{bar}
\begin{quote}

\sphinxcode{\sphinxupquote{class UserDefinedBinMapper(RectilinearBinMapper)}}
\end{quote}

\end{itemize}


\subsubsection{Methods and commands}
\label{\detokenize{development/style_guide:methods-and-commands}}\begin{itemize}
\item {} 
Refer to a method by its name without parentheses, and without prepending
the name of its class. Display methods in fixed\sphinxhyphen{}width font using the \sphinxcode{\sphinxupquote{\textasciigrave{}\textasciigrave{}}}
tag.

Correct: the \sphinxcode{\sphinxupquote{arange}} method of the \sphinxcode{\sphinxupquote{numpy}} library

Avoid: the \sphinxcode{\sphinxupquote{arange()}} method of the \sphinxcode{\sphinxupquote{numpy}} library

Avoid: the \sphinxcode{\sphinxupquote{numpy.arange}} method

\item {} 
When referring to the arguments that a method expects, mention the method
without arguments first, and then use the method’s name followed by
parenthesis and arguments.

Correct: WESTPA calls the \sphinxcode{\sphinxupquote{assign}} method as assign(coords, mask=None,
output=None)

\item {} 
Never use a method or command as a verb.

Correct: Run \sphinxcode{\sphinxupquote{cd}} to change the current working directory.

Avoid: \sphinxcode{\sphinxupquote{cd}} into the main simulation directory.

\end{itemize}


\subsubsection{Programming languages}
\label{\detokenize{development/style_guide:programming-languages}}\begin{itemize}
\item {} 
Some programming languages are both a language and a command. When referring
to the language, capitalize the word and use standard font. When referring
to the command, preserve capitalization as it would appear in a terminal and
use the \sphinxcode{\sphinxupquote{\textasciigrave{}\textasciigrave{}}} tag.

Using WESTPA requires some knowledge of Python.

Run \sphinxcode{\sphinxupquote{python}} to launch an interactive session.

The Bash shell provides some handy capabilities, such as wildcard matching.

Use \sphinxcode{\sphinxupquote{bash}} to run \sphinxcode{\sphinxupquote{example.sh}}.

\end{itemize}


\subsubsection{Scripts}
\label{\detokenize{development/style_guide:scripts}}\begin{itemize}
\item {} 
Use the \sphinxcode{\sphinxupquote{.. code\sphinxhyphen{}block::}} directive for short scripts. Options are
available for some languages, such as \sphinxcode{\sphinxupquote{.. code\sphinxhyphen{}block:: bash}} and
\sphinxcode{\sphinxupquote{.. code\sphinxhyphen{}block:: python}}.

\end{itemize}

\begin{sphinxVerbatim}[commandchars=\\\{\}]
\PYG{c+ch}{\PYGZsh{}!/bin/bash}
\PYG{c+c1}{\PYGZsh{} This is a generic Bash script.}

\PYG{n+nv}{BASHVAR}\PYG{o}{=}\PYG{l+s+s2}{\PYGZdq{}Hello, world!\PYGZdq{}}
\PYG{n+nb}{echo} \PYG{n+nv}{\PYGZdl{}BASHVAR}
\end{sphinxVerbatim}

\begin{sphinxVerbatim}[commandchars=\\\{\}]
\PYG{c+ch}{\PYGZsh{}!/usr/bin/env python}
\PYG{c+c1}{\PYGZsh{} This is a generic Python script.}

\PYG{k}{def} \PYG{n+nf}{main}\PYG{p}{(}\PYG{p}{)}\PYG{p}{:}
    \PYG{n}{pythonstr} \PYG{o}{=} \PYG{l+s+s2}{\PYGZdq{}}\PYG{l+s+s2}{Hello, world!}\PYG{l+s+s2}{\PYGZdq{}}
    \PYG{n+nb}{print}\PYG{p}{(}\PYG{n}{pythonstr}\PYG{p}{)}
    \PYG{k}{return}
\PYG{k}{if} \PYG{n+nv+vm}{\PYGZus{}\PYGZus{}name\PYGZus{}\PYGZus{}} \PYG{o}{==} \PYG{l+s+s2}{\PYGZdq{}}\PYG{l+s+s2}{\PYGZus{}\PYGZus{}main\PYGZus{}\PYGZus{}}\PYG{l+s+s2}{\PYGZdq{}}\PYG{p}{:}
    \PYG{n}{main}\PYG{p}{(}\PYG{p}{)}
\end{sphinxVerbatim}
\begin{itemize}
\item {} 
Begin a code snippet with a \sphinxcode{\sphinxupquote{\#!}} \sphinxstyleemphasis{shebang} (yes, this is the real term),
followed by the usual path to a program. The line after the shebang should be
an ellipsis, followed by lines of code. Use \sphinxcode{\sphinxupquote{\#!/bin/bash}} for Bash scripts,
\sphinxcode{\sphinxupquote{\#!/bin/sh}} for generic shell scripts, and \sphinxcode{\sphinxupquote{\#!/usr/bin/env python}} for
Python scripts. For Python code snippets that are not a stand\sphinxhyphen{}alone script,
place any import commands between the shebang line and ellipsis.

\end{itemize}

\begin{sphinxVerbatim}[commandchars=\\\{\}]
\PYG{c+ch}{\PYGZsh{}!/usr/bin/env python}
\PYG{k+kn}{import} \PYG{n+nn}{numpy}
\PYG{o}{.}\PYG{o}{.}\PYG{o}{.}
\PYG{k}{def} \PYG{n+nf}{some\PYGZus{}function}\PYG{p}{(}\PYG{n}{generic\PYGZus{}vals}\PYG{p}{)}\PYG{p}{:}
    \PYG{k}{return} \PYG{l+m+mi}{1} \PYG{o}{+} \PYG{n}{numpy}\PYG{o}{.}\PYG{n}{mean}\PYG{p}{(}\PYG{n}{generic\PYGZus{}vals}\PYG{p}{)}
\end{sphinxVerbatim}
\begin{itemize}
\item {} 
Follow the \sphinxhref{http://www.python.org/dev/peps/pep-0008/\#class-names}{PEP 0008 Python Style Guide} for Python scripts.
\begin{itemize}
\item {} 
Indents are four spaces.

\item {} 
For comments, use the \sphinxcode{\sphinxupquote{\#}} hash mark followed by a single space, and
then the comment’s text.

\item {} 
Break lines after 80 characters.

\end{itemize}

\item {} 
For Bash scripts, consider following \sphinxhref{https://google-styleguide.googlecode.com/svn/trunk/shell.xml}{Google’s Shell Style Guide}

\end{itemize}
\begin{itemize}
\item {} 
Indents are two spaces.

\item {} 
Use blank lines to improve readability

\item {} 
Use \sphinxcode{\sphinxupquote{; do}} and \sphinxcode{\sphinxupquote{; then}} on the same line as \sphinxcode{\sphinxupquote{while}}, \sphinxcode{\sphinxupquote{for}}, and
\sphinxcode{\sphinxupquote{if}}.

\item {} 
Break lines after 80 characters.

\end{itemize}
\begin{itemize}
\item {} 
For other languages, consider following a logical style guide. At minimum, be
consistent.

\end{itemize}


\subsubsection{Variables}
\label{\detokenize{development/style_guide:variables}}\begin{itemize}
\item {} 
Use the fixed\sphinxhyphen{}width \sphinxcode{\sphinxupquote{\textasciigrave{}\textasciigrave{}}} tag when referring to a variable.

the \sphinxcode{\sphinxupquote{ndim}} attribute

\item {} 
When explicitly referring to an attribute as well as its class, refer to an
attribute as: the \sphinxcode{\sphinxupquote{attr}} attribute of \sphinxcode{\sphinxupquote{GenericClass}}, rather than
\sphinxcode{\sphinxupquote{GenericClass.attr}}

\item {} 
Use the \sphinxcode{\sphinxupquote{\$}} dollar sign before Bash variables.

WESTPA makes the variable \sphinxcode{\sphinxupquote{\$WEST\_BSTATE\_DATA\_REF}} available to new
trajectories.

\end{itemize}


\section{Source Code Management}
\label{\detokenize{development/source_code:source-code-management}}\label{\detokenize{development/source_code::doc}}

\section{Documentation Practices}
\label{\detokenize{development/documentation:documentation-practices}}\label{\detokenize{development/documentation::doc}}
Documentation for WESTPA is maintained using \sphinxhref{http://sphinx-doc.org/}{Sphinx}
Docstrings are formatted in the \sphinxhref{https://github.com/numpy/numpy/blob/master/doc/HOWTO\_DOCUMENT.rst.txt}{Numpy style},
which are converted to ReStructuredText using Sphinx’ \sphinxhref{http://sphinxcontrib-napoleon.readthedocs.org/en/latest/}{Napoleon} plugin, which is
included with Sphinx 1.3.

The documentation may be built by navigating to the \sphinxcode{\sphinxupquote{doc}} folder, and
running:

\begin{sphinxVerbatim}[commandchars=\\\{\}]
\PYG{n}{make} \PYG{n}{html}
\end{sphinxVerbatim}

to prepare an html version or:

\begin{sphinxVerbatim}[commandchars=\\\{\}]
\PYG{n}{make} \PYG{n}{latexpdf}
\end{sphinxVerbatim}

To prepare a pdf. The latter requires \sphinxcode{\sphinxupquote{latex}} to be available.

A quick command to update the documentation in gh\sphinxhyphen{}pages repo is also available:

\begin{sphinxVerbatim}[commandchars=\\\{\}]
\PYG{n}{make} \PYG{n}{ghpages}
\end{sphinxVerbatim}

This command will run Sphinx html command and change the htmls to fit with the gh\sphinxhyphen{}pages
format, it also runs:

\begin{sphinxVerbatim}[commandchars=\\\{\}]
\PYG{n}{git} \PYG{n}{checkout} \PYG{n}{gh}\PYG{o}{\PYGZhy{}}\PYG{n}{pages}
\PYG{n}{git} \PYG{n}{commit} \PYG{o}{\PYGZhy{}}\PYG{n}{a}
\PYG{n}{git} \PYG{n}{push}
\end{sphinxVerbatim}

for you. Also note that this will change the current branch you are at to gh\sphinxhyphen{}pages
branch. It also leaves behind a doc/\_build folder that is no longer useful. Once you run
ghpages command I suggest going up a folder and removing the unnecessary doc folder
that is there by:

\begin{sphinxVerbatim}[commandchars=\\\{\}]
\PYG{n}{cd} \PYG{o}{.}\PYG{o}{.}\PYG{o}{/}
\PYG{n}{rm} \PYG{o}{\PYGZhy{}}\PYG{n}{r} \PYG{n}{doc}
\end{sphinxVerbatim}

Remeber to make sure you are indeed in gh\sphinxhyphen{}pages branch, this branch is not supposed to have
a folder named doc. Sometimes if you are not careful git checkout fails and you might end up
removing the folder you were working with if you are not careful.


\section{WESTPA Modules API}
\label{\detokenize{development/api:westpa-modules-api}}\label{\detokenize{development/api::doc}}

\subsection{Binning}
\label{\detokenize{development/api:binning}}

\subsection{YAMLCFG}
\label{\detokenize{development/api:yamlcfg}}

\subsection{RC}
\label{\detokenize{development/api:rc}}

\section{WESTPA Tools}
\label{\detokenize{development/api:westpa-tools}}


\renewcommand{\indexname}{Index}
\printindex
\end{document}